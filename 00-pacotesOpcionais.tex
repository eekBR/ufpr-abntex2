\usepackage{color}		    	% Controle das cores
\usepackage{graphicx}			% Inclusão de gráficos
\usepackage{lipsum}				% para geração de dummy text
\usepackage{csquotes}

%\usepackage[style=long]{glossaries}
%\usepackage{abntex2glossaries}

\usepackage{cancel} 		% permite representar o cancelamento de termos em texto ou equacoes	
\usepackage{xcolor} 		% cores extendidas	
\usepackage{smartdiagram}   	% gera diagramas a partir de listas
%\usepackage{float} 		% Para a figura ficar na posição correta	    
\usepackage{textcomp} 		% supporte para fontes da Text Companion 
\usepackage{longtable}		% uso de longtable

\usepackage{lscape}		% páginas em paisagem
\usepackage{multicol}		% mescla de colunas em tabelas
\usepackage{multirow}		% mescla de linhas em tabelas
\usepackage{newfloat} 		% criação do indice de quadros
%\usepackage{caption} 		% configura legenda 
%[format=plain]
%\renewcommand\caption[1]{%
	%\captionsetup{font=small}	% tamanho da fonte 10pt
	%,format=hang
	% \caption{#1}}
%\captionsetup{width=0.8\textwidth}
\captiondelim{-- }
\captiontitlefont{\small}
\captionnamefont{\small}


% uso do tikz e pgfplots
% ----------------------------------------------------------
%\usetikzlibrary{external}
\usetikzlibrary{arrows,calc,patterns,angles,quotes}
\usepackage{pgfplots}
\pgfplotsset{compat=1.15}


% Define o comando para citação de fontes em elementos gráficos (figuras, imagens,...).
% ----------------------------------------------------------
%  AUTOR(ano)
%
% parâmetro é a bibkey da fonte

\newcommand{\citefig}[2]{~\Citeauthor*{#1}\citeyear{#1}}

% Define os operadores matemáticos em portugues
% ----------------------------------------------------------
%

\DeclareMathOperator{\tr}{tr}
\DeclareMathOperator{\sen}{sen}
\DeclareMathOperator{\senh}{senh}
%\DeclareMathOperator{\tag}{tag}
\DeclareMathOperator{\tg}{tg}
\DeclareMathOperator{\tagh}{tagh}
\DeclareMathOperator{\tgh}{tgh}
\DeclareMathOperator{\cossec}{cossec}
%\DeclareMathOperator{\sen}{sen}

% Para fazer a listagem de codigos LaTeX na documentação
% ----------------------------------------------------------
\usepackage{listings}

% Comando para fazer 
%    a citação de documentos não publicados e informais e 
%    colocar as referências nas notas de rodapé
% ----------------------------------------------------------

\newcommand{\citenp}[1]{
	\cite{#1}\footnote{\fullcite{#1}}}

\newcommand{\textcitenp}[1]{
	\textcite{#1}\footnote{\fullcite{#1}}}


% para comentarios e sugestões
% ----------------------------------------------------------
\newcommand{\sugest}[1]{\textcolor{red!40}{#1}}


% simplificação para colocar figuras
% ----------------------------------------------------------
%   Parametros
%    1 caption
%    2 percent textwidth
%    3 arquivo da figura
%    4 fonte
%    5 fig:label
%    6 nota
%    7 legenda

\newcommand{\figura}[7]{
  {\centering
   \footnotesize
   \begin{figure}[!ht]
   \centering
        \caption{\uppercase{#1}}
        \includegraphics[width=#2\textwidth]{#3}
        \label{fig:#5}
    
    %ajustado p/ a largura da imagem
    \begin{minipage}{#2\textwidth} 
        \vspace{2mm}\centering
        %\begin{flushleft}
            \par FONTE:~#4
            \ifthenelse{\equal{#6}{}}{}
            { \par\hangindent=14mm NOTA: #6 }
            
            \ifthenelse{\equal{#7}{}}{}
            { \par\hangindent=14mm LEGENDA: #7 }
        %\end{flushleft}
    \end{minipage}
    \end{figure}
    }
}


% simplificação para colocar figuras cortadas
% ----------------------------------------------------------
%   Parametros
%    1 caption
%    2 percent textwidth
%    3 arquivo da figura
%    4 fonte
%    5 fig:label
%    6 nota
%    7 legenda
%    8 recorte esquerda direita em mm
%    9 recorte inferior superior em mm

\newcommand{\figurac}[9]{
 \centering\footnotesize
    \begin{figure}[!ht]
    {\centering
        \caption{\uppercase{#1}}
        \includegraphics[width=#2\textwidth, trim={#8mm #9 #8 #9},clip]{#3}
        \label{fig:#5}
        
    %ajustado p/ a largura da imagem
    \begin{minipage}{#2\textwidth} 
        \vspace{2mm}\centering
        %\begin{flushleft}
            \par FONTE:~#4
            \ifthenelse{\equal{#6}{}}{}
            {\par \hangindent=14mm NOTA: #6 }
          
            \ifthenelse{\equal{#7}{}}{}
            {\par \hangindent=14mm   LEGENDA: #7 }
        %\end{flushleft}
    \end{minipage}
    }
    \end{figure}
 
}

% simplificação para colocar imagens
% ----------------------------------------------------------
%   Parametros
%    1 caption
%    2 tabela
%    3 fonte
%    4 tab:label
%    5 nota
%    6 legenda

\newsavebox{\myboxa}
\newlength{\myboxlena}

\newcommand{\imagem}[6]{
	{
	\centering\footnotesize
  	    \sbox{\myboxa}{#2}
		\settowidth{\myboxlena}{\usebox{\myboxa}}
		\noindent
		\begin{figure}[!ht]
			\centering
			\caption{\uppercase{#1}}
			#2
			\label{fig:#4}
			
			%ajustado p/ a largura da imagem
			\begin{minipage}{\myboxlena} \footnotesize
				\vspace{2mm}\centering
				%\begin{flushleft}
				\par FONTE:~#3
				\ifthenelse{\equal{#5}{}}{}
				{ \par\hangindent=14mm NOTA: #5 }
				
				\ifthenelse{\equal{#6}{}}{}
				{ \par\hangindent=14mm LEGENDA: #6 }
				%\end{flushleft}
			\end{minipage}
		\end{figure}
	}
}

% simplificação para colocar tabelas
% ----------------------------------------------------------
%   Parametros
%    1 caption
%    2 tabela
%    3 fonte
%    4 tab:label
%    5 nota
%    6 legenda

\newsavebox{\mybox}
\newlength{\myboxlen}

\newcommand{\tabela}[6]
{
  \sbox{\mybox}{#2}
  \settowidth{\myboxlen}{\usebox{\mybox}}
  \noindent
  %\rule{\myboxlen}{1pt}\\
  \begin{table}[!ht]
    \centering\footnotesize
    \par\caption{\uppercase{#1}}
    
    \par  #2
 
    \label{tab:#4}
 
     \begin{minipage}{\myboxlen} 
        \vspace{2mm}\footnotesize
        %\begin{flushleft}
            FONTE:~ #3
            
            \ifthenelse{\equal{#5}{}}{}
            {\hangindent=14mm NOTA: #5 }
            
            \ifthenelse{\equal{#6}{}}{}
            {\hangindent=14mm LEGENDA: #6}
        %\end{flushleft}
    \end{minipage}
\end{table}
  % \centering
  % \usebox{\mybox}
  % box size: \the\myboxlen.
}

% Ambiente e lista de quadros
% ----------------------------------------------------------
\newfloat{quadro}{quadros}{QUADRO}
\newlistof{listofquadros}{quadros}{Lista de Quadros}
\newlistentry{quadro}{quadros}{0}
\cftsetindents{quadro}{0mm}{13mm}

%\PrepareListOf{quadro}{%
\renewcommand{\cftquadropresnum}{\normalsize{QUADRO}~}
\setlength{\cftquadronumwidth}{28mm}
%\renewcommand{\cftquadroname}{\quadroname\space} 
\renewcommand*{\cftquadroaftersnum}{\hfill--\hfill}
%}

\newsavebox{\myboxq}
\newlength{\myboxlenq}

% ----------------------------------------------------------
%   Parametros
%    1 caption
%    2 tabela/quadro
%    3 fonte
%    4 qua:label
%    5 nota
%    6 legenda

\newcommand{\qquadro}[6]
  {
   
   %\begin{center}
   \sbox{\myboxq}{#2}
   \settowidth{\myboxlenq}{\usebox{\myboxq}}
   \begin{quadro} \footnotesize
        \caption{\uppercase{#1}}
	\label{qua:#4}
        \noindent 
        
        #2 %placed inside the environment
	
   			
        %ajustado p/ a largura da tabela (quadro) se menor que 30mm a largura é ajustada para .5\linewidth
	 
        \ifdimcomp {\myboxlenq}{<}{30mm}{\settowidth{\myboxlenq}{\hspace{.5\linewidth}}}{}
    	
	\begin{minipage}{\myboxlenq} \footnotesize
    	    \vspace{2mm}\centering
    		%\begin{flushleft}
    		\par FONTE:~#3
    		\ifthenelse{\equal{#5}{}}{}
    		{ \par\hangindent=14mm NOTA: #5 }
    				
    		\ifthenelse{\equal{#6}{}}{}
    		{ \par\hangindent=14mm LEGENDA: #6 }
    		%\end{flushleft}
    	\end{minipage}
    %\end{center}
    \label{qua:#4}
  \end{quadro}
  } 
