\usepackage{color}		    	% Controle das cores
\usepackage{graphicx}			% Inclusão de gráficos
\usepackage{lipsum}				% para geração de dummy text
\usepackage{csquotes}

%\usepackage[style=long]{glossaries}
%\usepackage{abntex2glossaries}

\usepackage{cancel} 		% permite representar o cancelamento de termos em texto ou equacoes	
\usepackage{xcolor} 		% cores extendidas	
\usepackage{smartdiagram}   	% gera diagramas a partir de listas
%\usepackage{float} 		% Para a figura ficar na posição correta	    
\usepackage{textcomp} 		% supporte para fontes da Text Companion 
\usepackage{longtable}		% uso de longtable

\usepackage{lscape}		% páginas em paisagem
\usepackage{multicol}		% mescla de colunas em tabelas
\usepackage{multirow}		% mescla de linhas em tabelas
\usepackage{newfloat} 		% criação do indice de quadros
%\usepackage{caption} 		% configura legenda 
%[format=plain]
%\renewcommand\caption[1]{%
	%\captionsetup{font=small}	% tamanho da fonte 10pt
	%,format=hang
	% \caption{#1}}
%\captionsetup{width=0.8\textwidth}
\captiondelim{-- }
\captiontitlefont{\small}
\captionnamefont{\small}


% uso do tikz e pgfplots
% ----------------------------------------------------------
%\usetikzlibrary{external}
\usetikzlibrary{arrows,calc,patterns,angles,quotes}
\usepackage{pgfplots}
\pgfplotsset{compat=1.15}


% Define o comando para citação de fontes em elementos gráficos (figuras, imagens,...).
% ----------------------------------------------------------
%  AUTOR(ano)
%
% parâmetro é a bibkey da fonte

\newcommand{\citefig}[2]{~\Citeauthor*{#1}\citeyear{#1}}

% Define os operadores matemáticos em portugues
% ----------------------------------------------------------
%

\DeclareMathOperator{\tr}{tr}
\DeclareMathOperator{\sen}{sen}
\DeclareMathOperator{\senh}{senh}
%\DeclareMathOperator{\tag}{tag}
\DeclareMathOperator{\tg}{tg}
\DeclareMathOperator{\tagh}{tagh}
\DeclareMathOperator{\tgh}{tgh}
\DeclareMathOperator{\cossec}{cossec}
%\DeclareMathOperator{\sen}{sen}

% Para fazer a listagem de codigos LaTeX na documentação
% ----------------------------------------------------------
\usepackage{listings}

% Comando para fazer 
%    a citação de documentos não publicados e informais e 
%    colocar as referências nas notas de rodapé
% ----------------------------------------------------------

\newcommand{\citenp}[1]{
	\cite{#1}\footnote{\fullcite{#1}}}

\newcommand{\textcitenp}[1]{
	\textcite{#1}\footnote{\fullcite{#1}}}
