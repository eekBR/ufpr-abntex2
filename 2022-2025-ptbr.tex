\newpage
{\Huge Regras Oficiais de Softbol

 2022-2025}
\clearpage
.

\vfill
\begin{center}
	\begin{minipage}{.6\textwidth}


	{\centering\large{Dedicatória}}

	 \vspace{10mm}
	 Este livro de regras é dedicado a todos os Árbitros do Brasil que, apesar da barreira  da língua e dificuldades diversas, cedem um considerado tempo de suas vidas para a elevação e firmação deste tão intrigante esporte no nosso país.

	 Aos irmãos da ASB nos nomes do nosso atual presidente, Carlos OBA e do nosso sensei Fernando Matsumori, que sempre nos ensina e tira nossas dúvidas, aos companheiros da CBBS que sempre nos apoiam e também aos companheiros da WBSC, que não economizam esforços para apoiar a nossa capacitação técnica.

	 Mais  importante  ainda, dedicado á colônia nipo-brasileira que introduziu este esporte no país e, de certa forma, o mantém vivo por tantos anos.

\vspace{10mm}
\begin{CJK}{UTF8}{min}	励ましありがとうございます\end{CJK}
	\end{minipage}
\end{center}
\vfill.

\clearpage
\dominitoc% Initialization
\tableofcontents*

\newpage
\vfill
\begin{center}
	\begin{minipage}{.80\textwidth}
		\setlength{\parskip}{3mm}
		 Este Livro de Regras contém a tradução livre do Livro de Regras \textit{Official Rules  of Fast Pitch Softball} da WBSC - \textit{World Baseball Softball Confederation},  antigamente denominada de ISF - \textit{International Softball Federation}.

		 Livro de Regras Oficiais inclui as Regras, os Efeitos e os Apêndices. Estas  Regras versam sobre o Softball com arremesso rápido e o Softball com arremesso rápido modificado.

		 Os Apêndices e Efeitos formam parte das Regras nos quais estes são citados  e têm a mesma força e efeito que as Regras propriamente ditas. A Tabela de Regras, onde as Regras também podem ser achadas são apenas para  referência rápida e servem como palavras chaves para facilitar o entendimento.

		 A tradução livre para o português foi feita e revisada por uma equipe da ASB - Árbitros de Softball do Brasil - Entidade ligada à CBBS - Confederação Brasileira  de Baseball e Softball com revisão ortográfica realizada pela mãe e colaboradora de Curitiba: Dra. Priscilla Mayumi Bansho.

		 Novas regras e/ou mudanças estão destacadas em negrito com destaque em fundo cinza neste livro.

		 Apêndice 7 (mudanças de Regras) não foi traduzido por estar contido na integra dentro do manual, evitando redundâncias desnecessárias.

		 Alguns termos foram deixados em inglês quando a equipe de tradutores achou que seria benéfico e usual para o leitor.

		 Créditos aos revisores da ASB: Carlos Oba, Fernando Matsumori, Mario Yoshida, Nelson Yajima, Patricia Hamamoto, Armando Kunitake, Michel Uehara, Jaime Barbosa, Jorge  Afuso, Fernando Silva e  Cesar  Calderaro.

%Foto: Credito ao Gustavo ( Guto )
 	\end{minipage}
\end{center}
\vfill.

\chapter{O JOGO}

\minitoc% Creating an actual minitoc

\section{DEFINIÇÕES}

\subsection{APELAÇÃO -- BOLA VIVA ou BOLA MORTA}
 Uma apelação com bola viva ou bola morta é uma jogada ou situação em que um árbitro não pode dar uma decisão, a menos que seja solicitada por um  técnico, \gls{coach} ou jogador da equipe não infratora.

\subsection{EQUIPE NA DEFENSIVA}
 É a equipe cujos jogadores estão ocupando suas posições dentro do campo.
\subsection{CONFISCO DE JOGO}

Quando o árbitro de \gls{home} encerra o jogo, declarando vencedora a equipe não infratora.

\subsection{EQUIPE LOCAL E EQUIPE VISITANTE}
	\begin{enumerate}[label=(\alph*)]
		\item A equipe local pode ser definida de várias maneiras, incluindo sorteio com  moeda (\gls{coin toss}), acordo mútuo, indicação da competição ou indicação da  Liga.
		\item A equipe local inicia o jogo na defensiva, bate (ataca) na segunda metade do \gls{inning} e ocupa o \gls{dugout} do lado da terceira base.
		\item A equipe visitante inicia o jogo na ofensiva, bate (ataca) na primeira metade  de cada \gls{inning} e ocupa o \gls{dugout} do lado da primeira base.
	\end{enumerate}

\subsection{\textit{INNING}}
 É aquela parte de um jogo durante a qual ambas as equipes atacam ou defendem, e continuam atacando ou defendendo até que ocorram três eliminações. A nova metade de \gls{inning} inicia imediatamente após a última  eliminação da metade de \gls{inning} anterior.

\subsection{EQUIPE NA OFENSIVA}
 É a equipe que está batendo (atacando).

\subsection{\textit{PLAY BALL}}
 Para iniciar ou reiniciar um jogo, o árbitro de \gls{home} deve sinalizar que a bola
 está viva e declarar \gls{play ball}, desde que:
	 \begin{enumerate}[label=(\alph*)]
		\item  o arremessador esteja segurando a bola dentro do Círculo do Arremessador;
		\item  o receptor esteja dentro do \gls{catcher's box} e todos os outros  defensores estejam em território \gls{fair}.
	\end{enumerate}

\subsection{REUNIÃO PRÉ-JOGO}

É a reunião na área do \gls{homeplate}, na hora predeterminada, incluindo o árbitro, \glspl{coach} principais, técnicos ou representantes de ambas as equipes. Nessa  reunião, os Formulários de Escalação da Equipe (\gls{lineup}) são confirmados e trocados entre as equipes, e o árbitro de \gls{home} revê algumas regras especiais.

\subsection{PROTESTO}

É a ação de uma equipe na defensiva ou ofensiva, exceto uma apelação, para  contestar:

\begin{enumerate}[label=(\alph*)]
\item  a má interpretação ou aplicação de uma regra de jogo por um árbitro; ou
\item  a elegibilidade de um membro da lista da equipe adversária.
\end{enumerate}

\subsection{\textit{TIME} (TEMPO)}
 É o termo usado por um árbitro para ordenar a paralisação de jogada num jogo.

 Durante a paralisação de jogada a bola está morta.

\section{JOGO REGULAMENTAR -- REQUISITOS}
\subsection{JOGO REGULAMENTAR}
  Um Jogo Regulamentar consiste de sete \glspl{inning} completos, exceto nas situações a seguir.
\begin{enumerate}[label=(\alph*)]
	\item Não é necessário jogar os sete \glspl{inning} completos se a equipe local anota mais pontos em seis \glspl{inning} ou antes do terceiro \gls{out} na segunda metade do sétimo \gls{inning}.
	\item Um jogo que está empatado depois de jogados sete \glspl{inning} deve ser prorrogado jogando \glspl{inning} adicionais até que uma das equipes anote mais pontos do que a outra no fim de um \gls{inning} completo, ou até que a equipe local obtenha vantagem no placar em sua metade de \gls{inning}, antes de ser completada o terceiro \gls{out}.
	\item Um jogo encerrado pelo árbitro antes do sétimo \gls{inning} será um Jogo Regulamentar se cinco ou mais \glspl{inning} completos tiverem sido jogados, ou se a equipe local tiver anotado mais pontos do que os anotados pela  outra equipe em cinco ou mais \glspl{inning}, ou se for aplicada a Regra de Vantagem de Pontos (\gls{runaheadrule}).

	O árbitro tem poderes para encerrar um jogo a  qualquer momento por escuridão, chuva, incêndio, pânico ou outra causa que ponha os espectadores ou membros da equipe em risco.
 	\item Será declarado um Jogo Regulamentar empatado se o placar estiver igual quando a partida for encerrada depois de jogados cinco ou mais \glspl{inning} completos, ou se a equipe local tiver anotado o mesmo número de pontos da  equipe visitante no \gls{inning} incompleto.
 	\item Estas disposições não se aplicam a quaisquer atos da parte de jogadores e espectadores que possam dar motivo para o confisco do jogo. O árbitro de \gls{home} pode confiscar o jogo se qualquer membro de uma equipe ou espectador atacar fisicamente qualquer árbitro.
	\item Um jogo que não é considerado um Jogo Regulamentar ou é um jogo Regulamentar Empatado será jogado novamente desde o início. A escalação original pode ser alterada para o novo jogo.
\end{enumerate}

\subsection{JOGO CONFISCADO}

Um jogo é confiscado em favor da equipe não infratora quando:
\begin{enumerate}[label=(\alph*)]
	\item uma equipe não comparece ao campo;
	\item  uma equipe que está no campo se recusa a iniciar um jogo para o qual está escalada ou designada, no horário marcado -- ou dentro de um tempo estabelecido para confisco de jogo -- pelo Regulamento da competição em que ela está jogando;
	\item  após iniciado o jogo, uma equipe se recusa a continuar jogando, a menos que jogo tenha sido interrompido ou encerrado pelo árbitro de \gls{home};
	\item  depois que o árbitro de \gls{home} paralisa o jogo, uma equipe não reinicia a  partida dentro de dois minutos após o árbitro sinalizar e declarar \gls{play ball};
	\item  uma equipe emprega táticas destinadas a retardar ou acelerar o jogo;
	\item  após uma advertência do árbitro principal, qualquer uma destas regras é violada propositadamente, exceto se o arremessador continua infringindo reiteradamente a regra sobre arremesso. Nesse caso, ele será removido da  posição de arremessador para o resto do jogo e será delarado um arremessador ilegal;
	\item  a ordem para remover ou expulsar um jogador ou qualquer pessoa autorizada  a permanecer no \gls{bench} da equipe não é obedecida dentro de um minuto;
	\item  por causa da remoção ou expulsão de jogador(es) do jogo pelo árbitro, ou por qualquer outro motivo, há menos de nove jogadores (ou dez quando utilizando um JD e um FLEX disponível no \gls{lineup}) em qualquer das duas equipes;
	\item  Um jogador declarado inelegível retorna ao jogo e o arremessador efetua um arremesso; ou
	\item  é descoberto que um jogador, \gls{coach} ou técnico expulso está participando do jogo outra vez.
\end{enumerate}

\subsection{REGRA DE VANTAGEM DE PONTOS}
\begin{enumerate}[label=(\alph*)]
	\item Aplica-se a qualquer jogo de todos os Torneios e Campeonatos quando uma  equipe está liderando o placar por:
	\begin{itemize}
		\item 15 pontos depois de três (3) \glspl{inning};
		\item 10 pontos depois de quatro (4) \glspl{inning}; ou
		\item 7 pontos depois de cinco (5) \glspl{inning}.
	\end{itemize}
	\item São jogados \glspl{inning} completos, a menos que a equipe local anote a  quantidade necessária de pontos enquanto estão na ofensiva. Quando a equipe visitante alcança a quantidade necessária de pontos na primeira metade de \gls{inning}, a equipe local bate na segunda metade do \gls{inning}. Toda a jogada tem de estar concluída antes do jogo ser declarado vencido pela contagem da Regra de Vantagem de Pontos.

	Na segunda metade do \gls{inning}, nenhum tento acima  da contagem necessária para aplicar a Regra de Vantagem de Pontos deve ser contado, a menos que seja batido um \gls{homerun}, e nesse caso todos os pontos anotados são contados.
\end{enumerate}

\subsection{REGRA DE DESEMPATE}
\begin{enumerate}[label=(\alph*)]
	\item A partir da primeira metade do oitavo \gls{inning} e da metade de cada \gls{inning} subsequente até o término do jogo, a equipe na ofensiva inicia a sua vez de bater com o jogador a quem cabe bater em nono \footnote{NT: último batedor a voltar a bater no momento} nessa respectiva metade de \gls{inning} colocado na segunda base como um corredor.
	\item  O corredor colocado na segunda base pode ser substituído de acordo com as regras de substituição.
	\item  Se um corredor incorreto for colocado na segunda base, ele será eliminado se uma apelação apropriada for feita pela equipe defensiva após um arremesso (legal ou ilegal). Uma apelação apropriada pode ser feita a  qualquer momento que o corredor incorreto estiver na base. Se a equipe de ataque colocar o corredor correto na base antes que uma apelação apropriada seja feita, não há penalidade.
\end{enumerate}


\subsection{PONTOS ANOTADOS}
\begin{enumerate}[label=(\alph*)]
	\item É anotado um ponto cada vez que um corredor toca, em ordem, as três bases e o \gls{homeplate}, antes do terceiro \gls{out} da metade de \gls{inning}.
\item  Quando se usa a regra de desempate, o corredor colocado na segunda base não precisa tocar a primeira base para que um ponto seja anotado legalmente.
\item  Não é anotado um ponto se a terceira e/ou última eliminação do \gls{inning} resulta de:
	 \begin{enumerate}[label=\roman*.]
	\item uma jogada em que o batedor-corredor é declarado \gls{out} antes de tocar a  primeira base;
	\item  uma \gls{jogadaforcada}, inclusive numa Jogada de Apelação;
	\item  uma jogada em que o corredor deixa uma base antes de ser efetuado um arremesso; ou
	\item  uma jogada em que um corredor precedente é declarado \gls{out}.
	\end{enumerate}
\item  Podem ser feitas apelações para uma eliminação adicional, depois do terceiro \gls{out}, para invalidar ponto(s).

\end{enumerate}
\subsection{JOGADAS DE APELAÇÃO}

Numa Jogada de Apelação, o corredor será declarado \gls{out} somente se a  apelação for feita legalmente.

\begin{enumerate}[label=(\alph*)]
	\item Uma apelação pode ser feita enquanto a bola está viva ou morta, mas a equipe na defensiva perde a oportunidade de apelar se não se manifestar:
	 \begin{enumerate}[label=\roman*.]
		\item antes do próximo arremesso (legal ou ilegal), exceto em apelação sobre um Substituto Ilegal, Jogador Não Anunciado, Reingresso Ilegal, Jogador de Emergência ou Jogador Removido e jogadores que mudam de posições nas bases;
		\item  antes que todos os jogadores da equipe na defensiva tenham deixado o território \gls{fair}, a caminho do \gls{bench} ou da área do \gls{dugout} (se um defensor faz a apelação, ele tem de estar no campo interno quando se manifesta ao árbitro); ou
		\item no caso do último lance do jogo, antes que os árbitros tenham deixado o campo de jogo.
		\item a qualquer momento que um corredor incorreto estiver em uma base em uma  entrada de desempate (penalidade) ou como corredor temporário para o receptor ou arremessador.
	\end{enumerate}
\item Os corredores podem deixar suas bases durante uma apelação com bola viva  quando:
	\begin{enumerate}[label=\roman*.]
		\item a bola deixa o Círculo do Arremessador;
		\item a bola sai da mão do arremessador; ou
		\item o arremessador faz um movimento de lançamento indicando uma jogada, ou simula um lançamento.
	\end{enumerate}
\item  APELAÇÃO COM BOLA MORTA.

	  Uma vez que a bola tenha sido devolvida  ao campo interno e o árbitro tenha declarado \gls{time}, ou a bola tenha se tornado morta, qualquer membro da equipe na defensiva que esteja no campo interno, com ou sem a posse da bola, pode fazer uma apelação verbal sobre um corredor que tenha omitido uma base ou deixado uma base antecipadamente numa bola  \gls{fly} pega no ar.

	  Um \gls{coach} ou técnico pode fazer uma apelação com bola morta  somente após entrar no campo de jogo. O árbitro que recebe a apelação deve apreciá-la e dar a decisão sobre a jogada. Nenhum corredor pode deixar sua base durante esse período, visto que a bola permanece morta até o próximo arremesso.

	\vspace{3mm}
	 EXCEÇÃO: Um corredor que tenha deixado uma base antecipadamente numa  bola \gls{fly} pega no ar, ou tenha omitido uma base, pode tentar retornar para tal  base enquanto a bola está morta.

	 \begin{enumerate}[label=\roman*.]
		\item Se a bola fica fora de jogo, a apelação com bola morta não pode ser feita até que o árbitro de \gls{home} coloque uma nova bola no jogo.
		\item  Se o arremessador, de posse da bola, está em contato com o \gls{pitcher's plate} quando faz uma apelação verbal, não deve ser declarado um Arremesso Ilegal.
		\item  Se o arremessador faz uma apelação depois da ordem \gls{play ball}, o árbitro deve declarar \gls{time} outra vez e permitir o processo de apelação.
	 \end{enumerate}
	\item Apelações por uma eliminação adicional depois do terceiro \gls{out} são permitidas, desde que elas sejam feitas corretamente e com o objetivo de invalidar um ponto ou restabelecer a ordem de batedores correta.

	\item  Estes são os tipos de apelação:
	 \begin{enumerate}[label=\roman*.]
		\item omissão de uma base;
		\item  o corredor deixa a sua base num \gls{fly} pego no ar, antes da bola ter sido tocada por um defensor;
		\item  batedor fora de ordem;
		\item  tentativa de avançar à segunda base após alcançar a primeira base;
		\item  substituições ilegais;
		\item  o uso de um jogador não anunciado sob a Regra de Jogador de Emergência;
		\item  reingresso ilegal;
		\item  o uso de um jogador não anunciado sob a Regra de Jogador Designado; ou
		\item  corredores mudam de posições nas bases que eles ocupavam.
		\item  o uso de um corredor incorreto na segunda base em uma entrada de desempate (penalidade) ou como um corredor temporário para o receptor ou arremessador.
	\end{enumerate}
	\end{enumerate}

\subsection{VENCEDORA DO JOGO}

A vencedora do jogo é a equipe que anota mais pontos do que a outra num Jogo Regulamentar.

\begin{enumerate}[label=(\alph*)]
	\item  O placar de um Jogo Regulamentar encerrado antes do sétimo \gls{inning} é aquele registrado no fim do último \gls{inning} completo, a menos que a equipe local anote mais pontos do que a equipe visitante na segunda metade do \gls{inning} incompleto. Nesse caso, o placar é aquele do \gls{inning} incompleto.
	\item  placar de um jogo regulamentar empatado é aquele quando o jogo foi encerrado.
	\item  O placar de um jogo confiscado (\gls{forfeited game}) é 7-0 a favor da equipe não infratora.
\end{enumerate}

\subsection{MOTIVOS PARA UM PROTESTO}

\begin{enumerate}[label=(\alph*)]
	\item  Um protesto que será recebido e considerado inclui assuntos como:
	 \begin{enumerate}[label=\roman*.]
	 	\item  interpretação errônea de uma Regra;
		\item  não aplicação da regra correta a uma determinada situação por um árbitro; ou
		\item  não aplicação da penalidade correta a uma determinada infração.
	 \end{enumerate}
	\item  Depois de efetuado um arremesso (legal ou ilegal), nenhuma decisão do árbitro pode ser mudada.
	\item  A qualquer momento um protesto pode ser submetido à autoridade  competente (que não seja o árbitro de \gls{home}) para verificar a situação de um membro da relação da equipe (se está ou não elegível para participar do jogo).
\end{enumerate}

\subsection{PROTESTOS}
 Um protesto pode envolver tanto um assunto de apreciação como a interpretação de uma regra.

% Exemplo de uma situação desse tipo:
 \begin{exemplo}

Com um \gls{out} e corredor na segunda e na terceira bases, o batedor acertou um \gls{fair fly} (\gls{fly} para o território \gls{fair}) e foi declarado \gls{out}. O corredor da terceira  base saiu legalmente da base (fez \gls{tag up} depois que a bola foi pega), mas o da segunda base deixou a base antecipadamente. O corredor da terceira base havia cruzado o \gls{homeplate} antes da bola ser jogada à segunda base para  completar o terceiro \gls{out}.

O árbitro não validou o ponto.

\vspace{3mm}
As dúvidas sobre: se os corredores deixaram suas bases antes da bola ser pega no ar e se a jogada na  segunda base foi feita antes do corredor da terceira base pisar no \gls{homeplate} são assuntos exclusivamente de apreciação, e não podem ser protestados.

\vspace{2mm}
A  falha cometida pelo árbitro ao invalidar o ponto foi uma interpretação errônea de uma regra de jogo é uma decisão suscetível de protesto.

 \end{exemplo}


\subsection{PROTESTOS INVÁLIDOS}
 Nenhum protesto será recebido ou considerado se ele é baseado somente numa  decisão que implique na apreciação por parte de um árbitro, ou se a equipe que está protestando tiver vencido o jogo.

 Exemplos de protestos que não serão considerados:
\begin{exemplo}
\begin{enumerate}[label=(\alph*), itemsep=.75mm]
	\item se uma bola batida foi \gls{fair} ou \gls{foul};
	\item  se um corredor foi \gls{safe} ou \gls{out};
	\item  se uma bola arremessada foi \gls{strike} ou \gls{ball};
	\item  se um arremesso foi legal ou ilegal;
	\item  se um corredor tocou ou não uma base;
	\item  se um corredor deixou a base antecipadamente numa bola \gls{fly} pega no ar;
	\item  se uma bola \gls{fly} foi ou não pega no ar;
	\item  se uma bola \gls{fly} foi ou não um \gls{infieldfly};
	\item  se houve ou não uma Interferência;
	\item  se houve ou não uma Obstrução;
	\item  se um jogador, ou uma bola viva, entrou ou não numa área de bola morta, ou tocou ou não um objeto ou pessoa em área de bola morta;
	\item  se uma bola batida transpôs ou não, em voo, uma cerca;
	\item  se o campo está adequado para continuar ou recomeçar a partida;
	\item  se há iluminação suficiente para continuar a partida; ou
	\item  qualquer outro assunto que implique somente uma correta apreciação do árbitro.
\end{enumerate}
\end{exemplo}

\subsection{COMUNICAÇÃO DA INTENÇÃO DE PROTESTAR}
\begin{enumerate}[label=(\alph*)]
	\item  Exceto para a desqualificação de um jogador, o protesto tem de ser feito claramente ao árbitro, imediatamente antes do próximo arremesso, legal ou ilegal. O protesto no fim de um \gls{inning} tem de ser feito antes que todos os defensores deixem o território \gls{fair}, a caminho do \gls{bench} ou da área do \gls{dugout}, e no último lance do jogo, antes que os árbitros deixem o campo de jogo.
	\item  A partir do momento em que é apresentado o protesto, de acordo com esta Regra, o resto do jogo terá prosseguimento sob protesto.
	\item  O técnico ou o técnico interino da equipe reclamante deve comunicar o protesto ao árbitro de \gls{home}, e este tem de informar o técnico da equipe oponente e o anotador oficial.
	\item  Todas as partes interessadas têm de ser informadas sobre as condições que influíram na tomada da decisão, as quais ajudarão na correta solução do problema.
\end{enumerate}

\subsection{ÚLTIMO PRAZO PARA FORMALIZAR UM PROTESTO OFICIAL}
 Um protesto por escrito oficial tem de ser formalizado dentro de um tempo razoável.

	\begin{enumerate}[label=(\alph*)]
		\item Na falta de uma regra da Liga ou da Competição fixando um limite de tempo para formali\-za\-ção de um protesto, uma reclamação deve ser considerada se formalizada dentro de um tempo razoável, dependendo da natureza do caso e da dificuldade para se obter a informação em que é baseada o protesto.
		\item Geralmente, 48 horas após o tempo estabelecido para contestação é considerado um tempo razoável.
	\end{enumerate}

\subsection{PROTESTO POR ESCRITO FORMAL -- REQUISITOS}

 Um protesto por escrito formal tem de conter as seguintes informações para ser válido:
\begin{enumerate}[label=(\alph*)]
	\item a data, o horário e o local do jogo;
 	\item o(s) nome(s) do(s) árbitro(s) e anotador(es);
 	\item as Regras Oficiais ou as regras locais sob as quais é feito o protesto;
 	\item a decisão e as condições que influíram na tomada da decisão; e
 	\item todos os fatos essenciais envolvidos no assunto protestado.
\end{enumerate}

\subsection{RESULTADO DE PROTESTO}

A decisão tomada sobre um jogo protestado tem de chegar a um dos resultados a seguir.

	\begin{enumerate}[label=(\alph*)]
		\item O protesto é indeferido e o placar do jogo é mantido.
	 	\item Quando um protesto em razão de interpretação errônea de uma Regra é  reconhecido, o jogo deve recomeçar do ponto em que foi dada a decisão incorreta, com essa decisão corrigida.
	 	\item Quando um protesto sobre a situação irregular de um membro da lista da  equipe é reconhecido, o jogo deve ser confiscado (\gls{forfeited game}) a favor da  equipe prejudicada.
	\end{enumerate}

\chapter{CAMPO DE JOGO E EQUIPAMENTO}

\minitoc% Creating an actual minitoc

\section{DEFINIÇÕES}
\subsection{\textit{BAT} ADULTERADO}

Um \gls{bat} está Adulterado quando a estrutura física de um \gls{bat} legal foi modificada.

Exemplos de alteração de um \gls{bat}:
\begin{exemplo}
\begin{itemize}
	\item substituir o cabo de um \gls{bat} de metal por um cabo de madeira ou outro tipo de cabo; inserir material dentro do \gls{bat}; aplicar fita adesiva em excesso (mais de duas camadas) no cabo do \gls{bat};
	\item pintar a superfície superior ou inferior do \gls{bat} com outro propósito que não o de identificação;
	\item gravar uma marca de identificação "ID" na parte mais grossa de um \gls{bat} de metal; ou
	\item colocar uma empunhadura com o formato de um sino ou cone no \gls{bat}.
\end{itemize}
\end{exemplo}

 A substituição de uma empunhadura por outra legal não é considerada uma adulteração de \gls{bat}. Uma marca de identificação "ID" gravada  somente na saliência arredondada de segurança de um \gls{bat} de metal ou uma  marca a laser para os propósitos de identificação em qualquer parte de um \gls{bat} não é uma adulteração.

\subsection{LINHA DE BASE}
 É a linha reta entre duas bases consecutivas.

\subsection{\textit{BATTER'S BOX} (ÁREA DO BATEDOR)}
 É a área dentro da qual o batedor deve permanecer enquanto está posicionado com a intenção de bater o arremesso e ajudar a equipe na ofensiva a anotar pontos. As linhas são consideradas dentro do \gls{battersbox}.

\subsection{\textit{CATCHER'S BOX} (ÁREA DO RECEPTOR)}
 É aquela área dentro da qual o receptor tem de permanecer até o arremessador completar o arremesso. As linhas são consideradas dentro do \gls{catcher's box}. o receptor é considerado estar dentro do \gls{catcher's box}, exceto quando está tocando o solo fora do \gls{catcher's box}.

\subsection{\textit{COACHER'S BOX} (ÁREA DOS TREINADORES)}
 É a área em território \gls{foul} ao lado da primeira e terceira base do campo de jogo na qual os treinadores do time de ataque devem se posicionar (consulte o Apêndice 1-F para dimensões e correta localização)."

\subsection{\textit{DUGOUT} (ABRIGO PARA MEMBROS DA EQUIPE)}
 É a área em território de bola morta, destinada somente a membros da equipe.

 É proibido fumar, consumir álcool ou usar fumo de mascar nessa área. O ato de fumar inclui a inalação de quaisquer produtos do fumo, cigarros eletrônicos e \gls{vaping} (vaporização).

\subsection{TERRITÓRIO \textit{FAIR}}

 É aquela parte dentro do campo de jogo, incluindo as linhas de \gls{foul} da primeira  e terceira base, que vai do \gls{homeplate} até a parte inferior da cerca do campo externo e perpendicularmente para cima.

\subsection{TERRITÓRIO \textit{FOUL}}
 É qualquer parte do campo de jogo que não está em território \gls{fair}.

\subsection{CAPACETE}

 Um capacete que está rachado, quebrado, amassado ou adulterado é um capacete ilegal, e deve ser removido do jogo.

\begin{enumerate}[label=(\alph*)]
	\item O capacete para batedor prevenido, batedor, batedor-corredor e corredor deve ter duas orelheiras (uma em cada lado) e tem de ser do tipo que ofereça  segurança igual ou maior do que a proporcionada por capacete inteiramente de plástico, com estofamento na parte interna. Um forro que cobre somente as orelhas não atende às especificações de um capacete legal.
 	\item O capacete para receptor ou jogador da defensiva pode ser do tipo "coquinho", sem orelheiras.
\end{enumerate}

\subsection{\textit{BAT} ILEGAL}
 É um \gls{bat} que não atende às exigências da Regra 2.3.1.

\subsection{LUVA ou \textit{MITT} ILEGAL}

 É uma luva ou \gls{mitt} que não atende às especificações de uma luva legal, ou uso de um \gls{mitt} por um defensor que não seja um receptor ou defensor da  primeira base.

\subsection{CAMPO INTERNO (\textit{INFIELD})}

 É a área do campo em território \gls{fair} normalmente coberta por defensores do campo interno.

\subsection{\textit{MITT}}
 É uma luva diferente que consiste em uma peça única para colocação dos dedos, sem ranhuras individuais para os dedos e um bojo mais profundo que uma luva  normal. Uma luva tem ranhuras separadas para os dedos na parte externa e traseira (consulte o Apêndice 4 Especificações da luva).


\subsection{EQUIPAMENTO OFICIAL}
 Equipamento oficial é qualquer material de jogo (\glspl{bat}, luvas, capacetes etc.) que está sendo usado pela equipe na defensiva ou ofensiva durante o andamento do jogo. Equipamento da defensiva (luvas, por exemplo) deixado no campo pela equipe que está atuando na ofensiva não é equipamento oficial.

\subsection{CIRCULO DE AQUECIMENTO (\textit{ON DECK CIRCLE})}
 É uma área circular restrita onde o próximo batedor (batedor prevenido) deverá se manter até que seu turno de bater seja anunciado (consulte o Apêndice 1-F para dimensões).

\subsection{CAMPO EXTERNO (\textit{OUTFIELD})}
 É aquela parte do campo de jogo em território \gls{fair}, que está além do campo interno.

\subsection{CAMPO DE JOGO}
É a área -- incluindo a linha de bola morta -- dentro da qual a bola pode ser jogada  e pega.
\section{O CAMPO DE JOGO}

\subsection{CAMPO DE JOGO -- REQUISITOS}

	\begin{enumerate}[label=(\alph*)]
		\item O campo de jogo tem que ter uma área limpa e desobstruída, dentro das dimensões mínimas estipuladas no Anexo 1 (Desenho do Campo de Jogo e \gls{diamond}), e tem que incluir todas as características mostradas.

	 	\item É aconselhável que o campo de jogo tenha uma zona de advertência. Caso seja usada uma zona de advertência, ela tem que ser uma área dentro do campo de jogo e próxima (adjacente) a qualquer cerca permanente ao longo do campo externo e das linhas laterais.

	 	\item Não há nenhuma exigência em providenciar uma zona de advertência na  superfície permanente do campo externo (grama ou outro tipo de superfície) quando se usa uma cerca provisória (isto é, quando um jogo da modalidade Arremesso Rápido é realizado num campo apropriado para jogo da modalidade Arremesso Lento).

		\item Uma bola está "fora do campo de jogo" quando ela toca o solo, uma pessoa  no campo ou um objeto, fora da área de jogo.
	\end{enumerate}

\subsection{O CAMPO OFICIAL}

	\begin{enumerate}[label=(\alph*)]
		\item O leioute do campo oficial tem que obedecer às dimensões e especificações estipuladas no Anexo 1 e tem que incluir todas as características mostradas	 [linhas de \gls{foul}, linha de um metro (3 pés), linhas laterais; \gls{coachsbox}, \gls{battersbox} e \gls{catcher's box}; círculo do batedor prevenido e círculo do arremessador;	 e bases e \gls{homeplate}].

		\item Se durante o jogo for constatada incorreção na distância entre bases ou na  distância do \gls{pitcher's plate}, o erro tem de ser corrigido no início do \gls{inning} seguinte, e o jogo deve recomeçar e continuar depois disso.
	\end{enumerate}

\subsection{REGRAS DE CAMPO OU REGRAS ESPECIAIS}

 Regras de Campo ou Regras Especiais estabelecendo os limites do campo de jogo podem ser combinadas antes do início de um jogo e usadas sempre que
 \glspl{backstop}, cercas, arquibancadas, veículos, espectadores ou outros obstáculos estejam dentro da área prescrita.

	\begin{enumerate}[label=(\alph*)]\item   Qualquer obstáculo em território \gls{fair} que se encontre a uma distância menor do que as distâncias mínimas das cercas demonstradas na Tabela de Distâncias (Anexo 1) tem de estar claramente marcado para que os árbitros possam se orientar.
	\item  Se usar um campo de beisebol, o montículo do arremessador terá de ser removido, e a distância entre o \gls{homeplate} e o \glspl{backstop}, ajustada à medida  prescrita.
	\end{enumerate}


\section{EQUIPAMENTO DE JOGO}

\subsection{\textit{BAT} OFICIAL}
 Somente um \gls{bat} oficial que está dentro dos padrões da WBSC-SD ou ISF \gls{ESC} e está estampado com o logo da WBSC-SD ou ISF -- adotado e aprovado por \gls{ESC} -- deve ser usado em uma Competição da WBSC-SD ou ISF. A Lista de \gls{bat} Aprovado da  WBSC-SD e Logo Aprovada pode ser encontrada no Website da WBSC www.wbsc.org [Vide Anexo 2A (Especificações do \gls{bat}) para padrões de \gls{bat} aprovado].

\subsection{\textit{WARM-UP BAT} (\\textit{BAT} PARA FAZER AQUECIMENTO)}
 Somente um \gls{warm-up bat} que satisfaz as especificações estabelecidas no Anexo 2B (Especificações do \gls{bat}) para padrões de \gls{warm-up bat} aprovado pode ser usado.

\subsection{BOLA OFICIAL}
 Somente uma bola oficial que está dentro dos padrões da WBSC \gls{ESC} e está estampada com a marca -- adotada e aprovada -- da WBSC ou ISF \gls{ESC} deve ser usada em Competição da WBSC. Vide Anexo 3 (para padrões de bola aprovada).

\section{EQUIPAMENTO DE JOGADORES}

\subsection{LUVAS E \textit{MITT}}

\begin{enumerate}[label=(\alph*)]\item   Qualquer jogador pode usar uma luva, mas somente o receptor e o defensor  da primeira base podem usar um \gls{mitt}.
\item  Nenhum cordão da parte superior, bem como a tira ou outro dispositivo que  fica entre o polegar e o corpo da luva ou \gls{mitt} usado por um defensor da primeira   base ou receptor, ou da luva usada por qualquer defensor, podem ter mais de  12,70cm (5 polegadas) de comprimento.
\item  A luva do arremessador pode ser de qualquer cor ou combinação de cores, contanto que nenhuma cor (inclusive do trançado) seja igual à cor da bola.

 Uma  luva usada por qualquer jogador, exceto o arremessador, pode ser de qualquer cor ou combinação de cores.

\item  Luvas com círculos brancos, cinzentos ou com tom amarelado na parte externa, que deem a aparência de uma bola, não são equipamentos oficiais e não devem ser usadas. (Vide Anexo 4: Especificações da Luva, Desenho e Dimensões).
\end{enumerate}

\subsection{SAPATOS}
\begin{enumerate}[label=(\alph*)]
	\item Todos os membros da equipe têm de usar sapatos. Um sapato deve ter sua  parte superior feita de lona, couro ou materiais similares, e deve ser totalmente fechado.
	\item A sola e o salto do sapato podem ser lisas ou ter travas de borracha mole ou dura.
	\item Podem ser usadas placas de metal comum na sola e no salto, desde que suas pontas não sejam arredondadas e não se estendam mais de 1,90cm (3/4 de polegada) da sola ou do salto do sapato.
	\item Placas de plástico duro, nylon ou poliuretano, similares às placas de metal para sola e salto não são permitidas em qualquer categoria, em nenhum nível de jogo.
	\item O uso de sapatos com travas destacáveis atarraxadas sobre a sola e salto não é permitido; entretanto, sapatos com travas destacáveis atarraxadas dentro do sapato podem ser usados.
	\item Somente nas categorias menores e na modalidade Arremesso Rápido Modificado não é permitido o uso de travas de metal em nenhum nível de jogo.
\end{enumerate}


\subsection{EQUIPAMENTO DE PROTEÇÃO}

a) MÁSCARAS. Todos os receptores têm de usar uma máscara, protetor de garganta e capacete. Receptores (ou outros membros da equipe na defensiva) têm de usar máscara, protetor de garganta e capacete enquanto recebem os arremessos de aquecimento feitos do \gls{pitching plate} (placa do arremessador), ou quando estão na área de aquecimento. Se a pessoa que está recebendo os arremessos se recusar a usar a máscara, protetor de garganta e capacete, terá que ser substituída por outra pessoa que se disponha a fazê-lo. O receptor não precisa usar o protetor de garganta quando sua máscara já vem dotada de um dispositivo fixo com essa finalidade. Receptores estão autorizados a usar máscara do tipo utilizado por goleiro de hockey sobre gelo. Se a máscara não tiver um dispositivo fixo para proteger a garganta, o receptor terá de anexar um acessório que atenda a essa finalidade, antes de usá-la.

 \begin{enumerate}
	\item  PROTETORES DE ROSTO. Qualquer jogador da defensiva ou ofensiva pode usar um protetor de rosto feito de plástico que esteja aprovado1. Protetores de rosto que estejam rachados ou deformados, ou que tenham o acolchoado deteriorado, ou que não tenham acolchoado, não podem ser usados, e têm de ser retirados do jogo. Receptores não podem usar o protetor de rosto feito de plástico no lugar da máscara normal com protetor de garganta.
	\item  PROTETORES DE TÓRAX. Todos os receptores (adultos e de categorias menores) têm de usar um protetor de tórax.
	\item  CANELEIRAS. Receptores adultos e de categorias menores têm de usar caneleiras que ofereçam proteção à rótula quando estão atuando na defensiva.
	\item  PROTETOR DE PERNAS E BRAÇOS. Este equipamento pode ser usado por batedor e batedor-corredor.
 \end{enumerate}

\section{UNIFORMES}
\subsection{UNIFORMES DE JOGADOR}
Todos os jogadores de uma equipe têm de usar uniformes semelhantes em cor, estado (bom estado) e estilo. Um membro da equipe de uniforme pode, por motivos religiosos, usar cobertura específica para cabeça e vestuários que não estejam de acordo com estas Regras, sem penalidade.

\begin{enumerate}[label=(\alph*)]
	\item BONÉS  Aprovação WBSC , e no Brasil, consultar o CT -- Vigente ( CBBS )

   \begin{enumerate}[label=\roman*.]
 	\item Os bonés têm de ser semelhantes, são obrigatórios para todos os jogadores masculinos, e têm de ser usados corretamente.

\item  Bonés, viseiras e fitas para cabeça (tiaras) são opcionais para jogadoras do sexo feminino e e as jogadoras são livres para escolherem qual deles desejam usar. Se mais de um tipo for usado, todas as peças têm de ser da mesma cor, e cada peça do mesmo tipo tem de ser da mesma cor e estilo das cores do uniforme da equipe.

 O uso de viseiras de plástico ou material duro (que possa quebrar e ou machucar o atleta) não é permitido.

\item  Se um jogador da defensiva usar um capacete aprovado de cor similar à do boné do uniforme da equipe, não será exigido que ele use um boné.
\end{enumerate}
\item  CAMISETAS INTERNAS (\textit{UNDERSHIRT})
   \begin{enumerate}[label=\roman*.]
 	\item Um jogador pode usar uma camiseta interna colorida (pode ser branca). Não é necessário que todos os jogadores usem camiseta interna, porém se um deles usar, todas as camisetas internas usadas têm de ser semelhantes. Nenhum jogador pode usar camiseta que tenha as mangas expostas à vista desgastadas, desfiadas ou rasgadas.
	\item  Pode ser usada uma ou duas mangas para aquecimento (mangas de compressão ou "manguitos"), mas ela será tratada da mesma maneira que uma camiseta de mangas compridas. No caso de duas mangas,ambos os braços têm de estar cobertos, e as duas mangas têm de ser da mesma cor solida do uniforme do time e das camisetas internas de mangas usadas por qualquer jogador desse time.
\end{enumerate}
\item  CALÇAS/CALÇAS PARA \textit{SLIDINGS} \footnote{\gls{sliding}} Todas as calças de jogadores devem ser de um só estilo, ou todas longas ou todas curtas.

 Os jogadores podem usar calças para \glspl{sliding} de cor sólida e uniforme. uso de calças para \glspl{sliding} não é obrigatório para todos os jogadores, porém, se mais de um jogador estiver usando tal peça, todas elas têm de ser semelhantes em cor e estilo (excetuam-se as almofadas para \glspl{sliding} de uso temporário, com botão de pressão ou velcro). Nenhum jogador pode usar calças para \glspl{sliding} que tenham as partes expostas à vista (partes que cobrem as  pernas) desgastadas, desfiadas ou rasgadas.

\item  NÚMEROS. Nas costas de todas as camisas do uniforme tem de ser usado um número arábico de cor contrastante, com pelo menos 15,20cm (6 polegadas) de altura. Nenhum técnico, \gls{coach} ou jogador da mesma equipe pode usar números idênticos (1 e 01 são exemplos de números idênticos). Somente  números redondos (01 a 99) devem ser usados. Jogadores sem número não terão permissão para jogar.

\item  NOMES. Acima do número nas costas da camisa do uniforme pode ser inscrito o nome individual do jogador.

\item  PEÇAS MOLDADAS. Peças moldadas (gesso, metal ou outras substâncias duras em sua forma final) não podem ser usadas num jogo. Qualquer metal exposto (exceto uma peça moldada) tem de ser adequadamente coberto com um material macio, preso com fita adesiva e aprovado pelo árbitro.

\item   ADORNOS QUE PODEM CAUSAR DISTRAÇÃO. Nenhuma peça exposta, incluindo joia que, na opinião do árbitro, pode causar distração a jogadores da  equipe adversária pode ser usada ou ostentada.

  O árbitro tem de mandar retirá-la ou cobri-la.

  Braceletes e/ou colares com fins medicinais que, na opinião do árbitro, possam causar distração têm de estar presos ao corpo com fita adesiva, de tal maneira que a informação sobre a finalidade dessas peças fique visível.

   \end{enumerate}


\section{UNIFORMES DE \textit{COACHES}}
 Um \gls{coach} tem de estar devidamente trajado -- inclusive usar calçados apropriados -- ou vestido com o uniforme da equipe, que deve seguir o padrão de cor(es) do clube. Se um \gls{coach} decide usar um boné, este tem de estar de acordo com a Regra 2.5.1 (a).

\section{EQUIPAMENTO}

Não obstante qualquer disposição destas Regras, a WBSC-SD ou ISF \gls{ESC} se reserva o direito de negar ou revogar a  aprovação de qualquer equipamento que, na sua opinião única, mude significativamente a característica do jogo, afete a segurança de participantes ou espectadores ou torne a performance dos jogadores um produto de seu equipamento e não de sua habilidade individual.

\section*{EFEITOS}
\addcontentsline{toc}{section}{EFEITOS}

\begin{description}
	\item[Regra 2.4.2 -- Uso de sapatos impróprios]  Efeito:membro da equipe que continuar cometendo a infração, após uma advertência do árbitro, deverá ser expulso do jogo.

	\item [Regra 2.4.3 (a) -- Receptor deixa de usar um capacete] Efeito: Se o jogador continuar cometendo a infração após uma advertência do árbitro, deverá ser expulso do jogo.

	\item[Regra 2.4.3 (b-d) -- Jogador não usa equipamento obrigatório] Efeito: O jogador é removido do jogo. Se continuar participando, será expulso do jogo.

	\item[Regra 2.5.1 -- Uniforme impróprio ou uso incorreto do uniforme por um jogador] Efeito: Se o jogador se recusar a cumprir o que está determinado, deverá ser expulso do jogo.

	\item[Regra 2.6 -- \gls{coach} usa roupa imprópria] Efeito: Após uma advertência do árbitro, qualquer infração subsequente cometida por um \gls{coach} ou técnico da mesma equipe resultará na expulsão do \gls{coach} principal.
\end{description}

\chapter{PARTICIPANTES}
\minitoc% Creating an actual minitoc

\section{DEFINIÇÕES}
\subsection{\textit{coach} DE BASE}
 É um membro da equipe na ofensiva, que fica posicionado no campo e dentro do \gls{coachsbox}
enquanto sua equipe está atacando.

\subsection{\textit{coach} (TÉCNICO)}
 \gls{coach} é uma pessoa que é responsável pelas ações de sua equipe no campo e pela comunicação com o árbitro e com a equipe contrária. Um jogador pode ser um \gls{coach}, como substituto de um \gls{coach} ausente ou como um jogador- \gls{coach}.

\subsection{JOGADOR DESIGNADO (JD)}
É um jogador abridor da equipe na ofensiva, que está escalado para bater no lugar do Jogador Flex. ( Também conhecido como DP -- \gls{DP} )

\subsection{EXPULSÃO}
É o ato mediante o qual um árbitro ordena que um jogador, oficial ou qualquer membro da equipe deixe o jogo e o campo, por violação de uma regra, pelo resto do jogo.

\subsection{DEFENSOR}
 É qualquer jogador da defensiva da equipe que está no campo.

 \subsection{JOGADOR FLEX}
 É o jogador abridor que tem um Jogador Designado (JD) para bater no seu lugar, cujo nome está relacionado no 10° lugar no Formulário de Escalação da Equipe (\gls{lineup}). O Flex pode jogar em qualquer posição defensiva e pode entrar no jogo na ofensiva somente para bater no lugar do JD.

\subsection{\textit{coach} PRINCIPAL}
 técnico de uma equipe ou o \gls{coach} que assume as principais responsabilidades de um \gls{coach} é considerado o \gls{coach} principal.
\subsection{REENTRADA ILEGAL}
 Um Reentrada Ilegal ocorre quando:

\begin{enumerate}[label=(\alph*)]\item   um jogador, incluindo o JD e FLEX, retorna ao jogo numa posição na ordem de batedores para a qual não está legalmente habilitado, isto é, numa posição que não é a sua posição inicial original; ou
\item  um jogador retorna ao jogo na defensiva ou ofensiva e não está legalmente habilitado para entrar naquela posição.
\item  Um jogador inicial reentra no jogo pela segunda vez, não como um jogador substituto.
\item  Um jogador inelegível entra no jogo, ou
\item  O jogador FLEX entra no ataque no lugar de um jogador que não o JD.
\end{enumerate}

\subsection{SUBSTITUIÇÃO NÃO ANUNCIADA}
 Uma substituição não informada ocorre quando um jogador entra no jogo sem ser reportado ao Árbitro responsável como:
\begin{enumerate}[label=(\alph*)]\item   um substituto;
\item  um jogador elegível para entrar, retornar ou permanecer no jogo de
 acordo com as disposições da regra do jogador substituto;
\item  um jogador declarado inelegível; ou
\item  uma reentrada ilegal.
\end{enumerate}

\subsection{JOGADOR INELEGÍVEL}
 É aquele que não pode mais participar do jogo como um jogador, por ter sido removido pelo árbitro. Um jogador Inelegível pode continuar no jogo como um \gls{coach}.

\subsection{TROCA POR JOGADOR INELEGÍVEL}
 Um Jogador Inelegível é aquele que não pode entrar no jogo para substituir um Jogador Removido (jogador que tem de deixar o jogo para cuidar de um ferimento que tenha causado hemorragia). É um jogador que:

\begin{enumerate}[label=(\alph*)]\item   foi removido do jogo pelo árbitro por ter infringido uma Regra;
\item  está atuando no jogo naquele momento;
\item  não está atuando no jogo naquele momento, mas está elegível para reingressar no jogo.
\end{enumerate}

\subsection{DEFENSOR DO CAMPO INTERNO}

É um jogador da defensiva, incluindo o arremessador e o receptor, que geralmente está posicionado em qualquer lugar perto ou dentro das linhas do caminho da base que formam o território \gls{fair}. Um jogador que normalmente joga no campo externo pode ser considerado um defensor do campo interno se ele se move para uma área normalmente coberta por defensores do campo interno.

\subsection{FORMULÁRIO DE ESCALAÇÃO DA EQUIPE (\textit{LINE UP CARD})}
 É a lista de jogadores abridores, substitutos e \glspl{coach} entregue ao Chefe dos Árbitros e/ou Árbitro de \gls{home} e ao Anotador Oficial antes do início do jogo. o Árbitro de \gls{home} retém o Formulário de Escalação da Equipe durante o jogo.

\subsection{ESCALAÇÃO DA EQUIPE}
 Jogadores em campo, incluindo o JD e o Jogador Flex, que estão jogando na ofensiva e na defensiva.

\subsection{JOGADOR SOMENTE DA OFENSIVA ("OPO")}
 É um jogador que está no \gls{batting order} (ordem de batedores), exceto o JOGADOR FLEX, por quem o JD está jogando na defesa.
\subsection{REINGRESSO}

Ocorre quando um jogador abridor retorna ao jogo após ser substituído.
\subsection{REMOÇÃO DO JOGO}

Quando um árbitro declara que um jogador não está elegível para nova  participação no jogo em razão de transgressão de Regra. Qualquer pessoa  assim removida pode continuar no \gls{bench}, mas não pode mais participar do jogo, exceto como um \gls{coach}.
\subsection{JOGADOR SUBSTITUTO}

É aquele autorizado a entrar no jogo para substituir um jogador que tem de sair (do jogo) para tratar de um ferimento com hemorragia.

\subsection{ROOSTER (JOGADORES OFICIALIZADOS)}
 Lista de todos os jogadores elegíveis que podem entrar em um \gls{lineup} de jogo.

\subsection{JOGADORES ABRIDORES}
 São aqueles relacionados no Formulário de Escalação da Equipe (\gls{lineup}) que iniciam o jogo na defensiva ou na ofensiva.

\subsection{SUBSTITUTO}

\begin{enumerate}[label=(\alph*)]
	\item É um jogador (não um Jogador Abridor) que não tenha atuado no jogo, a não ser como Jogador de Emergência.
	\item É um jogador abridor que deixara o jogo uma vez e que pode retornar a esse  jogo.
\end{enumerate}

\subsection{MEMBRO DA EQUIPE}
 É uma pessoa autorizada a sentar no \gls{bench} da equipe.
\subsection{CORREDOR TEMPORÁRIO}

É um jogador que pode correr no lugar do receptor ou lançador quando este, com dois \glspl{out}, está ocupando uma base.
\subsection{JOGADOR REMOVIDO}

É um jogador que tem de deixar o jogo devido a um ferimento com hemorragia que não pode ser estancada num tempo razoável, ou quando o uniforme do jogador fica coberto de sangue.

\section{ESCALAÇÃO DA EQUIPE E LISTAS DE JOGADORES (\textit{LINE UP \& ROSTERS})}
\subsection{FORMULÁRIOS DE ESCALAÇÃO DA EQUIPE (\textit{LINE UP CARDS})}
\begin{enumerate}[label=(\alph*)]\item   O Formulário de Escalação contém:
  \begin{enumerate}[label=\roman*.]
 	\item o último sobrenome, o primeiro nome, a posição e o número de uniforme dos jogadores abridores;
	\item  o último sobrenome, o primeiro nome e o número de uniforme dos substitutos disponíveis; e
	\item  o último sobrenome e o primeiro nome do \gls{coach} principal.
	\item  O nome de um jogador abridor não pode estar no Formulário de Escalação da Equipe se ele não estiver presente, uniformizado, e na área de sua equipe.
 \end{enumerate}
\item  Um jogador Elegível que consta na relação de jogadores pode ser acrescentado à lista de reservas a qualquer momento durante o jogo.
\item  A lista masculina deve conter somente os nomes de jogadores, e a lista  feminina somente os nomes de jogadoras.
\item  Se um número de uniforme está inscrito incorretamente no Formulário de Escalação da Equipe, a mudança pode ser feita sem penalidade. Se um jogador que está usando uniforme com número incorreto infringe qualquer Regra, a  violação de Regra tem precedência e deve ser aplicada a ele ( o jogador ). Se o jogador permanece no jogo após cometer uma infração, o número do uniforme tem de ser corrigido antes de dar prosseguimento à partida.

\end{enumerate}

\subsection{JOGADORES}
\begin{enumerate}[label=(\alph*)]\item   Cada equipe tem que ter no mínimo nove (9) jogadores escalados em todos
 os momentos. Se estiver usando Jogador Designado (JD), uma equipe tem de
 ter dez (10) jogadores relacionados na escalação da equipe. O nome do JD tem
 de estar inscrito no Formulário de Escalação da Equipe inicial.
  \begin{enumerate}[label=\roman*.]
	\item As posições da equipe na defensiva são:
		\begin{tabular}{l|c}\hline
			arremessador & D1	\\\hline
			receptor &D2\\\hline
			defensor da primeira base &  D3\\\hline
			defensor da segunda base & D4\\\hline
			defensor da  terceira base &  D5\\\hline
			interbases & D6\\\hline
			defensor externo esquerdo & D7\\\hline
			defensor externo central & D8\\\hline
			defensor externo direito & D9\\\hline
	\end{tabular}
\item  As posições da equipe na defensiva com dez (10) jogadores são as mesmas
 de uma equipe com nove (9) jogadores, mais o JD.
\end{enumerate}
\item  Jogadores da equipe que estejam dentro do campo podem posicionar-se em
 qualquer lugar do território \gls{fair} no início de cada arremesso, exceto o receptor,
 que tem de estar no \gls{catcher's box}, e o arremessador, que tem de estar numa
 posição legal de arremesso ou dentro do Círculo do Arremessador, quando o árbitro põe a bola em jogo.
\item  Uma equipe tem de ter a quantidade exigida de jogadores Elegíveis na
 escalação, em todos os momentos, para continuar um jogo.
\end{enumerate}
\subsection{JOGADORES ABRIDORES}
\begin{enumerate}[label=(\alph*)]\item   Um jogador abridor é oficializado uma vez que o Formulário de Escalação da
 Equipe é confirmado pelo representante da equipe e pelo árbitro de \gls{home} na
 reunião pré-jogo no \gls{homeplate}.
\item  Nomes, números de uniformes e posições podem estar relacionados no Formulário de Escalação da Equipe antes da reunião pré-jogo.
\item  Em caso de ferimento ou doença, o representante da equipe pode fazer  mudanças no Formulário de Escalação da Equipe na reunião no \gls{homeplate}  antes das escalações serem declaradas oficiais. Um substituto listado pode  ocupar o lugar de um jogador abridor doente ou machucado cujo nome esteja na escalação inicial de sua equipe. Ele deve ser considerado o jogador abridor, e o outro jogador pode ser um substituto.
\item  O jogador abridor assim substituído na reunião no \gls{homeplate} pode entrar no jogo, mais tarde, como um substituto, a qualquer momento.
\item  Todos os jogadores iniciais, incluindo o JD e o FLEX, podem ser substituídos  ou reentrar na escalação uma vez e devem permanecer na mesma posição de  batedor sempre que estiverem na escalação. Uma violação desta regra é  considerada uma reentrada ilegal.
\end{enumerate}
\subsection{JOGADOR DESIGNADO (JD)}
\begin{enumerate}[label=(\alph*)]
	\item Um JD pode bater no lugar de qualquer jogador da defensiva designado como FLEX.
 	\item O JD pode jogar na defensiva no lugar de qualquer jogador, incluindo o Jogador Flex.
	  \begin{enumerate}[label=\roman*.]
		\item  Se o JD joga na defensiva no lugar de um jogador que não seja o Jogador Flex, esse jogador continua batendo e é identificado como o "OPO" (Jogador Somente da Ofensiva). Não se considera que o "OPO" tenha deixado o jogo e ele continua a bater, mas não joga na defensiva.
		\item Quando o JD joga na defensiva no lugar do Jogador Flex, isso é tratado como uma substituição e tem de ser comunicado ao árbitro.
		\item Quando o JD joga na defensiva no lugar do Jogador Flex, a quantidade de jogadores fica reduzida a nove (9), e o jogo pode terminar legalmente com nove (9) jogadores.
		\item A escalação (\gls{line-up}) de uma equipe pode reverter para 10 jogadores:
	 	\begin{enumerate}[label=\arabic*)]
			\item  inserindo um substituto na posição FLEX; ou
			\item reinserindo o FLEX inicial original, mas isto apenas por uma vez.
		\end{enumerate}
	\end{enumerate}
 	\item O JD e o Jogador Flex não podem estar no jogo atuando na ofensiva ao mesmo tempo.
\end{enumerate}

\subsection{JOGADOR FLEX (FLEX)}

\begin{enumerate}[label=(\alph*)]
	\item Se uma equipe anuncia o uso de um JD, ela tem de mencionar o nome de um Jogador Flex no Formulário de Escalação da Equipe.
	\item O Jogador Flex é colocado na décima (10a) posição no \gls{line-up} (escalação da equipe) inicial, em seguida aos nomes dos nove (9) batedores, e pode jogar em qualquer posição como defensor.
	\item O Jogador Flex pode entrar no jogo para atuar na ofensiva somente no lugar do JD.
	\begin{enumerate}[label=\roman*.]
		\item Quando o Jogador Flex atua como batedor, a quantidade de jogadores fica  reduzida a nove (9). A equipe pode terminar o jogo com nove (9) jogadores.
		\item O Jogador Flex pode atuar como batedor no lugar do JD quantas vezes for necessário. Cada vez que o Jogador Flex atua na ofensiva no lugar do JD, o árbitro de \gls{home} deve ser anunciado, já que essa alteração é tratada como uma  substituição.
		\item A escalação (\gls{line-up}) de uma equipe pode reverter para 10 jogadores:

		\begin{enumerate}[label=\arabic*)]
			\item inserindo um substituto na posição JD; ou
 			\item reinserindo o JD inicial original, mas isto apenas por uma vez.
 	\end{enumerate}

	\item  É considerada reentrada ilegal quando um jogador FLEX entra no jogo no ataque no lugar de um jogador que não seja o JD.
\end{enumerate}
\end{enumerate}
\subsection{JOGADOR DE EMERGÊNCIA}
\begin{enumerate}[label=(\alph*)]
	\item Um jogador reserva indicado no cartão de escalação pode substituir qualquer jogador na escalação. Múltiplas substituições podem ser feitas para um jogador que está listado na escalação inicial, mas nenhum reserva pode retornar ao jogo após ser substituído (removido do jogo), exceto como jogador reserva. Um reserva que reentrar no jogo como jogador é uma reentrada ilegal.
	\item O Jogador Removido não deve retornar ao jogo até que a hemorragia tenha  cessado, o ferimento esteja limpo e coberto e, se necessário, o uniforme seja substituído, mesmo que a camisa do uniforme tenha um número diferente. Não haverá penalidade se usar uma camisa com número diferente; entretanto, o árbitro tem de ser informado sobre a mudança de número.
	\item Um Jogador de Emergência pode jogar no lugar do Jogador Removido pelo resto do \gls{inning} em andamento e pelo \gls{inning} seguinte completo. O Jogador Removido pode retornar ao jogo a qualquer momento durante esse período, sem ser tratado como uma substituição. Um Jogador de Emergência não é considerado um substituto. Se o Jogador Removido não retorna, depois de terminado o restante de \gls{inning} e depois de completado o \gls{inning} seguinte completo, tem de ser feita uma substituição legal.
 	\item Um representante da equipe deve notificar o Árbitro Principal sobre todas as alterações do cartão de escalação. Uma apelação adequada e mantida sobre uma falha nesta comunicação resultará na declaração desse jogador como uma "substituição" não relatada. E as penalidades serão aplicadas.
 	\item Um Jogador de Emergência pode ser:

	\begin{enumerate}[label=\roman*.]
		\item um substituto legal que não tenha ainda atuado no jogo;
		\item um substituto legal que tenha atuado no jogo, mas que fora substituído depois; ou
		\item um jogador abridor que não esteja mais atuando e não tenha mais condição de retornar ao jogo.
	\end{enumerate}
	\item  Um jogador inelegível não pode retornar ao jogo como um jogador.
\end{enumerate}

\subsection{CORREDOR TEMPORÁRIO}

Um corredor temporário é admitido e considerado legal para substituir um receptor ou arremessador desde que esteja declarado no \gls{line-up} no topo do primeiro \gls{inning} ou receptor ou arremessador do meio \gls{inning} prévio que está em base, no ataque, com dois \glspl{out} declarados.

	\begin{enumerate}[label=(\alph*)]\item   O uso do corredor temporário é opcional para a equipe de ataque.
		\item  O corredor temporário pode ser usado em qualquer momento após a ocorrência da segunda saída.
		\item  O corredor temporário é o jogador programado para bater por último que não está na base no momento em que a opção é tomada.

		 Se um corredor incorreto for usado como um corredor temporário, o corredor será eliminado desde que uma apelação apropriada seja feita pela equipe de defesa após um arremesso ou uma jogada tenha sido feita e esta apelação  seja mantida. Uma apelação apropriada pode ser feita a qualquer momento que corredor incorreto estiver na base. Se a equipe de ataque colocar o corredor correto na base antes que uma apelação apropriada seja feita, não há penalidade.
	 \end{enumerate}

\subsection{SUBSTITUIÇÕES}

\begin{enumerate}[label=(\alph*)]
	\item Um substituto pode ocupar o lugar de qualquer jogador cujo nome está no \gls{line-up} (escalação da equipe). Podem ser feitas substituições múltiplas para o jogador relacionado na escalação inicial, mas nenhum substituto pode retornar ao jogo depois de ter sido retirado do \gls{line-up}, exceto como um Jogador de Emergência ou \gls{coach}. É considerada uma reentrada ilegal se um reserva  reentrar no jogo como jogador.
 	\item Um jogador abridor e seu(s) substituto(s) não podem estar no jogo ao mesmo tempo.
	\item Uma substituição tem de ser feita somente quando a bola está morta. o \gls{coach} ou o representante da equipe tem de comunicar imediatamente ao árbitro de \gls{home} antes de fazer a substituição. O substituto não é considerado legalmente no jogo até que um arremesso tenha sido efetuado ou uma jogada  tenha sido executada. O árbitro de \gls{home} anunciará a mudança ao anotador.
	\item Qualquer substituto que esteja legalmente no jogo, mas não tenha sido anunciado ao árbitro torna-se uma Substituição não comunicada (Ilegal).
	\item Não há qualquer infração se o técnico, o \gls{coach}, o representante da equipe ou o jogador que comete falta avisa o árbitro antes da apelação da equipe prejudicada.
	\item Um substituto que retorna ao jogo depois de ser substituído é um jogador ilegal (Reingresso Ilegal), a menos que ele esteja sendo usado como um Jogador de Emergência ou \gls{coach}.
	\item Um jogador inelegível não pode retornar ao jogo como um jogador.
\end{enumerate}

\section{APELAÇÕES}
\begin{enumerate}[label=(\alph*)]
	\item Apelações têm de ser feitas por um técnico, \gls{coach} ou jogador antes que um árbitro dê uma decisão sobre:
	 \begin{enumerate}[label=\roman*.]
	 	\item substituições ilegais não ;
	 	\item o uso de um jogador não anunciado quando está sendo aplicada a Regra de Jogador de Emergência;
		\item reingresso Ilegal; ou
		\item o uso de um jogador não anunciado quando está sendo aplicada a Regra de Jogador Designado.
	\end{enumerate}
\item  Uma apelação pelos motivos acima pode ser feita a qualquer momento enquanto o jogador está no jogo.
\end{enumerate}

 EFEITOS
 Regras: 3.2.2 (a), 3.2.3 (c) e 3.2.6 (c) -- Quando não completa um jogo com a quantidade necessária de jogadores
 Efeito: O jogo é confiscado a favor da equipe não infratora.

 Regras: 3.1.10 a-b, 3.2.4 b ii, , 3.2.5.c.ii, 3.2.6 d, 3.2.8 a-e -- Substituto Não Anunciado/Jogador Ilegal
	\begin{enumerate}[label=(\alph*)]\item   Substituto Ilegal;
	\item  Jogador de Emergência não anunciado; ou
	\item  Retorno de Jogador Removido não comunicado.
	\end{enumerate}
 Efeito:
\begin{enumerate}[label=(\alph*)]
	\item Substituto Não Anunciado ou Jogador Ilegal: é uma Jogada de Apelação.
 	\item A apelação tem de ser levada à atenção do árbitro enquanto o jogador ilegal ou o substituto não anunciado está no jogo.
	\item Uma vez que tenha sido efetuado um arremesso, ou tenha sido executada uma jogada, e o substituto não anunciado tenha sido descoberto, esse substituto é declarado um Jogador Inelegível.
	\item Um substituto legal tem de ocupar o lugar do Jogador Inelegível.

 Se a equipe infratora não tem um substituto legal, o jogo é confiscado em favor do time que não ofendeu a regra.

	\item Se o jogador ilegal sofre apelação enquanto está no \gls{battersbox}, um substituto legal deve assumir a contagem de arremessos (\gls{ball} e \gls{strike}).

	\begin{enumerate}[label=\roman*.]
		\item Toda ação anterior à descoberta é legal, exceto se o substituto não anunciado bater o arremesso e alcançar uma base, e depois, após ser descoberto, sofrer apelação antes de um arremesso ao Batedor Prevenido, ou no fim do jogo e antes do árbitro deixar o campo; todos os corredores (incluindo o batedor) devem retornar à base que estavam ocupando no momento do arremesso; o substituto não anunciado é "Declarado Inelegível" e é declarado \gls{out}.
		\item Todas as eliminações, com exceção dos casos mencionados no item (d) acima permanecerão.
	\end{enumerate}

	\item Se o substituto é um Jogador Ilegal, ele deve estar também sujeito à  penalidade por essa infração.
	\item Se o Jogador Ilegal é descoberto atuando na defensiva e, após executar uma jogada, é feita uma apelação correta, esse jogador é declarado Inelegível, e a equipe na ofensiva pode:

	\begin{enumerate}[label=\arabic*)]
		\item  aceitar o resultado da jogada; ou
		\item optar pelo retorno do batedor, assumindo a contagem de arremessos (\gls{ball} e \gls{strike})  que tinha antes do Jogador Ilegal ser descoberto. Cada corredor deve retornar à base que estava ocupando antes da jogada.
	\end{enumerate}
	\item Se um Jogador Inelegível retorna ao jogo, é declarado um confisco de jogo a favor da equipe não infratora.

	\begin{enumerate}[label=\roman*.]
		\item  Após ser comunicada uma apelação sobre substituição não anunciada ou reingresso não anunciado, considera-se que o jogador original ou seu substituto tenha deixado o jogo.
 	\end{enumerate}

 Regra 3.2.2 a, 3.2.3 c e 3.2.6.c
  Falha em completar o jogo com a quantidade mínima de jogadores

 EFEITO:
 Jogo é confiscado em favor do time que não ofendeu a regra

 Regras: 3.1.9, 3.1.12. 3.2.5 d e 3.2.8 -- Reingresso Ilegal
 Efeito:
 Esta é uma jogada de apelação a qual pode ser feita em qualquer momento enquanto o jogador ilegal estiver no jogo e não precisa necessariamente ser feita antes do próximo arremesso.

	 \begin{enumerate}[label=\roman*.]
	 	\item Substituto reingressando no jogo, não como um Jogador de Emergência.
		\item Jogador abridor reingressando no jogo pela segunda vez, não como um Jogador de Emergência.
		\item Um Jogador de Emergência Inelegível.
		\item Flex entrando no jogo para atuar na ofensiva por um jogador que não seja  Jogador Designado.
	\end{enumerate}


\begin{enumerate}[label=(\alph*)]
	\item O técnico/\gls{coach} mencionado no Formulário de Escalação da Equipe e o jogador que reingressara ilegalmente no jogo são expulsos.
	\item Um substituto legal tem de ocupar o lugar do jogador expulso por ter reingressado no jogo ilegalmente, antes do jogo ter prosseguimento.
	\item Se o técnico/\gls{coach} principal for expulso, ele tem de nomear um novo técnico/\gls{coach} principal.
	\item Em ações que ocorram enquanto o jogador que reingressara ilegalmente esteja no jogo devem ser aplicadas as penalidades impostas a Substituto Ilegal/Jogador ilegal.
	\item Se uma reentrada ilegal não for descoberta ou não for feita nenhuma  apelação e aconteça dos dois jogadores ( o ilegal e o original ) continuarem no jogo tendo como resultado um consecutivo número de jogadores na ordem de batedores que serão considerados reentradas ilegais:

	 \begin{enumerate}[label=\roman*.]
	 	\item apenas o mais recente reingresso ilegal deve sofrer apelação. Neste casos, este jogador e o técnico devem ser expulsos do jogo.
		\item os efeitos para uma substituição ilegal também tomarão efeito aqui.
		\item
		\item
	 	\item Um substituto legal deve ser colocado no jogo non lugar do substituto ilegal expulso
	 \end{enumerate}

Todos os outros jogadores que reingressaram ilegalmente que não forem objetos diretos da apelação devem retornar as suas posições normais na ordem de batedores sem que tenham sido consideradas entradas ilegais.

 Todavia, se o FLEX entrou no ataque no lugar de um outro jogador que não o JD, é uma de duas ou mais entradas ilegais e esteja em uma base no momento da apelação, mesmo que o FLEX seja o sujeito direto sobre o qual esteja sendo feita a apelação , o FLEX deve ser removido da base e colocado na decima posição no line- up. Esta movimentação não será considerada um out adicional. o flex que é removido de uma base não é trocado por utro jogados
\end{enumerate}
\end{enumerate}

 Regras : 3.1.10 c , e 3.1.11 - Jogador Inelegivel Retornando ao jogo

EFEITO:
 Quando um jogador inelegível retorna ao jogo, este será confiscado favor do time que não ofendeu a regra.
 em
\section{\textit{COACHES}}
\subsection{EM GERAL}
\begin{enumerate}[label=(\alph*)]
	\item Um \gls{coach} ou representante da equipe tem a responsabilidade de avisar o árbitro de \gls{home} quando ocorre mudança na escalação da equipe.
	\item Um \gls{coach} não pode usar linguagem que possa repercutir negativamente sobre jogadores, árbitros ou espectadores.
	\item Nenhum equipamento de comunicação deve ser usado entre:
	 \begin{enumerate}[label=\roman*.]
		\item os \glspl{coach} no campo;
		\item um \gls{coach} e o \gls{dugout};.
		\item um \gls{coach} e qualquer jogador; ou
		\item a área do espectador e o campo, incluindo o \gls{dugout}, um \gls{coach} e um jogador.
	\end{enumerate}
	\item Um \gls{coach} ou técnico da equipe na defensiva pode ser um membro da equipe (não um jogador), que permanece no \gls{dugout}, ou um \gls{coach} que esteja atuando como jogador (\gls{coach}-jogador).
	\item Um \gls{coach}-jogador num jogo pode orientar a sua equipe e lhe dar assistência durante o jogo.
\end{enumerate}

\subsection{\textit{coach} PRINCIPAL}

\begin{enumerate}[label=(\alph*)]
	\item O \gls{coach}( Técnico) principal (head coac\item  tem a responsabilidade de assinar Formulário de Escalação da Equipe.
	\item Caso o \gls{coach} principal seja expulso do jogo, ele deve submeter à apreciação do árbitro de \gls{home} o nome da pessoa que vai assumir as suas obrigações pelo resto do jogo.
\end{enumerate}

\subsection{\textit{COACHES} DE BASE (RUNNER COACH)}

\begin{enumerate}[label=(\alph*)]\item   São permitidos dois \glspl{coach} (Técnicos) de base, cuja função é dar assistência e passar instruções aos membros de sua equipe enquanto estão no ataque.
	 \begin{enumerate}[label=\roman*.]
		\item Cada \gls{base coach} tem de permanecer com ambos os pés dentro dos limites de seus respectivos \gls{coachsbox} (Caixa dos Técnicos). Um tem de estar posicionado perto da primeira base, e outro perto da terceira base.
		\item Um \gls{coach} de base pode deixar o \gls{coachsbox} para esquivar-se de um defensor ou para mandar um corredor deslizar, avançar ou retornar a uma base, desde que não interfira na jogada.
	\end{enumerate}
	\item Um \gls{coach} de base pode dirigir-se somente a membros de sua equipe.
	\item Um \gls{coach} de base pode levar consigo ao \gls{coachsbox}, um Livro de Anotações (\gls{scorebook}), caneta ou lápis e um \gls{indicator} (marcador de \gls{ball}, \gls{strike} e \gls{out}), que serão usados somente para anotar pontos ou registrar outros dados do jogo.
	\item Um jogador jovem ( ou de categorias mais baixas ) que atua como \gls{coach} nos \glspl{coachsbox} da primeira ou terceira base, e um jovem que participa do jogo como \gls{bat boy} ou \gls{bat girl} (recolhedor/recolhedora de \gls{bat}), têm que usar um capacete aprovado enquanto estão no campo ou dentro do \gls{dugout}.
\end{enumerate}

\section*{EFEITOS}
\addcontentsline{toc}{section}{EFEITOS}

\begin{description}
	\item[Regra 3.4 -- Transgressão da regra sobre responsabilidades dos \glspl{coach}]  Efeito:
	Na primeira infração, deve ser feita uma advertência. Qualquer infração subsequente cometida por um \gls{coach}/técnico da mesma equipe resultará na  expulsão do \gls{coach} Principal.
	\item[Regra 3.4.3 (d) -- Jogador jovem que atua como \gls{coach} no \gls{coachsbox}, sem usar um capacete] Efeito: Se repetir a infração após ser advertido, o jogador jovem tem que ser expulso.
\end{description}


\section{MEMBROS DA EQUIPE}
\subsection{MEMBROS EM GERAL}
\begin{enumerate}[label=(\alph*)]
	\item Nenhum membro da equipe pode contestar qualquer decisão do árbitro que implique apreciação.
	\item Durante um jogo, uma pessoa relacionada no Formulário de Escalação da Equipe ou com permissão para ficar no \gls{dugout} tem de permanecer dentro dele, a menos que as Regras permitam ou quando o árbitro considera justificável sua  permanência fora dessa área. Isso inclui jogadores, exceto o Batedor Prevenido (que tem de permanecer dentro do Círculo do Batedor Prevenido), no início do jogo, entre \glspl{inning}, ou quando um arremessador está se aquecendo. Nessa área, é proibido fumar, consumir álcool ou usar fumo de mascar.
 	\item Um membro da equipe não deve:
	\begin{enumerate}[label=\roman*.]
		\item fazer, ou permitir que outras pessoas façam ou mandem fazer comentário depreciativo ou insultuoso para/sobre jogadores adversários, dirigentes, árbitros  ou espectadores; ou
		\item cometer qualquer ato que possa ser considerado uma conduta antidesportiva.
	\end{enumerate}
\end{enumerate}

\section*{EFEITOS}
\addcontentsline{toc}{section}{EFEITOS}

\begin{description}
	\item[Regra: 3.5.1 (a) e (b) -- Contestando decisão do árbitro e conduta do \gls{dugout}] Efeito:

	\begin{itemize}
		\item Na primeira falta, a equipe será advertida.
		\item Na reincidência, o infrator será expulso.
	\end{itemize}
\item [Regra 3.51 (c) -- Conduta antidesportiva] Efeito:
	\begin{enumerate}[label=(\alph*)]
		\item Na primeira infração, o infrator pode ser advertido.

		\begin{enumerate}[label=\roman*.]
			\item Se a primeira infração for grave, o árbitro deve expulsar o infrator.
			\item Na segunda infração, o infrator é expulso.
		\end{enumerate}

		\item Um membro da equipe expulso do jogo deve ir diretamente para o vestiário pelo resto do jogo, ou deixar o campo.
		\item Se uma pessoa expulsa dessa maneira não deixar o jogo, imediatamente, justificará um confisco do jogo.
		\item Um árbitro encarregado pode denunciar um membro da equipe por conduta ofensiva, abuso verbal ou físico, a qualquer momento após um jogo ter sidoencerrado, caso em que o membro da equipe denunciado deve comparecer
		perante a organização sob a qual o jogo ou torneio esteja sendo realizado.
	\end{enumerate}
\end{description}

\section{ÁRBITROS}

\subsection{DIREITOS E OBRIGAÇÕES}

 Os árbitros são os representantes da Liga ou Organização pela qual tenham sido escalados para um jogo específico e, como tais, estão autorizados e incumbidos de fazer cumprir estas Regras.

 Eles têm autoridade para mandar um jogador, \gls{coach}, capitão ou técnico fazer ou deixar de fazer qualquer coisa que, na sua  opinião, seja necessária para garantir o cumprimento de uma ou todas estas regras, e para impor efeitos conforme aqui prescrito.

 O árbitro de \gls{home} tem  autoridade para dar decisões sobre quaisquer situações que não estejam  especificamente cobertas por estas Regras.

\subsection{\'ARBITRO DE \gls{home}}

 árbitro de \gls{home} é responsável por:
\begin{enumerate}[label=(\alph*)]
	\item decidir sobre a condição do campo de jogo para uma partida;
	\item posicionar-se atrás do \gls{homeplate} e atrás do receptor;
	\item ter o comando total do jogo e ser o responsável pela correta condução da partida;
	\item declarar \glspl{ball} e \glspl{strike};
	\item em harmonia e em colaboração com os árbitros de base, tomar decisões sobre jogadas, bolas batidas, \gls{fair} ou \gls{foul}, bolas pegas legalmente ou ilegalmente. Em jogadas nas quais o árbitro de base tem de deixar o campo interno, o árbitro de \gls{home} deve assumir as obrigações normalmente exigidas do árbitro de base;
	\item determinar e declarar se:
	 \begin{enumerate}[label=\roman*.]
	 	\item um batedor bate a bola por meio de \gls{bunt} ; ou
		\item uma bola batida toca o corpo ou a roupa do batedor;
	 \end{enumerate}
		\item tomar decisões nas bases quando solicitado a fazê-lo;
		\item determinar quando um jogo é confiscado; e
		\item assumir todas as obrigações quando é designado como um árbitro único para um jogo.
\end{enumerate}

\subsection{ÁRBITRO DE BASE}
\begin{enumerate}[label=(\alph*)]
	\item Um árbitro de base deve posicionar-se devidamente no campo de jogo, de acordo com o sistema de arbitragem adotado.
	\item Um árbitro de base deve auxiliar o árbitro de \gls{home} de todas as maneiras para aplicar estas Regras.
\end{enumerate}

\subsection{RESPONSABILIDADES DE UM ÁRBITRO ÚNICO}

Se for designado somente um árbitro para a partida, suas obrigações e a jurisdição devem estender-se para todos os casos protegidos por estas Regras.

A posição inicial do árbitro para cada arremesso deve ser atrás do \gls{homeplate} e atrás do receptor. Em cada bola batida ou jogada que se desenvolve, o árbitro deve sair de trás do \gls{homeplate} e mover-se para dentro do campo interno, a fim de obter a melhor posição para observar qualquer jogada em andamento.

\subsection{TROCA DE ÁRBITROS}
 Um árbitro não pode ser trocado durante um jogo com a anuência das equipes oponentes, a menos que ele se machuque ou adoeça e não consiga continuar atuando.

\subsection{APRECIAÇÃO DO ÁRBITRO}

\begin{enumerate}[label=(\alph*)]
\item Não deve haver reclamação alguma sobre qualquer decisão de um árbitro no campo com o pretexto de que ele não fora correto em sua decisão sobre:
\begin{itemize}
	\item se uma bola batida foi \gls{fair} ou \gls{foul},
	\item se um corredor foi \gls{safe} ou \gls{out},
	\item se uma bola arremessada foi \gls{strike} ou \gls{ball}; ou
	\item sobre qualquer jogada cuja decisão implique uma apreciação.
\end{itemize}


Nenhuma decisão dada por um árbitro deve ser mudada, a menos que ele seja convencido de que tal decisão infrinja uma destas Regras.

 Em caso de o técnico, capitão ou qualquer das duas equipes solicitar a mudança de uma decisão baseada unicamente em um ponto das Regras, o árbitro cuja decisão esteja sendo questionada deve, se estiver em dúvida, consultar seus companheiros, antes de tomar qualquer atitude. Somente o técnico ou o capitão de uma equipe está legalmente autorizado para reclamar sobre uma decisão e pode pedir sua mudança alegando conflito com estas Regras.

\item  Em nenhuma circunstância um árbitro deve pedir para mudar uma decisão dada por seus companheiros, ou criticar ou interferir nas suas atribuições, a não ser que seja por eles requisitado para isso.

\item  Os árbitros consultados podem retificar qualquer situação em que a mudança de decisão de um árbitro, ou uma decisão demorada de um árbitro, coloque um batedor-corredor ou corredor em risco, ou coloque a equipe na defensiva em  desvantagem. Essa correção não será possível depois que um arremesso, legal ou ilegal, tiver sido efetuado, ou se todos os jogadores da equipe na defensiva tiverem abandonado o território \gls{fair}.
\end{enumerate}

\subsection{PARALISAÇÃO DA PARTIDA}
\begin{enumerate}[label=(\alph*)]
	\item Um árbitro deve paralisar a partida quando, na sua opinião, as circunstâncias justificam tal medida.
	\item A partida deve ser paralisada quando o árbitro de \gls{home} deixa sua posição	para limpar o \gls{homeplate} ou para cumprir outras obrigações que não estejam diretamente relacionadas com decisão de jogadas.
	\item  O árbitro deve paralisar a partida sempre que um batedor ou um arremessador saia da posição por um motivo justo.
	\item  Um árbitro não deve declarar \gls{time} depois que o arremessador inicia o movimento de arremesso.
	\item  Um árbitro não deve declarar \gls{time} enquanto alguma jogada esteja em  andamento.
	\item  Em caso de ferimento, exceto quando na opinião do árbitro esse ferimento seja muito grave (que pode pôr o jogador em risco), não deve ser declarado \gls{time} até que todas as jogadas em andamento tenham sido completadas ou os corredores tenham sido mantidos em suas bases.
	\item   Os árbitros não devem paralisar a partida a pedido de jogadores, \glspl{coach}
	 ou técnicos, até que todas as ações em andamento -- de ambas as equipes --
	 tenham sido completadas.
\end{enumerate}

\section*{EFEITO}
\addcontentsline{toc}{section}{EFEITOS}

\begin{description}
	\item[Regra 3.6.7 -- Paralisação da partida\gls{time} declarado devido a um ferimento grave que põe um jogador em risco] Efeito: Quando é declarado \gls{time} em caso de ferimento, a bola torna-se morta, e o(s) corredor(es) pode(m) ser autorizado(s) a avançar uma ou mais bases que ele(s) teria(m) conquistado, na opinião do árbitro, se não tivesse(m) se machucado.
\end{description}

\section{ANOTADORES}

\subsection{RESPONSABILIDADES DO ANOTADOR OFICIAL}

 Anotador Oficial:

	\begin{enumerate}[label=(\alph*)]\item   deve preparar ou mandar preparar e guardar os dados de um jogo, conforme previsto nestas Regras;
	\item  deve ser a única autoridade para anotar todas as decisões envolvendo apreciação;
	\item  deve determinar se o avanço de um batedor à primeira base é o resultado de uma batida indefensável (hit) ou de um erro; e
	\item  não deve tomar uma decisão sobre anotação que seja contraditória ou que conflite com estas Regras ou com uma decisão do árbitro.
	\end{enumerate}

\chapter{ARREMESSADOR}
\minitoc% Creating an actual minitoc

\section{DEFINIÇÕES}
\subsection{CONFERÊNCIA DEFENSIVA (\textit{CHARGED DEFENSIVE CONFERENCE})}

 Ocorre quando um árbitro concede tempo à equipe na defensiva ou paralisa a  partida:

\begin{enumerate}[label=(\alph*)]
	\item para permitir que um representante da equipe na defensiva entre no campo de jogo para comunicar-se com qualquer defensor; ou
	\item quando ou numa situação em que um defensor vá ao \gls{dugout} e deu ao árbitro razão para acreditar que ele (defensor) tenha recebido instruções.
\end{enumerate}

\subsection{\textit{CROW HOP} (PULO DO CORVO/SALTO ALAVANCA)}

 É o ato de um arremessador que:
\begin{enumerate}[label=(\alph*)]
	\item dá o impulso de um lugar que não o \gls{pitcher's plate} para soltar a bola; ou
	\item dá um passo para fora do \gls{pitcher's plate}, estabelecendo um segundo impulso (ou ponto de partida), e depois, iniciando desse novo ponto de partida, completa arremesso.
\end{enumerate}
\subsection{ARREMESSADOR ILEGAL}

 É um jogador que está legalmente no jogo, mas não pode arremessar por ter sido removido da posição de arremessador pelo árbitro.

\subsection{SALTO}

\begin{enumerate}[label=(\alph*)]
	\item O arremessador se eleva do solo em seu primeiro movimento, com um impulso à frente do \gls{pitcher's plate} sem estabelecer um segundo ponto de partida como no \gls{crow hop}.
	\item O pé de apoio pode se desprender e/ou arrastar numa ação contínua e o ímpeto criado pelo movimento (à frente) do arremessador faz com que todo o corpo e ambos os pés (o pé de apoio e o pé livre com o qual dá o passo) estejam no ar ao mesmo tempo e se direcionem ao \gls{homeplate}.
 	\item O arremesso é completado quando o arremessador toca o solo e com um movimento contínuo joga a bola em direção ao \gls{homeplate}.

\end{enumerate}

\subsection{\textit{PASSED BALL} (BOLA ARREMESSADA DEFENSÁVEL QUE PASSA PARA TRÁS DO RECEPTOR)}

 É um arremesso que deveria ter sido agarrado ou controlado pelo receptor, com um esforço normal.

\subsection{ARREMESSO}

É o ato executado pelo arremessador que consiste em jogar a bola ao batedor.

\subsection{PÉ DE APOIO}

É o pé com o qual o arremessador dá o impulso a partir do \gls{pitcher's plate}.

\subsection{ARREMESSO DE RETORNO RÁPIDO}

Arremesso efetuado com o claro propósito de pegar o batedor desprevenido, ou seja, quando ele não está devidamente posicionado no \gls{battersbox} ou enquanto ele ainda está fora de equilíbrio por causa do arremesso anterior.

\subsection{ARREMESSO ESTILINGUE (SLINGSHOT PITCH)}

 Um arremesso em que o arremessador deixa cair o braço de arremesso para o lado e para trás antes de iniciar uma ação rápida de estilingue e acelerar com o movimento para frente. Para ser um arremesso legal, o arremessador deve soltar a bola no primeiro movimento para frente além do quadril e não fazer uma  rotação completa do braço arremessador. Um arremesso de estilingue é legal em arremesso rápido, mas ilegal em arremesso modificado.

 \section{REUNIÃO DA DEFENSIVA}
 \subsection{CONFERÊNCIA DEFENSIVA}
\begin{enumerate}[label=(\alph*)]
	\item  A equipe na defensiva tem direito a somente três (3) CONFERÊNCIAS
	 DEFENSIVAS num jogo de sete \glspl{inning}.
	\item  CONFERÊNCIAS DEFENSIVAS são cumulativas, ou seja, a contagem de conferências não recomeça com a entrada de um novo arremessador.
	\item  CONFERÊNCIAS DEFENSIVAS não realizadas em sete \glspl{inning} não podem ser usadas em jogos com \glspl{inning} extras.
	\item  Num jogo com \glspl{inning} extras, é permitida somente uma CONFERÊNCIA  defensiva em cada \gls{inning} extra. Uma CONFERÊNCIA defensiva não realizada num \gls{inning} extra de um jogo não pode ser usada em \gls{inning} extra  subsequente.
	\item  Uma CONFERÊNCIA defensiva termina quando o membro da equipe na  defensiva cruza a linha de \gls{foul} para retornar ao \gls{dugout} ou um defensor retorna  ao campo.

	\item  Uma reunião inclui jogadores posicionados no campo que deixam seus postos e se dirigem ao \gls{dugout} para receber instruções, independentemente de ter havido pedido de tempo \gls{time} ou não.
\end{enumerate}

\section*{EFEITO}

Regra 4.2.1 (a) - Tentativa de conferencias excessivas

 EFEITO:

 Na quarta reunião e em cada CONFERÊNCIA defensiva adicional num jogo de sete \glspl{inning} ou em qualquer CONFERÊNCIA defensiva que exceda o limite de uma reunião por \gls{inning} num jogo com \glspl{inning} extras, o arremessador que está no jogo durante a reunião é declarado um Arremessador Ilegal, e ele não pode voltar a arremessar pelo resto do jogo, mas pode jogar em outra posição defensiva.


\subsection{O QUE NÃO É UMA CONFERÊNCIA DEFENSIVA}

 Não é considerada uma CONFERÊNCIA defensiva quando:

\begin{enumerate}[label=(\alph*)]
	\item   um técnico, \gls{coach} ou membro da equipe na defensiva comunica uma  mudança de arremessador ao árbitro de \gls{home}, antes ou depois de conversar com o arremessador;
	\item  um técnico ou \gls{coach} comunica do \gls{dugout} (banco de reservas) uma mudança ao árbitro e, depois de fazer a mudança, cruza a linha de \gls{foul} para  falar com algum defensor;
	\item  um ou mais membros da equipe na defensiva e pelo menos um defensor se reúnem durante uma CONFERÊNCIA ofensiva, contanto que todos os
	 defensores estejam posicionados e prontos para reiniciar a partida quando a  ofensiva terminar a reunião;
	\item  instruções são passadas do \gls{dugout};
	\item  um técnico/\gls{coach} que está atuando como jogador se reúne com um defensor. O árbitro pode controlar as reuniões entre o técnico-jogador/\gls{coach}- jogador e um arremessador; primeiramente, deve adverti-lo, e se isso continuar, deverá expulsá-lo; ou
	\item  um árbitro tiver paralisado o jogo.
\end{enumerate}
\section{ARREMESSO LEGAL -- REQUISITOS}
\subsection{AÇÃO PRELIMINAR  ANTES  DE  EFETUAR  UM ARREMESSO}

Antes de efetuar um arremesso, as seguintes ações devem ocorrer.
\begin{enumerate}[label=(\alph*)]
	\item Todos os jogadores têm de estar posicionados em território \gls{fair}, e o receptor  tem de estar dentro do \gls{catcher's box} e numa posição para receber o arremesso.
\item   O arremessador de posse da bola tem de estar sobre ou perto do \gls{pitcher's plate}.
\item   O arremessador tem de ter o pé de apoio em contato com o \gls{pitcher's plate} e ambos os pés dentro dos 61,00cm (24 polegadas) de extensão do \gls{pitcher's plate}. Os quadris têm de estar alinhados com a primeira e a terceira bases.

  Somente na modalidade Arremesso Modificado -- O arremessador tem de ter ambos os pés em contato com o \gls{pitcher's plate} e dentro dos 61,00cm (24 polegadas) de extensão do \gls{pitcher's plate}. Os ombros têm de estar alinhados com a primeira e a terceira bases.

\item  O arremessador tem de receber, ou aparentar estar recebendo, uma senha  do receptor enquanto está sobre o \gls{pitcher's plate}, com as mãos separadas e a  bola na luva ou na mão com a qual efetua os arremessos.

\item  Após receber o sinal, o arremessador deve levar o seu corpo inteiro a uma  imobilidade total e completa, com a bola em ambas as mãos na frente do corpo. pé com o qual dá o passo (pé livre) tem de estar imóvel no início e durante a pausa. O pé livre pode mover-se para a frente somente com o início do arremesso. Qualquer movimento para trás feito com o pé livre durante ou depois da pausa é uma ação ilegal. Essa posição tem de ser mantida por não menos de dois (2) segundos e não mais de cinco (5) segundos antes de soltar a  bola. Quando o arremessador segura a bola com ambas as mãos ao lado do corpo, deve-se considerar que ele o fez na frente do corpo.

 Somente na modalidade Arremesso Modificado -- Essa posição tem de ser mantida por não menos de dois (2) segundos e não mais de dez (10) segundos antes de soltar a bola.
\end{enumerate}

\subsection{INÍCIO DO ARREMESSO}

\begin{enumerate}[label=(\alph*)]
	\item O arremesso inicia quando o arremessador tira uma mão da bola ou faz qualquer movimento relacionado com sua maneira de arremessar.

	O arremessador não pode usar um movimento de arremesso em que, após assumir  a posição de arremesso com a bola em ambas as mãos, faz um movimento para trás e para a frente e retorna a bola para ambas mãos na frente do corpo.

	\item O pé de apoio tem de permanecer em contato com o \gls{pitcher's plate} antes do início do arremesso. Levantar o pé de apoio do \gls{pitcher's plate} e retorná-lo ao \gls{plate} criando um movimento de balanço é um ato ilegal.
\end{enumerate}

 	Somente na modalidade Arremesso Modificado -- Ambos os pés têm de  permanecer em contato com o \gls{pitcher's plate} antes do início do arremesso.

	Levantar o pé de apoio do \gls{pitcher's plate} e retorná-lo ao \gls{plate} criando um movimento de balanço é um ato ilegal.

\subsection{ARREMESSO LEGAL (ENTREGA) ARREMESSO RÁPIDO}

Todos os itens a seguir devem ocorrer para que um arremesso seja considerado arremesso legal.
\begin{enumerate}[label=(\alph*)]
	\item O arremessador tem de jogar a bola ao batedor imediatamente após fazer os movimentos de arremesso.
	\item O arremessador pode apenas fazer uma rotação quando está adotando o estilo \gls{windmill} (estilo "molinete"). Entretanto, ele pode deixar o braço cair para  lado e para trás antes de iniciar este movimento de "molinete". Isso permite que o braço passe legalmente pelo quadril duas vezes no arremesso molinete mas apenas uma vez no arremesso estilingue.
	\item  O arremesso tem de ser feito com um movimento em que a mão do arremessador fique em nível inferior ao do cotovelo (\gls{underhanded motion}) e abaixo do quadril, e o punho não pode estar mais afastado do corpo do que o cotovelo; não pode haver uma parada ou reversão do movimento para a frente.
	\item  O arremessador tem de soltar a bola com um movimento contínuo da mão e do punho para a frente, passando em linha reta e paralelamente ao corpo.
	\item  No momento de completar o arremesso, o arremessador pode dar um passo com o pé livre (pé com o qual dá o passo) simultaneamente ao ato de soltar a  bola. O passo tem de ser para a frente, na direção do batedor e dentro do espaço de 61,00cm (24 polegadas), que corresponde à extensão do \gls{pitcher's plate}.

	 Não é um passo se o arremessador desliza qualquer dos pés através do \gls{pitcher's plate}, desde que seja mantido o contato com a placa e não haja um movimento para trás da placa.

	\item  O pé de apoio tem de permanecer em contato com o \gls{pitcher's plate}, ou pode desprender-se e arrastar-se para fora da placa ou estar no ar antes que o pé livre toque o solo. O arremessador pode saltar do \gls{pitcher's plate}, aterrissar e, com um movimento contínuo, arremessar ao batedor. O pé de apoio pode acompanhar o movimento contínuo do arremessador.

	\item Todo movimento do braço com o qual faz os arremessos tem que ser  contínuo enquanto o arremessador dá o passo, impulsiona ou salta do  \gls{pitcher's plate}.
	\item   O impulso do arremessador para arrastar, pular ou saltar tem de iniciar do \gls{pitcher's plate}. O arremessador não deve dar aquele salto conhecido por \gls{crow hop} (pulo do corvo/salto-alavanca) ou impulsionar de qualquer lugar que não seja o \gls{pitcher's plate}.
	\item O braço do arremesso pode continuar o movimento desde que a rotação (\gls{windup}) não continue.
	\item   O arremessador não deve derrubar a bola, ou fazê-la rolar ou saltar, intencionalmente, a fim de evitar que ela seja batida.
	\item  O arremessador tem vinte (20) segundos após receber a bola ou depois que	 árbitro declara \gls{play} para efetuar o arremesso seguinte.

	 ARREMESSO MODIFICADO

\end{enumerate}

Todos os itens a seguir devem ocorrer para que um arremesso seja considerado arremesso legal.

\begin{enumerate}[label=(\alph*)]
	\item  O arremessador tem de jogar a bola ao batedor imediatamente após fazer os movimentos de arremesso.
	\item  O arremessador pode levar a bola para trás do dorso quando movimenta o braço para trás, contanto que não haja uma parada ou reversão do movimento para a frente e ele não use um arremesso do tipo \gls{windmill} ("molinete") ou \gls{slingshot} ("pêndulo"); não é permitido que ele faça uma rotação completa do braço quando executa o arremesso.

	\item  A bola tem de estar na parte interna do pulso do arremessador quando ele movimenta o braço para baixo e durante a conclusão do arremesso.
	\item  O arremesso tem de ser feito com um movimento em que a mão do arremessador fique em nível inferior ao do cotovelo (\gls{underhanded motion}) e
	 abaixo do quadril, e a palma da mão pode estar virada para baixo.
	\item  Ao movimentar para a frente o braço com o qual faz os arremessos:
	 i. o cotovelo tem de estar "fechado" (direcionado) para o ponto de onde vai soltar a bola; e
	 ii. os ombros e o quadril devem formar um ângulo reto com o \gls{homeplate} quando a bola sai da mão do arremessador.
	\item  A bola tem de ser solta no primeiro impulso para a frente com o braço que utiliza para arremessar, e este tem de passar o quadril. O movimento para soltar a bola tem de ser completo e suave, sem parada brusca do braço perto do quadril.
	\item   Impulsionar com o pé de apoio de um lugar que não seja o \gls{pitcher's plate} antes que o pé livre tenha deixado a placa é um \gls{crow hop} (pulo do corvo/salt0- alavanca), e é ilegal.
	\item   No momento de completar o arremesso, o arremessador tem de dar um passo simultaneamente ao ato de soltar a bola. O passo tem de ser para a frente, na  direção do batedor e dentro do espaço de 61,00cm (24 polegadas), que corresponde à extensão do \gls{pitcher's plate} projetada para a frente. O pé livre tem de ser direcionado ao \gls{homeplate}, e não precisa tocar o solo à frente de, ou ao longo de uma linha reta entre o pé de apoio e o \gls{homeplate}. Não é um passo se o arremessador desliza qualquer dos pés sobre o \gls{pitcher's plate}, desde que o contato com a placa seja mantido. Levantar o pé de apoio do \gls{pitcher's plate} e retorná-lo à placa, criando um movimento de impulso, é um ato ilegal que infringe a Regra 4.3.2 (b).
		\item  O braço do arremessador pode acompanhar o movimento para soltar a bola, desde que esse movimento não continue depois de soltar a bola.
	\item   O arremessador tem de jogar a bola ao batedor, sem intenção de fazê-la rolar ou saltar após tocar o solo a fim de evitar que ela seja batida.
	\item   O arremessador tem vinte (20) segundos após receber a bola ou depois que árbitro declara \gls{play} para efetuar o arremesso seguinte.
\end{enumerate}

\subsection{POSICIONAMENTO DA DEFESA}

\begin{enumerate}[label=(\alph*)]
	\item   Um defensor não deve agir ou ocupar uma posição com intenção antidesportiva de distrair um batedor.
	\item  Quando o corredor da 3a base está tentando anotar ponto por meio de um  \gls{squeeze play} (jogada de pressão) ou \gls{steal} (roubo de base), nenhum defensor  pode ficar sobre ou na frente do \gls{homeplate} sem estar de posse da bola ou tocar o batedor ou o seu \gls{bat}.
\end{enumerate}

\subsection{SUBSTÂNCIAS ESTRANHAS}
\begin{enumerate}[label=(\alph*)]
	\item   Nenhum membro da equipe na defensiva pode, em nenhum momento durante jogo, aplicar uma substância estranha na bola. Um arremessador que lambe os dedos da mão com a qual efetua os arremessos tem de secá-los antes de ter contato com a bola.
	\item  Sob a supervisão e controle de um árbitro, um saco de resina (breu) pode ser usado pelo arremessador para secar as mãos e deve ser mantido no chão atrás do \gls{pitcher's plate}, dentro do círculo de arremesso, quando não estiver em uso. Durante o mau tempo ou condições de campo  molhado, com a permissão de um árbitro, o saco de resina pode ser colocado no bolso de trás do arremessador.
	\item  É permitido o uso de pano aprovado2 impregnado somente com resina, para secar a mão, e esse pano tem de ser guardado no bolso traseiro ou mantido preso no cinto.
	\item  Nenhum defensor pode aplicar resina na bola ou na luva para depois pôr a bola em contato com a resina.
	\item  O arremessador não pode usar fita em seus dedos, ou um \gls{sweatband} (faixa para enxugar suor), bracelete, ou outros objetos similares no pulso ou antebraço	(do braço com o qual faz os arremessos). Se um arremessador necessita usar \gls{sweatband} ou fita em seu braço com o qual faz os arremessos em razão de um ferimento, ambos os braços têm de ser cobertos com uma camiseta.
\end{enumerate}

\subsection{O RECEPTOR}
\begin{enumerate}[label=(\alph*)]\item   O receptor tem de permanecer no \gls{catcher's box} até que seja completado o arremesso.
\item  O receptor tem de devolver a bola imediatamente e diretamente ao arremessador após cada arremesso, inclusive depois de um \gls{foulball}, exceto:
	 \begin{enumerate}[label=\roman*.]
	\item depois de uma eliminação por \gls{strike} (\gls{strikeout});
	\item quando o batedor se torna um batedor-corredor;
	\item quando há algum corredor em base;
	\item quando ele pega uma bola \gls{fair} perto da linha de \gls{foul} e lança a qualquer base para tentar eliminar um corredor; ou
	\item quando, em situação de \gls{check swing} em terceiro \gls{strike} não agarrado, ele lança à 1a base para eliminar o batedor-corredor.
\end{enumerate}
\end{enumerate}
\subsection{LANÇAMENTO A UMA BASE}
 Após ter assumido a posição de arremesso, o arremessador não deve lançar ou simular um lançamento a uma base durante uma situação de bola viva enquanto seu pé está em contato com o \gls{pitcher's plate}. Se isso ocorrer durante uma Jogada de Apelação com bola viva, a apelação será cancelada. O arremessador pode parar ou sair da posição de arremesso dando um passo para trás do  ****

  Aprovado pela WBSC em jogos internacionais e pelo CT ( CBBS ) em campeonatos nacionais \gls{pitcher's plate} antes de separar as mãos. Se o passo for dado para a frente ou  para os lados, será aplicada a penalidade de um Arremesso Ilegal.

\section*{EFEITOS}  (4.3.1 até 4.3.7)
\addcontentsline{toc}{section}{EFEITOS}

 Regra 4.3.3 (***)  -- Um arremessador não solta a bola em 20 segundos

 Efeito: É concedido um \gls{ball} ao batedor.

 Regra 4.3.4 (a) -- Um defensor age de maneira antidesportiva ou se posiciona para distrair o batedor. Não é necessário que seja efetuado um arremesso

 Efeito: O jogador é expulso do jogo.

 Regra 4.3.4 (b) -- Um defensor fica parado na frente do \gls{homeplate} sem estar de posse da bola ou toca o batedor ou o \gls{bat} num possível \gls{squeeze play} (jogada de pressão)

 Efeito: A bola torna-se morta. O batedor é autorizado a ir à primeira base por Obstrução, é declarado um Arremesso Ilegal, e todos os corredores avançarão uma base.

 Regra 4.3.5 -- Um membro da equipe na defensiva continua aplicando uma substância estranha na bola ou continua infringindo qualquer dispositivo da  Regra 4.3.5

 Efeito: Um arremesso ilegal é declarado. Se a ação ilegal continuar, o arremessador é retirado do jogo e considerado umarremessador ilegal.

 Regra 4.3.6 (b) -- Um receptor não devolve a bola diretamente ao arremessador quando não há corredor(es) em base

 Efeito: É concedido um \gls{ball} ao batedor.

 Regras 4.3.1 a 4.3.7 -- Por uma infração das Regras 4.3.1 a 4.3.7 --Movimentos de arremesso impróprios [exceção para os efeitos expostos acima pelas
 Regras 4.3.3 (***)  , 4.3.5 e 4.3.6 (b) ] resultam em um Arremesso Ilegal sendo declarado

Efeito: Isto é uma Bola Morta Demorada, e devem ser aplicadas as seguintes normas e penalidades.

\begin{enumerate}[label=(\alph*)]
	\item   Se o Arremesso Ilegal não for batido, será concedido um \gls{ball} extra ao batedor (será concedida a primeira base se for o quarto \gls{ball}), e cada corredor  avançará uma base. Se um corredor avançar legalmente num Arremesso Ilegal, \gls{passedball} (arremesso que poderia ter sido agarrado ou controlado pelo receptor, com um esforço normal) ou lançamento descontrolado do receptor, qualquer base extra obtida poderá ser mantida. Se o corredor for declarado \gls{out} depois de avançar uma base, a eliminação será mantida.

	\item  Se o batedor bater o Arremesso Ilegal, a equipe na ofensiva terá o direito de optar pela aplicação da penalidade de um Arremesso Ilegal ou pelo resultado da jogada. Se o batedor bater o Arremesso Ilegal e alcançar a primeira base, e se todos os outros corredores avançarem pelo menos uma base na jogada, o Arremesso Ilegal será anulado, e todas as ações resultantes da jogada serão mantidas; nenhuma opção será dada.

	\item  Se o batedor errar ao tentar bater um Arremesso Ilegal depois de dois \glspl{strike}, o receptor derruba a bola mas consegue eliminar o batedor-corredor na primeira base, e os outros corredores avançam pelo menos uma base, a equipe na ofensiva tem o direito de optar pela aplicação da penalidade de um Arremesso Ilegal ou pelo resultado da jogada. Se o batedor-corredor alcançar a primeira base em consequência de um terceiro \gls{strike} não agarrado e todos os outros corredores avançarem pelo menos uma base, o Arremesso Ilegal será anulado, e todas as ações resultantes da jogada serão mantidas; nenhuma opção será dada. (tempo verbal)

	\item  Se o técnico da equipe na ofensiva não aceitar o resultado da jogada, a bola ficará morta. O árbitro concederá um \gls{ball} ao batedor (a primeira base se for quarto \gls{ball}) e todos os corredores serão autorizados a avançar uma base.

	\item  Se um Arremesso Ilegal atingir o batedor, a bola torna-se morta; o batedor é autorizado a ir à 1a base e todos os corredores avançam uma base. Nenhuma opção será dada.
\end{enumerate}
\section{ARREMESSOS DE AQUECIMENTO}

\begin{enumerate}[label=(\alph*)]
	\item No início do primeiro \gls{inning} de ambas as equipes ou quando um arremessador substitui outro, são permitidos no máximo cinco (5) arremessos de aquecimento, que devem ser feitos ao receptor ou a outro membro da equipe na  defensiva dentro de um (1) minuto. No início de cada metade de \gls{inning} (depois do primeiro \gls{inning}), o arremessador do \gls{inning} anterior tem um (1) minuto para  efetuar três (3) arremessos de aquecimento. Se tiver passado ou estiver prestes a passar um minuto sem que o arremessador tenha iniciado os arremessos de aquecimento, o árbitro lhe permitirá somente um (1) arremesso.

 i. Se a equipe na defensiva não utiliza um jogador do \gls{bench} para receber os arremessos de aquecimento enquanto o receptor (que estava ocupando uma base, ou estava no \gls{battersbox} ou no círculo do Batedor
 Prevenido) se prepara para ocupar a sua posição, o árbitro permitirá somente um (1) arremesso, exceto quando um novo arremessador entra no jogo;

 ii. Isso não se aplica se o árbitro retarda o início ou o reinício da partida por causa de substituição, reunião, ferimentos ou outras razões citadas  pelo árbitro de \gls{home}.

\item  A partida fica paralisada enquanto são efetuados os arremessos de aquecimento.
\item  Um arremessador que retorna para arremessar na mesma metade de \gls{inning} não tem direito a arremessos de aquecimento.
\end{enumerate}

\section*{EFEITO}
\addcontentsline{toc}{section}{EFEITOS}

 Regra 4.4 -- Arremessos de aquecimento excedentes
 Efeito: Será concedido um \gls{ball} ao batedor por cada arremesso excedente.

\section{ARREMESSO NULO}

O arremesso é anulado, a bola torna-se morta e todas as ações subsequentes a esse arremesso são canceladas por um árbitro quando:

\begin{enumerate}[label=(\alph*)]
	\item   o arremessador efetua o arremesso enquanto a partida está paralisada;
	\item  o arremessador tenta um retorno rápido da bola:

	 i. antes que o batedor tenha se posicionado no \gls{battersbox}; ou
	 ii. quando o batedor está fora de equilíbrio em consequência de um arremesso anterior.

	\item  um corredor é declarado \gls{out} por ter deixado uma base antes do

	arremessador soltar a bola de sua mão;
	\item  o arremessador inicia o arremesso antes que um corredor tenha retocado sua  base após um \gls{foulball} ter sido declarado; ou
	\item  um técnico, \gls{coach} ou jogador declara ou pede tempo \gls{time}, usa qualquer outra palavra ou frase, ou pratica qualquer outro ato enquanto a bola está viva e em jogo, com o evidente propósito de tentar fazer o arremessador cometer um Arremesso Ilegal. Nesse caso, o árbitro deve advertir a equipe infratora, e se qualquer membro da equipe advertida repetir esse tipo de ato, será expulso do jogo.
\end{enumerate}
\section{BOLA DERRUBADA}

 Se a bola escapa ou cai da mão do arremessador durante o arremesso:
	\begin{enumerate}[label=(\alph*)]\item   o árbitro de \gls{home} declara um \gls{ball} ao batedor;
		\item  a bola permanece em jogo; e
		\item  um corredor pode avançar a seu próprio risco.
	\end{enumerate}

\section{RETORNO DE ARREMESSADOR}

Não há limite quanto ao número de vezes que um jogador pode retornar à posição de arremessador, contanto que esse jogador não tenha saído do jogo ou não tenha sido declarado um Arremessador Ilegal por um árbitro.

\section{ARREMESSADOR ILEGAL}
 Um jogador que tenha sido declarado um Arremessador Ilegal em razão de excesso de Conferências Defensivas não pode retornar à posição de arremessador em nenhum momento, pelo resto do jogo. O arremessador ilegal pode jogar em outras posições defensivas e até continuar no ataque até o final da partida

\section*{EFEITO}
\addcontentsline{toc}{section}{EFEITOS}

Regra 4.8 -- Arremessador Ilegal. Arremessador declarado ilegal retorna à posição de arremessador e efetua um arremesso, legal ou ilegal

 Efeito:
	\begin{enumerate}[label=(\alph*)]\item   O Arremessador Ilegal é expulso.
	\item  Se o Arremessador Ilegal é descoberto antes do arremesso seguinte, o	técnico da equipe na ofensiva tem a opção de:

	 i. aceitar o resultado da jogada; ou
	 ii. ter a jogada anulada, e nesse caso o batedor bate novamente, assumindo a
	 contagem de \gls{ball} e \gls{strike} que tinha antes da descoberta do Arremessador
	 Ilegal; e
	 iii. cada corredor retorna à base que estava ocupando no momento do arremesso. Um arremesso não é uma jogada

\end{enumerate}

\chapter{BATEDOR E BATEDOR-CORREDOR}
\minitoc% Creating an actual minitoc

\section{DEFINIÇÕES}

\subsection{BASE POR \textit{BALLS} OU \textit{WALK}}
 Ocorre quando quatro arremessos são julgados \gls{ball} pelo árbitro de \gls{home}, incluindo arremessos ilegais. Ao batedor é concedido a primeira base. A bola é viva.Pode ser solicitado pelo técnico do ataque para evitar confronto.

\subsection{CAMINHO DA BASE}
 É a linha reta entre uma base e a posição do corredor no momento em que um jogador da defensiva está tentando (ou está prestes a tentar) tocá-lo com a bola.

\subsection{BOLA BATIDA}
 É qualquer bola arremessada que atinge o \gls{bat} ou é por ele atingida e cai em território \gls{fair} ou \gls{foul}. Não é necessário que tenha havido intenção de bater a bola.

\subsection{BATEDOR}
 É um jogador da ofensiva que entra no \gls{battersbox} com a intenção de ajudar sua equipe a anotar pontos. Ele continua sendo um batedor até que seja declarado \gls{out} pelo árbitro ou se torne um batedor-corredor.

\subsection{BATEDOR-CORREDOR}
 É um jogador que após terminar a sua vez de bater não foi ainda declarado \gls{out} nem tocou a primeira base.

\subsection{ORDEM DE BATEDORES}
 É a lista oficial de jogadores da ofensiva, cujos nomes devem estar relacionados na escalação da equipe na sequência em que devem bater.
\subsection{BOLA BLOQUEADA}
 Bola Bloqueada é uma bola batida, lançada ou arremessada que:
	\begin{enumerate}[label=(\alph*)]
		\item   fica alojada na cerca ou na roupa/equipamento do árbitro;
		\item  é tocada, parada ou manuseada por uma pessoa que não está atuando no jogo;
		\item  toca qualquer objeto que não é parte do equipamento oficial ou da área  de jogo;
		\item  é tocada por um jogador da defensiva que está em contato com o solo fora da área de jogo (as linhas são consideradas parte da área de jogo).
		\item  toca acidentalmente um \gls{coach} de base (dentro ou fora do \gls{coachsbox}) não é uma Bola Bloqueada, e permanece em jogo.
\end{enumerate}

\subsection{\textit{BUNT} (TOQUE)}
 É uma bola batida por meio de um toque intencional com o \gls{bat} (sem girar o \gls{bat}), a fim de fazê-la rolar lentamente dentro do campo de jogo.
\subsection{\textit{CATCH} (PEGADA LEGAL)}
 Ocorre uma pegada legal quando um defensor pega, com sua(s) mão(s) ou luva, uma bola batida ou lançada.
\begin{enumerate}[label=(\alph*)]
	\item   Para a pegada ser válida, o defensor tem de segurar a bola por um tempo suficiente para provar que ela foi agarrada firmemente, e/ou que seu ato de liberar foi voluntário e intencional. A pegada é válida se o jogador derruba a bola  no momento de retirá-la da luva -- depois de tê-la segurado dentro dela -- ou no ato do lançamento.

	\item  Se a bola está apenas sustentada no(s) braço(s) do defensor ou se sua queda ao solo é evitada por alguma parte do seu corpo, equipamento ou roupa, deve-se considerar que a pegada não está completa até que a bola esteja dominada  com sua(s) mão(s) ou luva.
	\item  Para a pegada ser válida, os pés do defensor têm de estar dentro do campo de jogo, tocando a linha demarcatória da área fora de jogo, ou estar no ar depois de deixar a área de bola viva. Se o jogador tem o controle da bola quando retorna ao solo na área de bola morta, a pegada é legal. Quando um jogador que está	 na área de bola morta retorna à área de bola viva, tem de ter ambos os pés	 tocando a área de jogo, antes de ter contato com a bola, para a pegada ser	 válida.
	\item  Não é uma pegada legal se um defensor (enquanto está tentando controlar a bola) colide com outro jogador, árbitro ou uma cerca, ou cai ao solo, e em razão disso derruba a bola.
	\item  Uma bola batida que, enquanto em voo, tem contato com qualquer coisa, exceto um jogador da defensiva, é tratada da mesma forma como se ela tivesse tocado o solo.
\end{enumerate}
\subsection{CONFERÊNCIA OFENSIVA}
 Ocorre quando a equipe na ofensiva pede a paralisação da partida para permitir que o técnico ou outro representante da equipe se reúna com qualquer membro de sua equipe. Isso inclui o batedor, o corredor, o Batedor Prevenido e os \glspl{coach} entre si.

\subsection{BOLA MORTA}
 É uma bola que não está em jogo. Nenhuma jogada pode ocorrer.

\subsection{BOLA MORTA DEMORADA (\textit{DELAYED DEAD BALL})}

 É uma situação de jogo em que a bola permanece viva até a conclusão de uma jogada; quando a jogada estiver totalmente concluída, e se for necessário, um árbitro declarará que a bola está morta e aplicará a regra apropriada.

\subsection{EQUIPAMENTO  OU UNIFORME FORa DO LUGAR APROPRIADO}

Quando um defensor toca intencionalmente ou pega uma bola batida \gls{fair}, uma bola lançada, ou uma bola arremessada, com seu boné, capacete, máscara, protetor, bolso, luva destacada da mão, ou qualquer parte do seu uniforme que esteja fora do lugar apropriado em seu corpo.

\subsection{BASE DESLOCADA}
 É uma base que foi tirada de sua posição correta.
\subsection{JOGADA DUPLA (\textit{DOUBLE PLAY})}
 É uma jogada executada pela equipe na defensiva, na qual dois jogadores da equipe na ofensiva são declarados \gls{out} em consequência de ação contínua.

\subsection{BOLA \textit{FAIR}}
 É uma bola batida legalmente e viva que:
\begin{enumerate}[label=(\alph*)]\item   permanece ou é tocada em ou sobre território \gls{fair} entre o \gls{homeplate} e a
 primeira base ou entre o \gls{homeplate} e a terceira base;
\item  passa pela primeira ou terceira base (rolando ou pulando) em ou sobre território \gls{fair}, independentemente de onde ela toque depois de passar sobre a  base;
\item  toca a primeira, segunda ou terceira base;
\item  enquanto está em ou sobre território \gls{fair}, toca o corpo ou a roupa de  um árbitro ou jogador;
\item  cai primeiro em território \gls{fair} além da primeira e terceira base;
\item  enquanto está sobre território \gls{fair}, sai do campo de jogo, passando por cima  da cerca do campo externo ("outfield");
\item   enquanto em voo, atinge o poste da linha de \gls{foul};
\item   é julgada \gls{fair fly} (\gls{fly} em território \gls{fair}) de acordo com a posição da bola em relação à linha de \gls{foul}, incluindo o poste de \gls{foul}, e não pela posição do defensor -- se ele se encontra em território \gls{fair} ou \gls{foul} -- no momento em que toca a bola. Não importa se a bola toca primeiro o território \gls{fair} ou \gls{foul}, desde que ela não toque qualquer coisa estranha ao terreno natural em território \gls{foul},
 e preencha todos os outros requisitos de uma bola \gls{fair}. A posição da bola no momento em que tem contato com o defensor determina se a batida é \gls{fair} ou
 \gls{foul}, sem levar em consideração se ela poderia rolar, sem ser tocada, para o território \gls{foul} ou \gls{fair}.
\end{enumerate}
 \subsection{SIMULAÇÃO DE TOQUE (\textit{FAKE TAG})}
 É uma forma de Obstrução em que um defensor, sem estar de posse da bola, impede a ação de um corredor que está avançando ou retornando a uma
 base. Não é necessário que o corredor pare ou deslize; a simples diminuição da velocidade quando ocorre uma simulação de toque caracteriza uma  Obstrução.

\subsection{BOLA \textit{FLY}}

 É uma bola batida para o ar.

\subsection{\textit{OUT} FORÇADO}

É aquela jogada que pode ser feita somente quando um corredor perde o direito à base que está ocupando porque o batedor se torna um batedor-corredor, e  antes que esse batedor-corredor (ou um corredor subsequente) tenha sido declarado \gls{out}. Numa jogada de apelação, a eliminação forçada é determinada pela situação forçada no momento em que é feita a apelação, e não no momento da infração. Se a situação forçada tiver deixado de existir em razão da eliminação de um corredor subsequente, antes da apelação, a jogada não será mais de eliminação forçada. Se um corredor forçado a avançar, após tocar a base seguinte, recuar por qualquer razão em direção à base que havia ocupado anteriormente, a \gls{jogadaforcada} será restabelecida.

\subsection{BOLA \textit{FOUL}}
 É uma bola legalmente batida que:

\begin{enumerate}[label=(\alph*)]
	\item   permanece em território \gls{foul} entre o \gls{homeplate} e a primeira base ou entre \gls{homeplate} e a terceira base;
	\item  passa pela primeira ou terceira base (rolando ou pulando) em ou sobre território \gls{foul};
	\item  cai primeiro em território \gls{foul} além da primeira ou terceira base;
	\item  enquanto está em ou sobre território \gls{foul} toca o corpo, o equipamento de jogo -- em uso ou removido do lugar onde normalmente é usado --, a roupa de um árbitro ou jogador, ou qualquer objeto estranho ao terreno natural;
	\item  toca o batedor ou o \gls{bat} em suas mãos pela segunda vez enquanto ele está dentro do \gls{battersbox};
	\item  vai diretamente do \gls{bat} -- sem subir além da cabeça do batedor -- para qualquer parte do corpo ou equipamento do receptor e é pega por outro defensor;
	\item   atinge o \gls{pitcher's plate} e rola sem ser tocada para o território \gls{foul}, antes de alcançar a primeira ou terceira base; ou
	\item   é julgada \gls{foul fly} (\gls{fly} em território \gls{foul}) de acordo com a posição da bola em relação à linha de \gls{foul}, incluindo o poste de \gls{foul}, e não pela posição do defensor -- se ele se encontra em território \gls{fair} ou \gls{foul} -- no momento em que toca a bola. A posição da bola no momento em que tem contato com o defensor determina se a batida é \gls{fair} ou \gls{foul}, sem levar em consideração se ela poderia rolar sem ser tocada para o território \gls{foul} ou \gls{fair}.
\end{enumerate}

\subsection{\textit{Foul tip}}
\begin{enumerate}[label=(\alph*)]
	\item   É uma bola batida que:

	 i. vai diretamente do \gls{bat} às mãos ou à luva do receptor;
	 ii. não sobe além da cabeça do batedor; e
	 iii.é agarrada legalmente pelo receptor.

	\item  Por cada \gls{foultip} é contado um \gls{strike}, e a bola permanece em jogo.

	 Não é uma pegada legal se a bola é agarrada no rebote, a menos que ela tenha tocado primeiro a mão ou a luva do receptor.
\end{enumerate}

\subsection{\textit{HIT BY PITCH} (ATINGIDO POR ARREMESSO)}

 É quando uma bola arremessada toca qualquer parte do corpo do batedor (incluindo suas mãos ou sua roupa) que esteja dentro do \gls{battersbox}, desde que ele não tenha girado o \gls{bat} para tentar bater essa bola ou o arremesso não tenha sido declarado \gls{strike}. Não importa se a bola toca o solo antes de atingir batedor.

\subsection{BOLA BATIDA ILEGALMENTE}
 Ocorre uma batida ilegal quando o batedor bate a bola para o território

\gls{fair} ou \gls{foul}:
\begin{enumerate}[label=(\alph*)]
	\item   enquanto um pé está completamente fora do \gls{battersbox} e sobre o solo no momento em que o \gls{bat} tem contato com a bola;
	\item  enquanto qualquer parte de um pé está tocando o \gls{homeplate} no momento em que o \gls{bat} tem contato com a bola;
	\item  com um \gls{bat} ilegal, não aprovado ou adulterado; ou
	\item  depois de ter dado um passo com qualquer dos pés inteiramente para fora do \gls{battersbox}, retorna e toca a bola com o \gls{bat} enquanto está com ambos os pés dentro do \gls{battersbox}.
\end{enumerate}

\subsection{BOLA PEGA ILEGALMENTE}

 Ocorre uma pegada ilegal quando um defensor pega uma bola batida, lançada ou arremessada, usando seu boné, máscara, luva ou qualquer parte do seu uniforme enquanto tais itens estão fora do lugar apropriado.

\subsection{BOLA EM VOO}

É qualquer bola batida, lançada ou arremessada que não tenha ainda tocado o solo ou algum objeto, exceto um defensor.

\subsection{\textit{INFIELD FLY}}

 É uma bola \gls{fair fly} (exceto um \gls{line drive} ou um \gls{fly} resultante de \gls{bunt} ) que pode ser pega por um defensor do campo interno mediante um esforço normal, na seguinte situação: a primeira e segunda base, ou a primeira, segunda e terceira base estão ocupadas e há menos de dois \glspl{out}. O arremessador, o receptor e qualquer defensor do campo externo posicionado no campo interno, na jogada, serão considerados defensores do campo interno para os propósitos desta regra.

\subsection{EM RISCO}

É um termo que indica que a bola está em jogo e um jogador da ofensiva pode ser declarado \gls{out}.

\subsection{BASE POR \textit{BALLS} INTENCIONAL OU \textit{WALK}}
%% INTENCIONAL
 É um lance em que a equipe na defensiva concede a primeira base ao batedor, sem arremessar quatro \glspl{ball}. A bola é morta.


\subsection{BOLA \textit{FLY} DERRUBADA INTENCIONALMENTE}

 É um lance em que com menos de dois \glspl{out} e corredor na primeira base, um defensor derruba, intencionalmente, uma bola \gls{fair fly} (depois de tê-la controlado com a mão ou luva), incluindo um \gls{line drive} ou um \gls{fly} resultante de \gls{bunt} , que pode ser pega por um defensor do campo interno mediante um  esforço normal. Um \gls{trappedball} ou um \gls{fly} que o defensor deixa cair ao solo sem tocar a bola não é considerado como uma bola derrubada intencionalmente.

 Se for declarado um \gls{infieldfly}, a Regra de \gls{infieldfly} terá prioridade.

 \subsection{INTERFERÊNCIA}

 Ocorre uma Interferência quando:

\begin{enumerate}[label=(\alph*)]
	\item   um jogador ou membro da equipe na ofensiva impede, estorva ou confunde um jogador da defensiva que está tentando executar uma jogada;
	\item  um árbitro impede que o receptor tente fazer um lançamento para eliminar um corredor que está fora da base;
	\item  um árbitro ou corredor é atingido por uma bola batida \gls{fair}:

	 i. antes dela tocar um defensor, incluindo o arremessador;
	 ii. antes dela passar um defensor do campo interno, exceto o arremessador, sem
	 ser tocada; ou
	 iii. depois dela passar um defensor, exceto o arremessador, e na opinião do árbitro outro defensor teria chance de realizar um \gls{out}.

	\item  um espectador entra no campo de jogo ou alcança o campo de jogo e impede que um defensor apanhe a bola, ou tem contato com a bola sobre a  qual um defensor está tentando fazer uma jogada.
\end{enumerate}

\subsection{\textit{LINE DRIVE} (BOLA BATIDA QUE VAI EM LINHA RETA)}

 É uma bola batida em voo que vai em linha reta, com força e diretamente, para dentro do campo de jogo.

\subsection{OBSTRUÇÃO}

Ocorre uma Obstrução quando:
\begin{enumerate}[label=(\alph*)]
	\item   um jogador ou membro da equipe na defensiva estorva ou impede que um batedor gire o \gls{bat} ou bata uma bola arremessada; ou
	\item  um defensor impede o avanço de um batedor-corredor ou corredor que está correndo as bases legalmente enquanto:

	\begin{enumerate}[label= \roman*.]
		\item não está de posse da bola;
		\item não está em ação para defender uma bola batida;
		\item simula um toque (\gls{fake tag}), sem estar de posse da bola;
		\item tendo a posse da bola, empurra um corredor para fora da base; ou
		\item tendo a posse da bola, não está em ação para fazer uma jogada sobre batedor-corredor ou corredor.
	\end{enumerate}
\end{enumerate}

\subsection{BATEDOR PREVENIDO (\textit{ON DECK BATTER} )}
 É o jogador da ofensiva cujo nome segue o do batedor do turno na ordem de
 batedores.

\subsection{JOGADA OPCIONAL}
 É uma jogada em que o técnico/\gls{coach} da equipe na ofensiva tem a alternativa de aceitar a penalidade da ação ilegal ou o resultado da jogada. Tal escolha pode  ser feita nos seguintes casos:

	\begin{enumerate}[label= \roman*.]
	\item obstrução cometida pelo receptor;
	\item uso de uma luva ilegal;
	\item uma substituição ilegal;
	\item um arremesso ilegal; ou
	\item um arremessador ilegal retorna ao jogo e efetua arremessos.
	\end{enumerate}

\subsection{\textit{OVER-SLIDE} (ULTRAPASSAGEM DA BASE)}

 Quando um batedor-corredor ou corredor desliza para uma base que está tentando alcançar, ultrapassa-a e perde o contato com ela, o batedor-corredor/corredor corre o risco de ser declarado \gls{out}. Um batedor-corredor pode ultrapassar a primeira base sem correr o risco de ser declarado \gls{out}, desde que regresse imediatamente à primeira base.

\subsection{MAU LANÇAMENTO (\textit{OVERTHROW})}

É uma bola lançada de um defensor a outro, que vai além das linhas que  delimitam o campo de jogo ou se torna uma bola bloqueada.

\subsection{\textit{PICK-OFF PLAY}}

Uma tentativa da equipe defensiva de eliminar um corredor que está fora de sua base como resultado de uma bola arremessada pelo \gls{pitcher} ou lançada pelo
\gls{catcher}.

\subsection{JOGADA (\textit{PLAY})}

 Depois de um arremesso e enquanto a bola estiver viva:

	\begin{enumerate}[label=(\alph*)]
		\item   um batedor recebe um arremesso estando dentro do bater box, movimenta o bastão para a bola ou a acerta e corre para a primeira base;
		\item  depois que um arremesso sai das mãos do arremessador algum corredor tenta  roubar a próxima base ou avança após uma bola batida; ou
		\item  Algum jogador do time de defes tenta eliminar um batedor corredor ou corredor entre bases por toque ou em \gls{jogadaforcada}.
	\end{enumerate}

\subsection{CORREDOR}
 É um jogador da equipe na ofensiva que, ao terminar a sua vez de bater, conseguiu alcançar a primeira base e ainda não está declarado \gls{out}.

\subsection{\textit{SLAP HIT} (BATIDA 'COLOCADA')}

 É uma bola batida -- exceto um \gls{bunt}  -- com um movimento curto e controlado com o \gls{bat} (o \gls{bat} é movimentado de cima para baixo), e não com um \gls{swing} completo. Os dois tipos mais comuns de "slap hit" são:

\begin{enumerate}[label=(\alph*)]\item   aquele em que o batedor assume uma posição como se fosse executar um \gls{bunt} , mas depois, ou impulsiona a bola contra o solo fazendo um \gls{swing} rápido e curto ou empurra (com o \gls{bat}) a bola para o campo interno; ou
\item  aquele em que o batedor dá passos acelerados (dentro do \gls{battersbox}) na  direção do arremessador antes de bater o arremesso fazendo um \gls{swing} rápido e curto ou antes de empurrar (com o \gls{bat}) a bola para o campo interno.
\end{enumerate}

\subsection{\textit{SQUEEZE PLAY} (JOGADA DE PRESSÃO ou APERTADA)}

 É uma jogada em que a equipe na ofensiva, com um corredor na terceira base, tenta anotar ponto com esse corredor por meio de um toque na bola dado pelo batedor.

\subsection{ROUBO DE BASE}

 É uma jogada em que um corredor tenta avançar à base seguinte ou ao \gls{homeplate} durante ou depois de um arremesso ao batedor.

\subsection{ZONA DE \textit{STRIKE}}

 É o espaço sobre qualquer parte do \gls{homeplate} entre a parte inferior do esterno (caixa torácica) e a parte inferior da rótula dos joelhos do batedor quando ele  assume a sua postura habitual para bater a bola arremessada.

 (Somente Arremesso Modificado -- O espaço sobre o \gls{homeplate} entre as axilas do batedor e a parte superior de seus joelhos quando ele assume a sua postura habitual para bater a bola arremessada.) A postura habitual para bater a bola é aquela que o batedor assume depois que a bola é arremessada, quando ele decide se tenta ou não batê-la.

\subsection{\textit{TAG} (TOQUE)}

 Um \gls{tag} legal é o ato de um defensor tocar com a posse de bola na luva:

\begin{enumerate}[label=(\alph*)]
	\item   um batedor-corredor ou corredor que esteja fora da base, com a bola firmemente segurada em sua mão ou luva. A bola não será considerada firmemente segurada se ela estiver "pipocando" ou for derrubada pelo defensor  depois de tocar o batedor-corredor ou corredor, a menos que esse jogador bata intencionalmente na bola que o defensor esteja segurando em sua mão ou luva. corredor tem de ser tocado com a mão ou luva que esteja segurando a bola; ou
	\item  uma base com a bola firmemente segurada em sua mão ou luva. A base pode ser tocada com qualquer parte do corpo para ser um \gls{tag} legal (por exemplo, o defensor pode tocar a base com um pé, com uma mão, sentar sobre a base,  etc.). Isso deve ser aplicado a qualquer eliminação forçada ou jogada de apelação.
\end{enumerate}

\subsection{ATO DE RETOCAR UMA BASE (\textit{TAGGING UP})}

 É o ato de um corredor que volta para sua base, ou nela permanece, antes de avançar legalmente numa bola batida \gls{fly} após o primeiro contato do defensor  com a bola.

\subsection{REGRA DO TERCEIRO \textit{STRIKE}}

 Esta regra deve ser aplicada quando o receptor não agarra o terceiro \gls{strike} antes que a bola toque o solo e:

	\begin{enumerate}[label=(\alph*)]\item   há menos de dois \glspl{out} e a primeira base não está ocupada; ou
		\item  há dois \glspl{out}.
	\end{enumerate}

\subsection{LANÇAMENTO}

É o ato de um defensor jogar a bola a outro defensor.

\subsection{\textit{TRAPPED BALL}}
 \gls{trappedball} é:

\begin{enumerate}[label=(\alph*)]
	\item   uma bola batida legalmente para o ar (\gls{fly}) ou uma bola batida que vai em linha reta (\gls{line drive}) que toca o solo ou uma cerca antes de ser pega por um defensor;
	\item  uma bola batida legalmente para o ar (\gls{fly}) que é pega contra a cerca (prensada entre a luva e a cerca), com a luva ou com a mão;
	\item  uma bola lançada a qualquer base para efetivar um \gls{out} forçado ("force out"), que o defensor consegue segurar colocando a luva SOBRE a bola que está tocando o solo e não SOB a bola; ou
	\item  uma bola arremessada \gls{strike} que toca o solo antes de o receptor pegá-la.
\end{enumerate}

\subsection{JOGADA TRIPLA}
 É uma jogada de ação contínua da defensiva na qual três jogadores da ofensiva são declarados \gls{out}s.

\subsection{TURNO DE BATER}
 Inicia quando um batedor entra no \gls{battersbox} e continua até que ele seja declarado \gls{out} ou se torne um batedor-corredor.

\subsection{ARREMESSO DESCONTROLADO (\textit{WILD PITCH})}

 É um arremesso tão alto, tão baixo, ou tão fora do \gls{homeplate} que o receptor não consegue parar ou controlar com um esforço normal.

\subsection{LANÇAMENTO DESCONTROLADO}

É um lançamento em que a bola atirada de um defensor a outro não pode ser pega ou controlada e permanece em jogo.

\section{CONFERÊNCIA OFENSIVA}

\begin{enumerate}[label=(\alph*)]
	\item   Não é uma CONFERÊNCIA ofensiva quando um arremessador está vestindo um agasalho enquanto está sobre uma base, ou quando jogadores da ofensiva  se reúnem enquanto a equipe na defensiva está reunida, ou o jogo está paralisado, desde que estejam prontos para jogar quando a reunião da defensiva terminar, ou o árbitro reiniciar o jogo.
	\item  É permitida somente uma conferência de ataque por \gls{inning}.
\end{enumerate}
\section*{EFEITO}

Regra 5.2 (b) -- Segunda Conferência
 Efeito: Expulsão do \gls{coach} ou técnico que insistir por uma segunda
 conferência

\section{BATEDOR PREVENIDO}

\begin{enumerate}[label=(\alph*)]
	\item  No início de um \gls{inning}, é o jogador que deve começar batendo; ele tem de permanecer no Círculo do Batedor Prevenido até ser chamado ao "batter's box".
	\item  Uma vez iniciado o \gls{inning}, é o jogador da ofensiva que, na escalação de batedores, é o próximo a entrar no \gls{battersbox}.
	\item  O Batedor Prevenido:
	\begin{enumerate}[label=\roman*.]
		\item pode permanecer dentro de quaisquer dos Círculos do Batedor Prevenido, de modo que fique atrás do batedor, e não do lado aberto do batedor. O Batedor prevenido, no intervalo de innings, deve utilizar o circulo de aquecimento mais próximo do seu próprio dugout ;
		\item tem de usar um capacete;
		\item  pode exercitar com dois \glspl{bat} oficiais de softbol (no máximo), um \gls{warm-up bat} (\gls{bat} para fazer aquecimento) aprovado, ou com combinação que não exceda o limite de dois \glspl{bat}. Um \gls{bat} com o qual o Batedor Prevenido está  exercitando não pode ter nada anexado a ele, a não ser um acessório aprovado pela WBSC-SD ou ISF;
		\item  pode deixar o círculo do Batedor Prevenido:

			 1) quando chega a sua vez de bater;
			 2) para orientar os corredores que avançam da terceira base para "home plate"; ou
			 3) para evitar uma possível Interferência numa bola \gls{fly} ou numa bola lançada.

		\item  não deve interferir na ação de um jogador da defensiva com chance de fazer uma jogada.
	\end{enumerate}
\end{enumerate}

\section*{EFEITO}
\addcontentsline{toc}{section}{EFEITOS}

	Regra 5.3 (v) -- Atrapalha um jogador da defensiva que tem oportunidade de fazer uma jogada

	 Efeito: a bola é morta e,

 1) se a Interferência é cometida quando um jogador da defensiva está tentando eliminar um corredor:
\begin{enumerate}[label=(\alph*)]
	\item   o corredor que está mais perto do \gls{homeplate} no momento da Interferência é declarado \gls{out}; e
	\item  outros corredores devem retornar à última base tocada no momento da Interferência, a menos que sejam forçados a avançar porque o batedor-corredor se tornara um corredor.
	\item se a Interferência é cometida quando um jogador da defensiva está tentando pegar uma bola \gls{fly}, ou sobre uma bola \gls{fly} que um defensor está tentando pegar:
	\begin{enumerate}[label=(\alph*)]
		\item  o batedor-corredor deve ser declarado \gls{out}, e
		\item  os corredores devem retornar à base que estavam ocupando no momento do arremesso.

	\end{enumerate}
\end{enumerate}

	 Regra 5.3 (c) (ii) -- O batedor se recusa a usar capacete quando ordenado a fazê-lo

	Efeito: Após uma advertência, o jogador deve ser expulso do jogo.

 	Regra 5.3 (e) -- O batedor usa equipamento de aquecimento ilegal

	Efeito: O equipamento de aquecimento ilegal deve ser removido do jogo.

	O jogador que continuar usando o equipamento que fora removido será  expulso do jogo.


\section{BATEDOR}
\subsection{ORDEM DE BATEDORES}

\begin{enumerate}[label=(\alph*)]
	\item   A ordem de batedores tem de ser seguida durante todo o jogo, a menos que um jogador seja substituído por outro, e nesse caso o substituto tem de ocupar lugar do jogador substituído na ordem de batedores.
	\item  O primeiro batedor em cada \gls{inning} deve ser o batedor cujo nome vem em seguida ao do último batedor que completou o turno de bater no \gls{inning} anterior.
	\item  Quando o terceiro \gls{out} num \gls{inning} ocorre antes do batedor completar o turno de bater, esse batedor tem de ser o primeiro batedor no próximo \gls{inning}. a contagem de \gls{ball} e \gls{strike} é cancelada.
	\item  Um jogador bate fora de ordem quando o batedor correto deixa de bater na sequência em que está relacionado no formulário de escalação da equipe.
\end{enumerate}

\section*{EFEITO}
\addcontentsline{toc}{section}{EFEITOS}

	\begin{description}
		\item[Regra 5.4.1 -- Batedor fora de ordem] Efeito: Isto é uma jogada de apelação que pode ser feita pelo técnico, \gls{coach} ou jogador da equipe na defensiva. A equipe na defensiva perde o seu direito de apelar sobre um batedor fora de ordem quando todos os jogadores da defensiva tiverem deixado o território \gls{fair}, a caminho do \gls{bench} ou \gls{dugout}.
	\end{description}

\begin{enumerate}[label=(\alph*)]
	\item   Quando o erro é descoberto enquanto o batedor incorreto está no \gls{battersbox}:

	 \begin{enumerate}[label=\roman*.]
	 	\item o batedor correto pode ocupar legalmente o seu lugar e assumir a contagem de \gls{ball} e \gls{strike} do batedor incorreto; e
	\item qualquer ponto anotado ou avanço nas bases enquanto o batedor incorreto estava no \gls{battersbox} é legal.
	\end{enumerate}

\item  Quando o erro é descoberto após o batedor incorreto ter completado a sua  vez de bater, e antes de ser efetuado um arremesso legal ou ilegal a outro batedor:
	 \begin{enumerate}[label=\roman*.]
	 	\item  o jogador que deveria ter batido é declarado \gls{out};
	\item  qualquer avanço ou ponto anotado em consequência da ação do batedor impróprio que se tornara um batedor-corredor será anulado. Qualquer \gls{out} feito antes de descobrir esta infração será mantida;
	\item  o próximo batedor é o jogador cujo nome vem em seguida ao do jogador declarado \gls{out} por não ter batido na sua vez. Se o próximo jogador for o batedor incorreto declarado \gls{out}, deverá bater aquele que o segue na ordem de batedores;
	\item  se o batedor impróprio for declarado \gls{out}, ele não poderá bater no mesmo \gls{inning} até que todos os outros batedores na ordem de batedores tenham  completado a sua vez de bater. Se a sua vez de bater chegar antes disso, deve bater o Batedor Prevenido;
	\item  se o batedor declarado \gls{out} nessas circunstâncias completar o terceiro \gls{out}, o batedor correto no \gls{inning} seguinte será o jogador que iria bater em seguida caso a eliminação tivesse ocorrido em uma jogada normal; e
	\item  se o terceiro \gls{out} é feito sobre um batedor-corredor ou corredor antes da descoberta da infração, ainda pode ser feita uma apelação para restabelecer   vea ordem de batedores correta.
	\end{enumerate}

	 Se o erro é descoberto depois do primeiro arremesso legal ou ilegal ao próximo Batedor:

	 \begin{enumerate}[label=\roman*.]
 		\item a situação do batedor incorreto fica legalizada;
		\item todos os pontos anotados e os avanços de corredores são legais;
		\item  o próximo batedor na ordem de batedores é aquele cujo nome vem em seguida ao do batedor incorreto;
		\item  ninguém é declarado \gls{out} por ter deixado de bater; e
		\item jogadores que tenham deixado de bater e não tenham sido declarados \gls{out}s perdem a sua oportunidade de bater, e têm de aguardar até chegar  novamente a sua vez na ordem normal de batedores.
	\end{enumerate}

	\item  Nenhum corredor será removido da base que está ocupando, para bater no seu turno correto. Ele simplesmente perde a sua vez de bater, sem qualquer  penalidade. O batedor cujo nome vem em seguida ao seu, na ordem de batedores, torna-se o batedor legal. Isso não se aplica a um batedor-corredor  que tenha sido retirado da base pelo árbitro, de acordo com a Seção (b) (ii) acima.

\end{enumerate}

\subsection{EXIGÊNCIAS PARA BATEDOR}
\begin{enumerate}[label=(\alph*)]
	\item   Um batedor tem de usar um capacete aprovado.
	\item  Um batedor tem de ocupar a sua posição no \gls{battersbox} dentro de 10 segundos depois que o árbitro declara \gls{play ball}.
	\item  Nenhum manager, técnico ou jogador pode apagar as linhas da caixa de um batedor, em qualquer momento.
	\item  O batedor tem de ter ambos os pés completamente dentro do \gls{battersbox} antes do início do arremesso. Os pés do batedor podem tocar as linhas, mas nenhuma parte de um pé pode estar fora das linhas antes ao arremesso.
	\item  Após entrar no \gls{battersbox}, o batedor tem de manter pelo menos um pé inteiramente dentro do \gls{box} entre arremessos, a menos que:

	 \begin{enumerate}[label=\roman*.]
	 	\item a bola seja batida para o território \gls{fair} ou \gls{foul};
	\item o ímpeto de um \gls{swing} ou tentativa de \gls{swing}, que inclui um \gls{slap} (movimento curto e controlado do \gls{bat}) ou \gls{swing} interrompido, leve o batedor para fora do \gls{battersbox};
	\item  seja forçado a sair do \gls{battersbox} por um arremesso;
	\item  ocorra um \gls{wild pitch} ou \gls{passedball};
	\item  haja uma tentativa de jogada em qualquer base;
	\item  seja declarado tempo. \gls{time};
	\item  o arremessador deixe o círculo do arremessador ou o receptor deixe o \gls{catcher's box}; ou
	\item  o árbitro decida que o arremesso efetuado depois de contados três \glspl{ball} é um \gls{strike}, e o batedor ache que foi um \gls{ball}.
	\end{enumerate}

\end{enumerate}
\subsection{\textit{BALLS} E \textit{STRIKES}}
 Cada bola arremessada legalmente que não é batida pelo batedor é declarada \gls{ball} ou \gls{strike} pelo árbitro de \gls{home}.

\begin{enumerate}[label=(\alph*)]\item   É declarado um \gls{ball}, e a bola permanece viva, a menos que ela se torne
 morta por qualquer outra razão:

\begin{enumerate}[label=\roman*.]
	\item  quando um batedor não gira o \gls{bat} numa bola arremessada que não entra na zona de \gls{strike}, toca o \gls{homeplate}, ou toca o solo, antes de chegar ao \gls{homeplate};
	\item quando o receptor não devolve a bola diretamente ao arremessador conforme exige a regra; ou
	\item quando o arremessador não efetua o arremesso dentro de 20 segundos.
\end{enumerate}

\item  É declarado um \gls{ball}, e a bola fica morta:

\begin{enumerate}[label=\roman*.]
	\item por cada bola arremessada ilegalmente que não é batida pelo batedor;
	\item quando o técnico prefere não aceitar o resultado da jogada depois que a  bola é batida;
	\item por cada arremesso de aquecimento excedente. ou
	\item quando o arremessador não arremessar a bola dentro de vinte (20) segundos;
 \end{enumerate}
 Quando, a qualquer momento, um membro do time defensivo apagar as linhas do \gls{battersbox} de um rebatedor, uma "bola" será atribuída no próximo rebatedor do time ofensivo programado ou no rebatedor do time ofensivo que esteja no bastão. Neste caso, um arremesso não precisa ser lançado. Quando, a qualquer momento, um membro do time ofensivo apagar as linhas da caixa de um batedor, um \gls{strike} será contado ao próximo rebatedor do time ofensivo programado ou ao rebatedor do time ofensivo que esteja no bastão. Um arremesso não precisa ser lançado.

\item  É declarado um \gls{strike}, a bola permanece viva, e os corredores podem avançar correndo o risco de serem declarado \gls{out}:

\begin{enumerate}[label=\roman*.]
	\item  quando qualquer parte de uma bola arremessada entra na zona de \gls{strike} sem tocar o solo, e o batedor não gira o \gls{bat}, (Somente Arremesso Rápido -- desde que o topo da bola esteja no ou abaixo do esterno, ou a parte inferior da bola esteja na ou acima da parte mais baixa da rótula dos joelhos);
	\item por cada bola arremessada legalmente que o batedor tenta bater e erra (seu \gls{bat} não tem contato com a bola); ou
	\item por cada \gls{foultip}.
\end{enumerate}

\item  É declarado um \gls{strike}, a bola torna-se morta, e os corredores têm de retornar a suas bases sem o risco de serem declarado \gls{out}s, mas não precisam tocar  as bases intermediárias:

	 \begin{enumerate}[label=\roman*.]
	 	\item quando uma bola arremessada atinge o batedor enquanto a bola está na zona  de \gls{strike};
	\item por cada bola arremessada que o batedor tenta bater e erra (o \gls{bat} não tem contato com a bola), e essa bola toca qualquer parte do batedor;
	\item por cada bola \gls{foul} quando o batedor tem menos de dois \glspl{strike};
	\item quando qualquer parte do corpo ou roupa do batedor é atingida pela bola batida enquanto ele está dentro \gls{battersbox}, e a contagem de arremessos é menos de dois \glspl{strike};
	\item quando o batedor não entra no \gls{battersbox} dentro de 10 segundos depois que o árbitro declara \gls{play ball} (não é necessário que o arremessador tenha efetuado um arremesso);
	\item quando um membro da equipe na ofensiva apaga intencionalmente as linhas do \gls{battersbox};

	 1) se um batedor apagar as linhas, o árbitro declarará um \gls{strike} (não é necessário que o arremessador tenha efetuado um arremesso);

	 2) quando o \gls{coach} ou um membro da equipe que não seja um jogador apaga as linhas, deve ser declarado um \gls{strike} ao próximo batedor relacionado (ou seu substituto) no formulário de escalação da equipe; e

	 3) se qualquer pessoa continuar apagando as linhas intencionalmente após a primeira infração, essa pessoa será expulsa do jogo;

	\item quando o batedor sai do \gls{battersbox} colocando ambos os pés para fora das linhas e retarda o jogo, e nenhuma das exceções é aplicada (não é necessário que o arremessador tenha efetuado um arremesso).

	\end{enumerate}

\end{enumerate}

\subsection{O BATEDOR É DECLARADO \textit{OUT}}
\begin{enumerate}[label=(\alph*)]
	\item   A bola permanece viva, e os corredores podem avançar correndo o risco de serem declarados \gls{out}s quando:

	 i. o receptor agarra um terceiro \gls{strike} (um arremesso que o batedor deixa passar sem girar o \gls{bat} ou tenta bater, ou um \gls{foultip}); ou

	 ii. com menos de dois \glspl{out} e a primeira base ocupada, é declarado o terceiro \gls{strike}.

\item  A bola é declarada morta, e os corredores têm de retornar à base que estavam ocupando no momento do arremesso, mas não precisam tocar as bases intermediárias quando o batedor:

 i. tenta bater um terceiro \gls{strike} e erra (o \gls{bat} não tem contato com a bola), e a bola toca qualquer parte do corpo do batedor; ou não tenta batê-lo, e a bola arremessada atinge o batedor enquanto o arremesso está na zona de \gls{strike};

 ii. não usa um capacete apropriado para batedor quando ordenado a fazê-lo pelo árbitro;

 iii. entra no \gls{battersbox} com um \gls{bat} adulterado ou ilegal, ou é descoberto usando um \gls{bat} adulterado ou ilegal. Nesse caso, o \gls{bat} é retirado do jogo. Se \gls{bat} tiver sido adulterado, o batedor será expulso do jogo;

 iv. tem um pé completamente fora das linhas do \gls{battersbox} e tocando o solo, ou qualquer parte de um pé está tocando o \gls{homeplate}, quando ele bate a bola, independentemente de a bola batida ser \gls{fair} ou \gls{foul};

 v. deixa o \gls{battersbox} para ganhar um impulso na corrida, mas está dentro do \gls{battersbox} quando seu \gls{bat} tem contato com a bola. Se não tiver contato com a bola arremessada, não haverá penalidade;

 vi. muda de um \gls{battersbox} para outro, passando na frente do receptor, enquanto o arremessador está recebendo a senha ou aparenta estar recebendo uma senha, com os pés em contato com o \gls{pitcher's plate}, ou a qualquer momento depois disso antes de o arremesso ser completado; ou

 vii. toca uma bola \gls{fair} com o \gls{bat} pela segunda vez em território \gls{fair}, a menos que:

	 1) ele esteja dentro do \gls{battersbox}, e o contato seja feito enquanto o \gls{bat} está em suas mãos. É declarado um \gls{foulball}; ou

	 2) ele derrube o \gls{bat} e a bola role contra esse \gls{bat} em território \gls{fair}; e,
	 na opinião do árbitro, não houve intenção alguma de interferir no curso dessa bola; a bola batida deve ser julgada se é \gls{fair} ou \gls{foul}, dependendo de onde ela parou ou foi tocada primeiro por um defensor.

	\item  A bola é declarada morta, e os corredores têm de retornar à última base que, na opinião do árbitro, estavam ocupando no momento em que o batedor cometeu a Interferência, ou seja:

	\begin{enumerate}[label=\roman*.]
	 	\item saiu do \gls{battersbox} e atrapalhou o receptor que estava pegando ou lançando a bola;
		\item estorvou intencionalmente o receptor enquanto estava dentro do \gls{battersbox};
		\item interferiu numa jogada no \gls{homeplate}; Se, no julgamento do árbitro, a ação do batedor constituir interferência intencional, o corredor que tentar pontuar também estará fora (\gls{out}). ou
		\item interferiu intencionalmente numa bola lançada enquanto estava dentro ou fora do \gls{battersbox}.
	\end{enumerate}
\end{enumerate}

\section{BATEDOR-CORREDOR}
\subsection{O BATEDOR TORNA-SE UM BATEDOR-CORREDOR}
\begin{enumerate}[label=(\alph*)]
	\item   Quando acerta uma batida \gls{fair} ou \gls{foul} legalmente. A bola é viva em batida \gls{fair} ou em \gls{foul fly} pego no ar. A bola é morta em batida \gls{ground} declarada \gls{foul}.
	\item  De acordo com a Regra do Terceiro \gls{strike}. A bola é viva.
	\item  Tem de avançar à primeira base e tocá-la:

	 \begin{enumerate}[label=\roman*.]
	 	\item quando o árbitro de \gls{home} declara o quarto \gls{ball} e a bola está viva; ou
		\item quando a equipe na defensiva resolve conceder quatro \glspl{ball} intencionalmente ao batedor. A comunicação dessa intenção ao árbitro de
	 \gls{home} pode ser feita pelo arremessador, receptor ou \gls{coach} principal. A bola torna-se morta.
	\end{enumerate}

	 1) A comunicação ao árbitro deve ser entendida como quatro \glspl{ball} efetuados pelo arremessador. A comunicação pode ser feita a qualquer
	 momento antes do batedor iniciar e completar a sua vez de bater, independentemente da contagem de arremessos.
	 2) Se tal concessão for feita a dois batedores, o segundo batedor não poderá ser autorizado a ir à primeira base enquanto o primeiro batedor
	 não a tiver alcançado. Se o árbitro, equivocadamente, permite que dois batedores "andem" ao mesmo tempo, e o primeiro batedor deixa de tocar
	 a primeira base, nenhuma apelação por omissão de base deve ser aceita sobre o primeiro batedor.
	 3) A bola torna-se morta e os corredores não podem avançar, a menos  que sejam forçados.

	\item  Quando o receptor ou qualquer outro jogador da defensiva obstrui/estorva o batedor, ou impede que ele gire o \gls{bat} ou bata uma bola arremessada.
	\item  Quando uma bola \gls{fair} atinge o corpo, equipamento que está usando, ou a roupa do árbitro, ou um corredor.
	\item  Quando é atingido por um arremesso ( \gls{hitbypitch}). As mãos do batedor não são consideradas uma parte do \gls{bat}. A bola torna-se morta e o batedor adquire o direito de ir à primeira base, sem o risco de ser declarado \gls{out}. Se o batedor não tentar evitar ser atingido pela bola arremessada, o árbitro declarará um \gls{ball} e não concederá uma base. A bola ficará fora de jogo (bola morta).
	\item   É declarado um \gls{homerun} quando uma bola batida \gls{fly} que está em território \gls{fair}:

	 \begin{enumerate}[label=\roman*.]
	 	\item passa sobre a cerca em território \gls{fair};
		\item bate na luva ou corpo do defensor e passa diretamente sobre a cerca em território \gls{fair}, ou toca o topo da cerca em território \gls{fair} e passa sobre essa cerca;
		\item  toca o poste de \gls{foul}, acima do nível da cerca; ou
		\item  é tocada por um defensor que está em área de bola morta, e essa bola, na opinião do árbitro, teria passado sobre a cerca em território \gls{fair}.
	\end{enumerate}
	 Não é um \gls{homerun} se uma bola batida \gls{fly} que está em território \gls{fair}:

		 1) passa sobre a cerca a uma distância menor do que aquela prescrita na Regra 2, Anexo 1 (A) ( Dimensões Oficiais do Campo) e Anexo 1 (F) (Tabela de Referência Rápida) -- essa distância deve ser marcada para orientação do árbitro;
		 2) bate na luva ou no corpo do defensor e passa sobre a cerca em território \gls{foul};
		 3) toca primeiro a cerca, desvia após ter contato com um defensor e depois passa sobre a cerca; ou
		 4) é tocada por um defensor que está em área de bola morta, e essa bola, na opinião do árbitro, não teria passado sobre a cerca em território \gls{fair}.

	\item   Quando qualquer pessoa, exceto um membro da equipe, entra no campo de jogo e interfere:

		 \begin{enumerate}[label=\roman*.]
		 	\item  numa bola batida \gls{ground} que está em território \gls{fair};
		\item  na ação de um defensor que está prestes a efetuar uma defesa ou pegar uma bola lançada;
		\item na ação de um defensor que está prestes a lançar uma bola; ou
		\item numa bola lançada por um defensor.
		\end{enumerate}
\end{enumerate}

\section*{EFEITOS}
\addcontentsline{toc}{section}{EFEITOS}

 Regra 5.5.1 (d) -- Um jogador da defensiva impede que o batedor gire o \gls{bat} ou bata uma bola arremessada

 Efeito:

	 1) O árbitro deve fazer o gesto de Bola Morta Demorada (\gls{delayeddeadball}).
	 A bola permanece viva até a jogada ser concluída.
	 2) O técnico da equipe na ofensiva tem a opção de aceitar a concessão pela falta cometida pelo receptor (Obstrução) ou aceitar o resultado da jogada.
	 3) Se o batedor bater a bola e chegar a salvo (\gls{safe}) à primeira base, e se todos os outros corredores tiverem avançado pelo menos uma base em consequência da bola batida, a Obstrução será cancelada. Uma vez que um corredor passa uma base, mesmo sem pisá-la, ele é considerado como se tivesse alcançado essa base. Toda ação resultante da bola batida será mantida. Nenhuma opção será dada.
	 4) Se o técnico não aceitar o resultado da jogada, será aplicada a regra de "Obstrução do Receptor", e nesse caso será concedida a primeira base ao batedor; os outros corredores serão autorizados a avançar somente se forem forçados.

 Regra 5.5.1 (e) -- Uma bola \gls{fair} atinge o corpo, o equipamento que está usando ou a roupa do árbitro ou de um corredor

 Efeito:

	 1) Depois de tocar um defensor (incluindo o arremessador). A bola permanece em jogo.
	 2) Depois de passar um defensor, exceto o arremessador, e nenhum outro defensor tinha chance de eliminar um corredor. A bola permanece em jogo.
	 3) Antes de passar um defensor, exceto o arremessador, sem ter sido tocada. A bola torna-se morta.

\subsection{O BATEDOR-CORREDOR É DECLARADO \textit{OUT}}
	\begin{enumerate}[label=(\alph*)]
		\item   A bola permanece viva e um corredor pode avançar a seu próprio risco quando:
		\begin{enumerate}[label=\roman*.]
			\item o receptor derruba o terceiro \gls{strike} e o batedor-corredor é tocado legalmente com a bola enquanto está fora da base, ou pisa a primeira base depois que a  bola lançada a essa base é pega legalmente por um defensor;
			\item um defensor pega legalmente uma bola batida \gls{fly} antes que ela toque o solo, algum objeto ou uma pessoa, exceto um jogador da defensiva;
			\item após acertar uma batida \gls{fair}, um corredor é tocado enquanto está fora da base, ou um batedor-corredor é declarado \gls{out} pela bola lançada antes de chegar à primeira base;
			\item em vez de ir à primeira base, entra na área de sua equipe

			 1) após acertar uma batida \gls{fair};
			 2) após ter obtido uma base por \glspl{ball} (\gls{baseonballs}); ou
			 3) quando tem que avançar legalmente à primeira base.

			\item  é declarado um \gls{infieldfly};
			\item  após acertar uma batida \gls{fair}, toca somente a parte da base dupla que está em território \gls{fair}, na sua primeira tentativa de alcançar a primeira base, e ocorre uma jogada nessa base. Isso é tratado da mesma forma que uma omissão de base. Se a equipe na defensiva apelar, o batedor-corredor será declarado \gls{out} (Jogada de Apelação). A equipe na defensiva perde a oportunidade de apelar se, depois que o batedor-corredor ultrapassa a base, não se manifesta sobre a omissão de base antes que ele retorne à porção \gls{fair} da primeira base;
			\item  desvia mais de um (1) metro [três (3) pés] do caminho da base para evitar ser tocado com a bola na(s) mão(s) de um defensor; ou
			\item  quando qualquer pessoa, exceto outro corredor, presta ajuda física a um corredor numa bola \gls{fly} (o batedor-corredor é declarado \gls{out} se essa bola é pega).
		\end{enumerate}

		\item  A bola é declarada morta, o corredor tem de retornar à última base tocada legalmente no momento do arremesso, mas não precisa tocar as bases intermediárias quando o batedor-corredor:
		\begin{enumerate}[label=\roman*.]
	\item  não usa um capacete aprovado quando ordenado a fazê-lo pelo árbitro;
	\item  corre fora da faixa de um (1) metro [três (3) pés] e, na opinião do árbitro,

 1) estorva o defensor que está pegando a bola lançada à primeira base;
 ou

 2) interfere numa bola lançada e impede que um defensor execute uma  jogada na primeira base. Uma bola lançada que atinge um batedor- corredor não caracteriza, necessariamente, uma Interferência.

	\item  interfere na jogada de um defensor que está tentando pegar uma bola batida. Um batedor-corredor pode correr fora da faixa de um metro para se esquivar de um defensor que está tentando pegar a bola batida;
	\item  interfere na jogada de um defensor que está tentando lançar uma bola;
	\item  interfere intencionalmente numa bola lançada;
	\item  interfere numa bola batida \gls{fair} (fora do \gls{battersbox}) antes de chegar à primeira base;
	\item  interfere na jogada de um terceiro \gls{strike} não agarrado;
	\item  após bater a bola, atira seu \gls{bat} de tal forma que pode interferir na jogada de um defensor que tem chance de fazer um \gls{out};
	\item  o Batedor Prevenido estorva um jogador da defensiva que está tentando pegar uma bola \gls{fly}, ou interfere numa bola \gls{fly} que um defensor está tentando pegar.
	\item  um membro da equipe na ofensiva (ataque), que não seja um batedor, batedor-corredor, corredor ou Batedor Prevenido, interfere na ação de um defensor que está tentando pegar uma bola \gls{foul fly} (bola \gls{fly} que está em território \gls{foul}), ou num \gls{foul fly} que um defensor está tentando pegar. Se, na  opinião do árbitro, a Interferência é cometida com clara intenção de evitar uma  jogada dupla, o corredor que está mais perto do \gls{homeplate} no momento em que a falta é cometida deve também ser declarado \gls{out}.
	\item  interfere intencionalmente numa jogada no \gls{homeplate} para evitar um \gls{out} evidente nessa base. Se, na opinião do árbitro, a Interferência foi cometida com propósito de atrapalhar uma jogada no \gls{homeplate}, o corredor também é declarado \gls{out};
	\item   dá um passo para trás na direção do \gls{homeplate} para evitar ou retardar um toque de um defensor;
	\item  numa situação de \gls{jogadaforcada}, toca somente a porção \gls{fair} da base dupla e colide com um defensor que, usando também a porção \gls{fair} da base, está prestes a pegar uma bola lançada;
	\item  Com menos de dois \glspl{out} e um corredor na primeira base, um defensor derruba, intencionalmente, uma bola \gls{fair fly} (incluindo um \gls{line drive} ou um \gls{fly} resultante de \gls{bunt} ), que poderia ser pega por um defensor do campo interno com um esforço normal, depois de tê-la controlado com a mão ou luva.
	\item   Um \gls{bunt}  executado depois do segundo \gls{strike} resulta em \gls{foulball}, exceto se um corredor atrapalha um defensor que está tentando pegar uma bola \gls{fly} resultante de \gls{bunt}  em território \gls{foul}, ou interfere numa bola \gls{foul fly} que um  defensor está tentando pegar; nesse caso, o batedor-corredor volta a bater, com um \gls{strike} adicional pela bola \gls{foul} se a contagem de bolas era menos de dois  \glspl{strike} quando ele bateu a bola. Se um \gls{fly} resultante de \gls{bunt}  for pego, a bola  continuará viva e em jogo.
\end{enumerate}

		\item  Um corredor tem de retornar à última base que, na opinião do árbitro, foi tocada no momento da Interferência, e a bola torna-se morta quando:

		\begin{enumerate}[label=\roman*.]
			\item na opinião do árbitro, o corredor precedente (aquele que está imediatamente à frente) que ainda não está declarado \gls{out} atrapalha  intencionalmente um defensor que está tentando

				 1) pegar uma bola lançada; ou
				 2) lançar a bola para tentar completar a jogada.

			\item  uma pessoa, exceto um membro da equipe, entra no campo de jogo e interfere

				1) na ação de um defensor que está prestes a pegar uma bola \gls{fly}; ou

				2) numa bola \gls{fly} que, na opinião do árbitro, um jogador da defensiva é capaz de pegar.
		\end{enumerate}
	\end{enumerate}
\section*{EFEITOS}
\addcontentsline{toc}{section}{EFEITOS}

 Regra 5.5.2 (a) (v) -- É declarado um \gls{infieldfly}
 Efeito:

 A bola está viva e um corredor pode avançar correndo o risco da bola ser pega, ou retocar a base e avançar após a bola ser tocada, como em qualquer bola \gls{fly}. Se um \gls{infieldfly} declarado se torna um \gls{foulball}, ele é tratado da mesma forma que qualquer \gls{foulball}.

 Se num \gls{infieldfly} declarado a bola cai ao chão sem ter contato com um defensor e salta para o território \gls{foul}, antes de passar a primeira ou terceira base, é um \gls{foulball}.

 Se num \gls{infieldfly} declarado a bola cai ao chão sem ter contato com um defensor, fora da linha de base, e salta para o território \gls{fair}, antes de passar a primeira ou terceira base, é um \gls{infieldfly}.

 Regra 5.5.2 (b) (ii) a (xi) -- Batedor-corredor comete Interferência
 Efeito:

 EXCEÇÃO: Se ocorrer uma jogada sobre um corredor antes da Interferência e

 1) ele for declarado \gls{out}, o resultado dessa jogada será mantido; ou
 2) ele não for declarado \gls{out}, o resultado dessa jogada será mantido, a menos que a Interferência cometida pelo batedor-corredor ocasione o terceiro \gls{out}. Outros corredores sobre os quais não tenha havido jogada têm de retornar à base tocada legalmente no momento do arremesso.

 Regra 5.5.2 (c) (ii) - Corredor precedente (aquele que está imediatamente à frente) comete Interferência
 Efeito: A bola torna-se morta e o corredor também é declarado \gls{out}.

\section{BASE DUPLA}
 Devem ser aplicadas as seguintes regras quando for usada a base dupla:
\begin{enumerate}[label=(\alph*)]
	\item O batedor-corredor está sujeito ao seguinte.

	\begin{enumerate}[label=\roman*.]
		\item Uma bola batida que atinge a porção \gls{fair} é declarada \gls{fair}, e aquela que atinge somente a porção \gls{foul} é declarada \gls{foul}.

		\item Um jogador da defensiva tem de usar somente a porção \gls{fair} da base dupla, exceto em jogada com bola viva feita do território \gls{foul} do lado da primeira base;
	  nesse caso, o batedor-corredor e o jogador da defensiva podem usar tanto a  porção \gls{foul} como a porção \gls{fair} da base dupla. Quando o jogador da defensiva  usa a porção \gls{foul} da base dupla, o batedor-corredor pode correr em território

	\gls{fair}, e se for atingido por um lançamento feito do território \gls{foul} do lado da primeira base, isso não caracteriza uma Interferência. Se for apontada uma Interferência intencional, o batedor-corredor deve ser declarado \gls{out}. A faixa  de um metro (três pés) duplica-se em lançamentos feitos do território \gls{foul} do lado da primeira base.

		\item Numa jogada na primeira base em que o batedor-corredor que tenta alcançar a primeira base através de uma batida, ou de um terceiro \gls{strike} não agarrado, toca somente a porção \gls{fair} da base dupla, se a equipe na defensiva apelar antes do batedor-corredor retornar à porção \gls{fair}, ele será declarado \gls{out}. Isso é tratado da mesma forma que uma omissão de base; a equipe na defensiva  pode apelar.

		\item  Depois de ultrapassar a base correndo, o batedor-corredor tem de retornar à porção \gls{fair}.

		\item Quando, em uma bola batida para o campo externo não está havendo jogada na base dupla, o batedor-corredor pode tocar tanto a porção \gls{foul} como a porção \gls{fair} da base.
	\end{enumerate}

	\item  Devem ser aplicadas as seguintes regras a um corredor.

	\begin{enumerate}[label=\roman*.]
	\item Depois de ultrapassar a base correndo, tem de retornar à porção \gls{fair}.
	\item Quando vai retocar a base numa bola \gls{fly}, tem de usar a porção \gls{fair}.
	\item Numa jogada em que se tenta surpreender o corredor fora da base, esse corredor tem de retornar à porção \gls{fair}.
 	\item Se o corredor ficar sobre a porção \gls{foul} somente após retornar à porção \gls{fair}, ele será considerado como se não estivesse em contato com a base,  e poderá ser declarado \gls{out} se:

 1) for tocado com a bola; ou

 2) ficar sobre a porção \gls{foul} da base enquanto o arremessador está de posse da bola dentro do círculo do arremessador.
\end{enumerate}
\end{enumerate}

 \section{USO DE LUVA ILEGAL}

 Quando um defensor faz uma jogada sobre um batedor-corredor ou corredor enquanto está usando uma luva ilegal, o técnico da equipe prejudicada tem a  opção de:
	\begin{enumerate}[label=(\alph*)]\item   aceitar o resultado da jogada;
		\item  no caso de um batedor-corredor, mandar o jogador bater novamente, assumindo a contagem de \gls{ball} e \gls{strike} anterior ao arremesso; os outros corredores têm de retornar às bases que estavam ocupando legalmente no momento do arremesso; ou
		\item  no caso de um corredor, ter toda a jogada anulada; nesse caso, os corredores têm de retornar às bases que estavam ocupando legalmente no momento da  jogada. Se a jogada for o resultado da ação de um batedor que termina a sua  vez de bater, esse jogador deverá bater novamente, assumindo a contagem de \gls{ball} e \gls{strike} que tinha antes de completar o seu turno, e os corredores têm de retornar às bases que estavam ocupando no momento do arremesso. Um arremesso efetuado pelo arremessador não é considerado uma jogada.
	\end{enumerate}

\section{REMOÇÃO DO CAPACETE}
\begin{enumerate}[label=(\alph*)]
	\item   Quando a bola está viva, um batedor, batedor-corredor ou corredor será declarado \gls{out} quando usar um capacete incorretamente, de propósito, ou o remover deliberadamente, durante uma jogada, exceto num \gls{homerun} para fora do campo. A declaração de \gls{out} de um batedor-corredor ou corredor por ter removido o capacete, deliberadamente, não cancela qualquer situação de  \gls{jogadaforcada}, entretanto, se um capacete usado por um batedor, batedor-corredor ou corredor se soltar acidentalmente de seu lugar apropriado, não haverá penalidade.
\item  Quando a bola está morta, um corredor tem de retornar à última base tocada no momento do contato:

 \begin{enumerate}[label=\roman*.]
 	\item  se uma bola lançada ou batida tem contato com o capacete removido propositalmente, ou um defensor tem contato com o capacete removido propositalmente, enquanto está tentando fazer uma jogada; ou

	\item quando uma bola batida ou lançada tem contato com o capacete que  se soltara acidentalmente, e esse contato interfere na jogada que está sendo executada; ou, quando um jogador da defensiva tem contato com um capacete que está sobre o solo, e esse contato impede que ele faça  uma jogada, o batedor-corredor ou corredor que estava usando esse  capacete é declarado \gls{out}, mesmo que ele tenha anotado ponto. o ponto é anulado.
 \end{enumerate}
\end{enumerate}

\section{TOCAR AS BASES EM ORDEM LEGAL}
\begin{enumerate}[label=(\alph*)]
	\item   O batedor-corredor e todos outros corredores têm de tocar as bases em ordem legal (isto é, primeira, segunda e terceira base e \gls{homeplate}).

 EXCEÇÃO: Se um corredor for obstruído numa base e não conseguir tocar essa base, ou for colocado na segunda base de acordo com a Regra de Desempate (\gls{tie-breaker rule}).

	\item  Um corredor que está retornando a uma base enquanto a bola está viva, correndo o risco de ser declarado \gls{out}, tem de retornar:

	 \begin{enumerate}[label=\roman*.]
	 	\item  à base que deixara antes de uma bola \gls{fly} ser pega no ar ou tocada por um defensor; ou
		\item à base que omitira; e tem de tocar as bases em ordem inversa.
	\end{enumerate}

	\item  Quando um corredor está retornando a uma base enquanto a bola está morta, ele não precisa tocar as bases intermediárias, a menos que tenha omitido uma base; nesse caso, a equipe na defensiva pode apelar se ele não retocar a base omitida.
	\item  Quando um corredor ou batedor-corredor adquire o direito a uma base, tocando-a antes de ser declarado \gls{out}, ele tem o direito de ocupar essa base  até que tenha tocado legalmente a base seguinte, em ordem, ou até que seja  forçado a desocupá-la para um corredor subsequente. A bola está em jogo e os  corredores podem avançar, correndo o risco de serem declarado \gls{out}s.
	\item  Quando um corredor desloca uma base de sua posição correta, nem ele, nem  o(s) corredor(es) que o segue(m) na mesma série de jogadas são obrigados a  acompanhar uma base que está exageradamente fora de posição. A bola está em jogo e os corredores podem avançar ou retornar, correndo o risco de serem declarado \gls{out}s.
	\item  Dois corredores não podem ocupar a mesma base simultaneamente. o corredor que ocupava legalmente a base primeiro é autorizado a nela permanecer, a menos que seja forçado a avançar. O outro corredor poderá ser declarado \gls{out} se for tocado com a bola.
	\item   O fato de um corredor precedente ser declarado \gls{out} por ter omitido uma base ou por ter deixado uma base antecipadamente numa bola \gls{fly} pega no ar não afeta a situação de um corredor subsequente que toca as bases em ordem correta. Se a falta cometida pelo corredor -- não tocar uma base em ordem normal ou deixar uma base antecipadamente numa bola \gls{fly} pega no ar -- causar o terceiro \gls{out} do \gls{inning}, nenhum corredor subsequente poderá
	 anotar ponto.
	\item   Nenhum corredor poderá retornar para tocar uma base que omitira ou deixara ilegalmente, depois que um corredor subsequente tiver anotado ponto, ou depois  que ele tiver deixado o campo de jogo.

	 i) As bases deixadas antecipadamente numa bola \gls{fly} pega no ar têm de ser retocadas antes de avançar às bases concedidas.

	\item   As bases concedidas têm de ser tocadas em ordem legal.
\end{enumerate}

\section*{EFEITO}
\addcontentsline{toc}{section}{EFEITOS}

Regra 5.9 (*** a (i) -- Tocar as bases
 Efeito:
 corredor será declarado \gls{out} se a defensiva apelar legalmente por ele ter omitido uma base ou deixado uma base antes da bola ser tocada numa bola \gls{fly} pega no ar.

\section{CORREDORES}

\subsection{CORREDORES PODEM AVANÇAR CORRENDO O RISCO DE SEREM DECLARADO \gls{out} ENQUANTO A BOLA ESTÁ VIVA}
\begin{enumerate}[label=(\alph*)]\item   Quando a bola deixa a mão do arremessador em seu arremesso.
	\item  Numa bola lançada ou numa bola batida \gls{fair} que não é bloqueada.
	\item  Numa bola lançada que atinge um árbitro ou um jogador da ofensiva.
	\item  Quando uma bola \gls{fly} pega legalmente tem o primeiro contato com um defensor.
	\item  Quando uma bola batida \gls{fair}:

	 \begin{enumerate}[label=\roman*.]
	 	\item  atinge um árbitro ou corredor depois de passar um defensor, exceto o arremessador, e desde que nenhum outro defensor tenha uma chance de	 fazer um \gls{out};
		\item é tocada por um defensor, incluindo o arremessador; ou
	 	\item atinge um fotógrafo, encarregado da manutenção do campo, policial etc. designados para o jogo; a bola permanece viva.
	\end{enumerate}

	\item  Quando uma bola viva fica alojada no uniforme ou equipamento de um jogador da defensiva.
	\item   Quando, a qualquer momento, deixa de tocar uma base para a qual está autorizado a avançar, antes de tentar seguir para a base seguinte.
	\item   Quando, após ultrapassar a primeira base correndo, faz uma tentativa de ir para a segunda base.

	 i) Quando, após deslocar uma base, tenta ir para a base seguinte.

	\item   Quando, um arremesso ilegal não batido -- e que é também um \gls{wild pitch} ou	 \gls{passedball} -- tenta avançar além de uma base que lhe é concedida (pelo arremesso ilegal).

	\item   Quando avança além da base que lhe é concedida numa jogada em que:

	\begin{enumerate}[label=\roman*.]
	 	\item um defensor toca, intencionalmente, uma bola lançada, com um equipamento removido do lugar onde normalmente é usado; ou
		\item um defensor toca, intencionalmente, uma bola batida \gls{fair}, com um  equipamento removido do lugar onde normalmente é usado.
	\end{enumerate}

	\item Quando avança além da base até a qual está protegido ou que lhe é concedida por ter sido obstruído.
	\item Quando avança além da base para a qual é forçado a ir por causa de uma base por \glspl{ball} concedida ao batedor.

\end{enumerate}

\section*{EFEITO}
\addcontentsline{toc}{section}{EFEITOS}

Regra 5.10.1 (h \& i) -- Não toca uma base ou continua avançando à base seguinte
 Efeito: corredor será declarado \gls{out} se a defensiva apelar legalmente.

\subsection{BASES CONCEDIDAS A CORREDOR(ES) POR OBSTRUÇÃO}

Quando ocorre uma Obstrução, inclusive num \gls{run-down play}:

\begin{enumerate}[label=(\alph*)]\item   deve ser sinalizada uma Bola Morta Demorada; a bola permanece viva até que a jogada seja concluída;
	\item  ao corredor obstruído, e a cada um dos corredores afetados pela Obstrução, será concedida a base ou bases que, na opinião do árbitro, estes teriam alcançado se não tivesse ocorrido a obstrução. Se o árbitro achar que há justificativa, um jogador da defensiva que faz uma simulação de toque (\gls{fake tag}) poderá ser expulso do jogo;

	\item  se o corredor obstruído for declarado \gls{out} antes de chegar à base que teria alcançado se não tivesse corrido a Obstrução, será declarada uma bola morta.

	Ao corredor obstruído, e a cada corredor afetado pela Obstrução, será concedida a base ou bases que, na opinião do árbitro, estes teriam alcançado se a obstrução não tivesse acontecido; e
	\item  um corredor obstruído nunca pode ser declarado \gls{out} no espaço -- entre duas bases -- onde ocorreu a Obstrução, a menos que:

\begin{enumerate}[label=\roman*.]
	\item ele cometa um ato de Interferência depois que a Obstrução é apontada, ou sofra uma apelação legal por:

	 \begin{enumerate}[label=\arabic*)]
	 	\item ter omitido uma base, desde que não tenha sido obstruído nessa base e impedido de tocá-la;
	 	\item ter deixado uma base antes de uma bola \gls{fly} ter o primeiro contato com um defensor; ou
		\item ter passado a base que teria alcançado se não tivesse ocorrido a  Obstrução; o corredor obstruído pode ser declarado \gls{out} e a bola permanece viva.
	\end{enumerate}
	\item se o corredor obstruído obtém com segurança a base que lhe teria sido concedida, na opinião do árbitro, e há uma jogada subsequente sobre outro corredor, ele não estará mais protegido no espaço -- entre as bases -- onde fora  obstruído, e poderá ser declarado \gls{out}.
 \end{enumerate}

 A bola permanece viva.

 Os corredores  obstruídos são ainda obrigados a tocar todas as bases em ordem correta; e se  não o fizerem, poderão ser declarados declarado \gls{out}s em uma apelação correta da equipe na defensiva. EXCEÇÃO: Se um corredor for obstruído numa base e não conseguir tocar essa base.

\end{enumerate}

\subsection{CORREDORES SÃO DECLARADO \gls{out}}

\begin{enumerate}[label=(\alph*)]
	\item   Um corredor é declarado \gls{out} e a bola permanece viva quando:

\begin{enumerate}[label=\roman*.]
	\item enquanto corre para qualquer base em ordem normal ou inversa, desvia mais de um (1) metro [três (3) pés] do caminho da base, para evitar ser tocado;
\item  enquanto a bola está em jogo, e ele não está em contato com uma base, é tocado por um defensor;
\item  numa \gls{jogadaforcada}, e antes do corredor ter contato com a base para a qual é forçado a avançar, um defensor, enquanto controla a bola em sua(s) mão(s) toca a base, ou toca a bola na base, ou toca o corredor. Se um corredor obrigado deixar a base, após tocar a base seguinte retorna, por alguma razão, à base que estava ocupando, a situação de \gls{jogadaforcada} é restabelecida;
\item  não retorna para tocar a base que estava ocupando anteriormente, ou que omitira, e a equipe na defensiva apela legalmente;
\item qualquer pessoa, exceto outro corredor, presta-lhe ajuda física enquanto a bola está em jogo. Quando a bola se torna morta após um
 \gls{homerun}, uma bola \gls{foul} não pega ou uma concessão de bases, a bola permanece morta.
\item  ultrapassa fisicamente um corredor precedente, antes que esse corredor tenha sido declarado \gls{out}. A bola permanece viva. O corredor  não será declarado \gls{out} se a bola se tornar um \gls{foulball} ou um \gls{foul fly} não pego no ar ou se um corredor ultrapassar um corredor precedente numa jogada em que a bola fica morta. A bola continuará morta.
\item  deixa a sua base para avançar a outra base antes que uma bola \gls{fly} pega no ar tenha tocado um defensor;
\item deixa de tocar a(s) base(s) intermediária(s) em ordem normal ou inversa, a menos que seja obstruído no momento em que esteja tentando tocá-la(s);
\item  o batedor-corredor que se tornara um corredor, tocando a primeira base, tenta correr para segunda base, após ultrapassar a primeira base,
 e é tocado enquanto está fora da base;
\item  após passar correndo ou deslizando pelo \gls{homeplate}, sem pisá-lo, não tenta retornar para reparar a falha, e um defensor, com a bola na(s) mão(s) e enquanto mantém contato com o \gls{plate}, apela ao árbitro por uma decisão;
\item  abandona uma base e entra na área de sua equipe ou deixa o campo de jogo, enquanto a bola está viva;
\item  em qualquer bola \gls{fly}, se posiciona atrás da base, e não fica em contato com ela, para iniciar a corrida à base seguinte tomando impulso a partir dessa posição, ou seja, para fazer um \gls{running start}; ou
\item  quando corredores mudam de posições nas bases.
\end{enumerate}
	\item  Um corredor é declarado \gls{out} e a bola torna-se morta quando:

	 \begin{enumerate}[label=\roman*.]
	 	\item não obedece à ordem do árbitro exigindo o uso de um capacete aprovado para batedores;
		\item não fica em contato com a base para a qual foi autorizado a ir até que a bola arremessada legalmente deixe a mão do arremessador.
		 É declarado um "Arremesso Nulo" e os outros corredores têm de retornar à última base que estavam ocupando no momento do arremesso; ou
		\item  está afastado de sua base legalmente após um arremesso, ou em razão de o batedor ter completado a sua vez de bater, e enquanto o arremessador, com a bola na mão, está dentro do Círculo do Arremessador, não retorna imediatamente à sua base, ou não tenta avançar à base seguinte.
		\item  Uma vez que o corredor retorna a uma base por qualquer razão, ele
		 será declarado \gls{out} se deixar essa base. Um corredor não será
		 declarado \gls{out} se:

			  \begin{enumerate}[label=\arabic*)]
			 	\item for feita uma jogada sobre ele ou outro corredor (uma simulação de toque é considerada uma jogada);
				\item o arremessador não está mais dentro do Círculo do Arremessador, com a bola na mão; ou
			 	\item o arremessador efetua um arremesso ao batedor.
			\end{enumerate}
		\item Uma base por \glspl{ball} ou um terceiro \gls{strike} não agarrado em que o corredor é autorizado a correr é tratado da mesma forma que uma bola batida. O batedor-corredor pode continuar avançando após ultrapassar a primeira base, e é autorizado a correr em direção à segunda base, desde  que ele não pare na primeira base. Se ele parar após ultrapassar a primeira base fazendo uma curva, terá de retornar à base, ou continuar  avançando à segunda base, imediatamente; e
		\item  o batedor-corredor é declarado \gls{out} por ter interferido numa jogada no \gls{homeplate} para tentar evitar uma eliminação evidente de um corredor  que esteja avançando para \gls{home}. O corredor que está avançando é declarado \gls{out} e os outros corredores têm de retornar à última base que  estavam ocupando no momento do arremesso.
	\end{enumerate}

	\item  Um corredor é declarado \gls{out}, a bola torna-se morta, e os outros corredores
	 têm de retornar à última base que estavam ocupando legalmente no momento em que ocorreu a Interferência; a bola ficou bloqueada ("\gls{blocked ball}); ou um \gls{out} foi declarado (a menos que tenham sido forçados a avançar porque o batedor se tornou um batedor-corredor), quando:

\begin{enumerate}[label=\roman*.]
	\item é atingido por uma bola batida \gls{fair} não tocada, em território \gls{fair}, enquanto está fora da base, e na opinião do árbitro algum defensor tinha oportunidade de fazer uma jogada para fazer um \gls{out};
	\item chuta, intencionalmente, uma bola que um defensor não tenha conseguido defender;
	\item interfere na ação de um defensor que está tentando pegar uma bola batida \gls{fair}, sem levar em consideração se ela foi tocada antes pelo defensor ou por outro defensor, incluindo o arremessador, ou estorva um defensor que está efetuando um lançamento, ou interfere, intencionalmente, numa bola lançada;
	\item interfere na ação de um defensor que está tentando pegar uma bola batida \gls{foul fly}, ou num \gls{foul fly} que um defensor está tentando pegar.

	 Se essa Interferência, na opinião do árbitro, é uma evidente tentativa de evitar um \gls{doubleplay} (jogada dupla), o corredor subsequente imediato deve também ser declarado \gls{out}. O batedor-corredor volta a bater, com um \gls{strike} adicional pela bola \gls{foul}, se a contagem de arremessos anterior à batida era menos de dois \glspl{strike}. Se essa Interferência causar  terceiro \gls{out}, o batedor-corredor voltará a bater como o primeiro batedor no próximo \gls{inning}, com a contagem original de \gls{ball} e \gls{strike} cancelada;

	\item depois de um corredor, batedor ou batedor-corredor ter sido declarado \gls{out}, ou após um corredor ter anotado ponto, tal corredor, batedor ou batedor-corredor interfere na ação de um defensor que tem oportunidade de fazer uma jogada sobre outro corredor. O corredor que está mais perto do \gls{homeplate} no momento da Interferência deve ser declarado \gls{out}.

	 Se um corredor continuar correndo após ter sido declarado \gls{out} e atrair um lançamento, tal ato será tratado como uma forma de Interferência;

	\item um ou mais membros da equipe na ofensiva param em, ou se reúnem ao redor de, uma base para a qual um corredor está avançando, e assim confundem os defensores e contribuem para dificultar a execução de uma  jogada. É considerado membro de uma equipe o \gls{bat boy} (pessoa encarregada de recolher os \glspl{bat} ao \gls{bench}) ou qualquer outra pessoa autorizada a permanecer no \gls{bench} da equipe;

	\item o \gls{coach} da terceira base corre na direção do \gls{homeplate}, sobre a/perto da linha de base, enquanto um defensor está tentando fazer uma jogada sobre uma bola batida ou lançada e, assim, atrai um lançamento ao \gls{homeplate}. O corredor que está mais perto do \gls{homeplate} deve ser  declarado \gls{out};

	\item um \gls{coach}, ou qualquer membro da equipe na ofensiva que não seja um batedor, batedor-corredor, Batedor Prevenido ou corredor, interfere,  intencionalmente, numa bola lançada enquanto está no \gls{coachsbox}, ou estorva a equipe na defensiva, que tem oportunidade de fazer uma jogada sobre um corredor ou batedor-corredor. O corredor que está mais perto do \gls{homeplate} no momento da Interferência deve ser declarado \gls{out};
	\item movimentando-se em pé, colide, intencionalmente, com um jogador da defensiva que, com a bola na(s) mão(s), está preparado para tocá-lo. Se  a má intenção for flagrante, o infrator será expulso;

	\item corre as bases em ordem inversa, ou fica afastado da linha de base enquanto não está tentando avançar, para confundir os
	 defensores ou ridicularizar o jogo;
	\item o Batedor Prevenido interfere na ação de um jogador da defensiva que
	 está tentando eliminar um corredor; o corredor que está mais perto do \gls{homeplate} deve ser declarado \gls{out}; ou
	\item um equipamento não oficial da equipe na ofensiva causa um \gls{blocked ball} (bola bloqueada) e provoca uma Interferência no momento em que está ocorrendo uma jogada sobre o corredor. Se esse corredor tiver anotado ponto antes de ser declarado um \gls{blocked ball}, o corredor que está mais perto do \gls{homeplate} deve ser declarado \gls{out}.
	\end{enumerate}
	\item  Quando o árbitro de \gls{home} -- ou a sua roupa -- interfere na ação do receptor que está tentando eliminar um corredor num \gls{stealing} (roubo de base),  ou numa tentativa de \gls{pickoff play} (jogada em que o arremessador tenta segurar  corredor na base, ou eliminar o corredor que está fora da base) ou ainda num \gls{passedball} (bola defensável que passa para trás do receptor) ou \gls{wild pitch}  (arremesso descontrolado), a bola lançada pelo receptor atinge o árbitro, não é  uma Interferência do Árbitro, e a bola permanece viva.
\end{enumerate}

\section*{EFEITOS}
\addcontentsline{toc}{section}{EFEITOS}

\begin{description}
	\item[Regra 5.10.3 (a) (vii a x) -- Deixa a base antecipadamente numa bola \gls{fly}, omite uma base ou tenta chegar à segunda base ou omite o \gls{homeplate}]  	Efeito: corredor não será declarado \gls{out}, a menos que a equipe na defensiva faça uma apelação legal.

	EXCEÇÃO: Um corredor que tenha deixado uma base antecipadamente numa bola \gls{fly} pega no ar, ou tenha omitido uma base, pode tentar retornar a essa base enquanto a bola está morta.

	\item [Regra 5.10.3 (a) (xiii) -- Corredores mudam de posições nas bases] Efeito: Esta é uma jogada de a\-pe\-la\-ção.

	Quando a apelação é feita corretamente, cada corredor que tiver mudado de posição nas bases, se for descoberto, será declarado \gls{out}, e o \gls{coach} Principal será expulso por conduta antidesportiva. A ordem das eliminações será determinada pela posição dos corredores imediatamente depois da mudança. O corredor que, após a troca, estiver mais perto do \gls{homeplate} será declarado \gls{out} primeiro.

	O próximo corredor que tiver chegado mais perto do \gls{homeplate} após mudar de posição nas bases será o segundo a ser declarado \gls{out} e assim por diante.

	A apelação pode ser feita a qualquer momento antes que todos os corredores que mudaram de posições estejam no \gls{dugout}, ou o \gls{inning} tenha terminado.

	Se um dos corredores que mudou de base estiver numa base, ele e todos os corredores que tiverem mudado de bases serão declarados \gls{out}s, mesmo que tenham pisado o \gls{homeplate}, e o(s) ponto(s) anotado(s) por esse(s) corredor(es) incorreto(s) será(ão) anulado(s).

 \item[Regra 5.10.3 (c) (i), (c) (ii) e (c) (iii)] -- Se essa Interferência, na opinião do árbitro, é uma evidente tentativa de evitar uma jogada dupla, o corredor
subsequente imediato também deve ser declarado \gls{out}

\item[Regra 5.10.3 (d) -- Interferência do árbitro] Efeito: Deve ser sinalizada uma Bola Morta Demorada; a bola permanece viva até a conclusão da jogada.

\begin{enumerate}[label=\roman*.]
	\item Se o corredor sobre o qual está ocorrendo a jogada é declarado \gls{out}, a eliminação é mantida e a bola continua viva.

	\item Se é declarado \gls{safe}, a bola torna-se morta e todos os corredores têm de retornar à última base que estavam ocupando no momento do lançamento.
\end{enumerate}

\end{description}


\subsection{O CORREDOR NÃO É DECLARADO \gls{out}}
\begin{enumerate}[label=(\alph*)]
	\item   Quando corre atrás ou à frente do defensor, e fora do caminho da base, a fim de evitar interferir na ação de um defensor que está tentando pegar a bola batida no caminho da base.
	\item  Quando não corre em linha reta para a base, desde que o defensor que está na linha reta não esteja de posse da bola.
	\item  Quando mais de um defensor tenta pegar uma bola batida, e ele tem contato com aquele que, na opinião do árbitro, não tinha o direito de pegar a bola.
	\item  Quando, enquanto está fora da base, é atingido por uma bola batida \gls{fair} não tocada, sobre a qual, na opinião do árbitro, nenhum defensor teria conseguido fazer uma jogada para fazer um \gls{out}.
	\item  Quando é atingido por uma bola batida \gls{fair} não tocada, em território \gls{foul}, sobre a qual, na opinião do árbitro, nenhum defensor teria conseguido fazer uma jogada para fazer um \gls{out}.
	\item  Quando é atingido por uma bola batida \gls{fair} após esta ter tocado, ou ter sido tocada por qualquer defensor, incluindo o arremessador, e ele não podia evitar contato com a bola.
	\item   Quando é atingido por uma bola batida \gls{fair} não tocada enquanto está em contato com sua base, a menos que interfira intencionalmente no curso da bola
	 ou na ação de um defensor que está fazendo uma jogada. A bola torna-se morta ou permanece viva, dependendo da posição do defensor que está mais perto da base no momento que a bola tocar o corredor.

	1. a bola permanece viva se o defensor mais próximo da base estiver posicionado à frente da base; ou

	2. a bola fica morta se o defensor estiver atrás da base.

	\item   Quando é tocado enquanto está fora da base:

	1) com uma bola que não está firmemente segura por um jogador da defensiva; ou

	2) com a mão, enquanto segura a bola com a luva, ou com a luva, enquanto a bola está na outra mão.

	\item  Quando, numa Jogada de Apelação, a equipe na defensiva não solicita a decisão do árbitro antes que seja efetuado o arremesso seguinte, legal ou ilegal, ou antes que todos os jogadores da defensiva tenham deixado o território \gls{fair}, a caminho do \gls{bench} ou \gls{dugout}ou ainda no caso de uma ultima jogada do jogo, antes que os árbitros tenham deixado o campo.
	\item   Quando um batedor-corredor se torna um corredor após tocar a primeira base, ultrapassa-a e depois retorna diretamente à base.
	\item   Quando não lhe é concedido tempo suficiente para retornar a uma base. Ele não será declarado \gls{out} por estar fora da base antes do arremessador soltar a bola, e poderá avançar como se tivesse deixado a base legalmente.

	\item Quando tiver iniciado um avanço legalmente. Ele não pode ser interceptado pelo arremessador que está recebendo a bola enquanto está sobre o \gls{pitcher's plate}, nem por aquele que está tocando o \gls{pitcher's plate} enquanto está segurando a bola.
	\item  Quando permanece na sua base e tenta avançar à base seguinte depois que uma bola \gls{fly} tem contato com um defensor.
	\item  Quando desliza para uma base e a desloca de sua posição correta. Considera-se que a base tenha seguido o corredor. Um corredor que chega a salvo (\gls{safe}) a uma base não deve ser declarado \gls{out} por não estar em contato com a base deslocada. Ele pode retornar a essa base, sem o risco de ser declarado \gls{out}, quando ela tiver sido recolocada. Um corredor corre o risco de ser declarado \gls{out} se tentar avançar além da base deslocada antes que ela esteja outra vez na posição correta.
	\item  Quando um \gls{coach} interfere, acidentalmente numa bola lançada ou batida enquanto está dentro do \gls{coachsbox}.
	\item  Quando a bola tem contato com um equipamento não oficial da equipe na ofensiva e não é evidente que poderia ocorrer jogadas. A bola torna-se morta e todos os corredores têm de retornar à última base que estavam ocupando no momento em que a bola foi declarada morta, mas no retorno não precisam tocar as bases intermediárias. (sem tabela?)
\end{enumerate}

\section*{EFEITO}
\addcontentsline{toc}{section}{EFEITOS}

Rule 5.10.4 p Contato com equipamento não oficial e jogada obviamente parada

 Efeito: A bpola é declarada morta e todo corredor deve retornar a ultima base tocada por ele no momento no qual a bola foi declarada morta. No retorno, não há necessidade de retocar as bases

\section{CONCESSÃO DE BASES (EXCETO POR OBSTRUÇÃO)}

 EFEITO -- Regra ou Ocorrência

\subsection{Concessão de Uma Base}

		\begin{enumerate}[label=\roman*.]
		\item O batedor-corredor adquire o direito de ocupar a primeira base, desde que avance e toque a base, e todos os outros corredores podem avançar uma base, se forçados, a partir da base que ocupavam no momento do arremesso, nas seguintes circunstâncias:

		\begin{enumerate}[label=\arabic*)]
		 	\item quando o árbitro de \gls{home} declara o quarto \gls{ball}; a bola é viva;
			\item quando o arremessador deixa o batedor "andar" intencionalmente; a bola torna-se morta;
			\item  quando o batedor é obstruído, e a equipe na ofensiva opta pelo direito de ele ocupar a primeira base; a bola torna-se morta;
			\item  quando uma bola batida tem contato com um árbitro ou corredor antes de passar um defensor, excluindo o arremessador; a bola torna-se morta; ou
			\item  quando o batedor é atingido por um arremesso; a bola torna-se morta.
		\end{enumerate}
		\item Um corredor é autorizado a avançar uma base nas seguintes circunstâncias; a bola torna-se morta, exceto nos casos abaixo:

		\begin{enumerate}[label=\arabic*)]
			 	\item quando o arremessador faz um arremesso ilegal, e esse arremesso não é batido, ou se é batido, o técnico da equipe na ofensiva opta pelo direito do corredor avançar uma base ao invés de aceitar o resultado da jogada; a bola torna-se morta;
				\item  quando a bola arremessada sai do campo de jogo ou fica alojada no \glspl{backstop} (barreira situada atrás do \gls{homeplate}); a concessão é a partir da base que o corredor ocupava no momento do arremesso;
				\item  quando um defensor leva involuntariamente uma bola para fora do campo de jogo; a concessão é a partir da última base tocada no momento em que o defensor deixou o campo de jogo. Um defensor que, de posse de uma bola viva, entra no \gls{dugout} ou na área da equipe para tocar um corredor é considerado como se a tivesse levado involuntariamente à área de bola morta;
				\item  quando um jogador perde a posse da bola durante uma jogada, e essa bola vai para área de bola morta, a concessão é a partir da última base tocada no momento em que a bola entrou na área de bola morta;
				\item  quando um equipamento da defensiva causa uma "Bola Bloqueada", a concessão é a partir da última base tocada pelos corredores no momento em que ocorreu o arremesso; e
				\item  quando um equipamento que está fora do lugar apropriado tem contato com uma bola arremessada.

			  Se uma bola arremessada escapa do receptor, e este a recupera com equipamento fora do lugar apropriado quando nenhum corredor está avançando, nenhuma possível jogada é evidente, ou não ocasiona qualquer vantagem, não deve ser concedida uma base a nenhum corredor; a bola permanece viva, e o batedor pode avançar à primeira base somente quando obtém base por \glspl{ball} (\gls{baseonballs}) ou quando é aplicada a Regra do Terceiro \gls{strike}. Ele pode avançar além da primeira base a seu próprio risco.
		\end{enumerate}

\subsection{Concessão de duas bases}

	\begin{enumerate}[label=\roman*.]
	\item O batedor-corredor e o(s) corredor(es) são autorizados a avançar duas bases a partir da base que ocupavam no momento do arremesso e a bola torna-se morta nas seguintes circunstâncias.

	 \begin{enumerate}[label=\arabic*)]
		 	\item quando uma bola batida \gls{fair} vai para fora de um campo de jogo a uma  distância menor do que as dimensões de um campo oficial;
			\item  quando uma bola batida \gls{fair fly} (\gls{fly} que está em território \gls{fair}) toca a luva ou o corpo do defensor e passa sobre a cerca em território \gls{foul};
			\item  quando uma bola batida \gls{fair fly} toca a cerca, desvia após ter contato com  um defensor e passa por cima da cerca;
			\item  quando uma bola batida \gls{fair} é tocada por um defensor que está em território de bola morta e, na opinião do árbitro, a bola não teria passado sobre a cerca em território \gls{fair};
			\item  quando uma bola batida \gls{fair} pula sobre uma cerca, ou rola por baixo ou através de uma cerca, ou vai para fora da linha que delimita o campo de jogo;
			\item  quando uma bola batida \gls{fair} é desviada:

%			\begin{enumerate}[label=(\alph*)]
%				\item
				(a) por um jogador da defensiva ou árbitro, ou
%				\item

				(b) por um corredor, depois de ter passado um defensor, exceto arremessador, e desde que nenhum outro defensor tenha tido uma chance de fazer um \gls{out} e a bola tenha ficado fora de jogo em território \gls{foul}.
%			\end{enumerate}

		\item  quando uma bola batida \gls{fair} tem contato com um defensor que está em  território de bola morta e, na opinião do árbitro, a bola não teria passado sobre a cerca em território \gls{fair}.


			\item Quando a bola lançada sai do campo de jogo ou é bloqueada, a concessão é a partir da base que o corredor ocupava no momento em que a bola deixou a
			 mão do defensor. Se dois corredores estão entre as mesmas duas bases, a concessão é baseada na posição do corredor precedente. Se um corredor toca
			 a base seguinte e retorna para sua base original, essa base original é considerada a "última base tocada" para os propósitos de concessão de bases
			 em razão de um lançamento descontrolado (\gls{overthrow}).

			\item Quando um equipamento da defensiva causa uma Bola Bloqueada, a concessão é:

%				 \begin{enumerate}[label=\arabic*)]
%				 	\item
				 	1) a partir da última base tocada no momento do lançamento; ou
%					 \item

					2) a partir da última base tocada no momento do arremesso, numa bola batida \gls{fair}.
%				\end{enumerate}

			\item  Quando uma bola lançada tem contato com um equipamento fora do lugar apropriado, deve ser declarada uma Bola Morta Demorada.
			\item Um corredor é autorizado a avançar somente duas bases, e a bola torna-se  morta quando, na opinião do árbitro, um defensor leva, chuta, empurra ou lança uma bola viva da área de jogo para área de bola morta, intencionalmente. a  concessão é a partir do momento em que a bola foi chutada, empurrada ou lançada, ou do momento em que a bola foi levada para área de bola morta.
		\end{enumerate}
	\end{enumerate}

\subsection{Concessão de Três Bases}

		 O batedor-corredor e os corredores são autorizados a avançar três (3) bases, e deve ser declarada uma Bola Morta Demorada, quando um equipamento fora do lugar apropriado tem contato com uma bola batida \gls{fair}. Bases serão concedidas a partir do momento do arremesso. Os corredores são protegidos até o momento que tocarem as bases concedidas, porém caso um deles queira tentar conquistar outra base, é permitido mas com risco de ser declarado \gls{out}

\subsection{Concessão de Quatro Bases}

		 O batedor-corredor e os corredores são autorizados a ir para \gls{home}, e a bola torna-se morta nas seguintes circunstâncias:

		\begin{enumerate}[label=\roman*.]
		\item quando o árbitro declara um \gls{homerun}; ou
		\item quando uma bola \gls{fair} tem contato com um equipamento fora do lugar apropriado, e, na opinião do árbitro, ela teria passado sobre a cerca do campo externo em voo.
	\end{enumerate}

\subsection{Concessões de acordo com a opinião do árbitro}

O batedor-corredor e os corredores são autorizados a avançar à base que, na opinião do árbitro, teriam alcançado se não tivesse ocorrido a Interferência; a bola torna-se morta:

 \begin{enumerate}[label=\roman*.]
 	\item Quando uma pessoa, exceto um membro da equipe, interfere numa bola batida \gls{ground} ou numa bola lançada, ou estorva um defensor que está preparado para pegar uma bola, incluindo bolas \gls{fly}.Se no julgamento do arbitro o defensor conseguiria apanhar a bola caso a interferência não tivesse ocorrido, batedor corredor é declarado out e o corredor deve retornar a base que tocou por ultimo antes da interferência , ou
\item  Quando a bola fica alojada no equipamento ou roupa do árbitro ou na roupa de um jogador da ofensiva.
\end{enumerate}
\end{enumerate}


\appendix
	\renewcommand{\thechapter}{ANEXO~\arabic{chapter}}
	\renewcommand{\thesection}{\Alph{section}}
	\renewcommand{\thesubsection}{\Alph{subsection}}


\chapter{DESENHO DO CAMPO DE JOGO E \textit{DIAMOND}}
\minitoc% Creating an actual minitoc


\section{DIMENSÕES OFICIAIS DO CAMPO DE JOGO}

\section{DIMENSÕES OFICIAIS DO DESENHO DO \textit{DIAMOND}}

\section{DIMENSÕES OFICIAS DAS BASES}

\section{DIMENSÕES OFICIAIS DO \textit{BATTER'S BOX} E \textit{CATCHER'S BOX}}

\section{DIMENSÕES OFICIAIS DO \textit{HOME PLATE} E \textit{PITCHER'S PLATE}}

\section{TABELA DE REFERÊNCIA RÁPIDA \textit{BACKSTOP} E LINHAS LATERAIS (LINHA DE BOLA MORTA/CERCA LATERAL)}

\begin{description}

 \item[\textit{BACKSTOPS}] (barreira situada atrás da área do \gls{homeplate}) e as linhas/cercas laterais devem estar situadas a 7,62m (25 pés), no mínimo, e a 9,14m (30 pés), no máximo, atrás das linhas de \gls{foul}.
 A área entre as linhas de \gls{foul} e o \glspl{backstop}, e entre as linhas de \gls{foul} e as linhas/ cercas laterais, tem de estar  desobstruída.

\item [BASES]
 Distâncias:
 \begin{itemize}
 	\item  \gls{homeplate} até primeira/terceira base: 18,29m (60 pés) da parte de trás da placa até a parte de trás da base.
 	\item \gls{homeplate} até segunda base: 25,86m (84 pés 10 \textonequarter{}  polegadas) da parte de trás da placa até o meio da base.

 \end{itemize}

 As bases devem ser feitas de lona ou outro material apropriado, e devem estar firmemente fixadas em sua posição.

 Metade da Base Dupla (Primeira Base) é fixada em território \gls{fair}, e é parte do território \gls{fair}, e a outra metade (de cor bem diferente e contrastante), em território \gls{foul}, e é parte do território \gls{foul}.

\item [\textit{BATTER'S BOXES}]
 Um em cada parte do \gls{homeplate}. Devem medir 0,91m (3 pés) por 2,13m (7 pés). As linhas internas do \gls{battersbox} devem estar a 15,20cm (6 polegadas)
 do \gls{homeplate}. A linha dianteira do \gls{box} deve estar a 1,22m (4 pés) na frente de uma linha traçada através do centro do \gls{homeplate}. As linhas são consideradas dentro do \gls{battersbox}.


\item [\textit{CATCHER'S BOX}]
 Medida: 3,05m (10 pés) de comprimento dos cantos externos traseiros dos \glspl{batter's box} e deve ter 2,57m (8 pés 5 polegadas) de largura. As linhas são consideradas dentro do \gls{catcher's box}.

\item [\textit{COACHES' BOXES}]
 É aquela área atrás de uma linha de 4,57m (15 pés) traçada fora do \gls{diamond} (campo). A linha é paralela à linha da primeira/terceira base, e está a 3,66m (12 pés) dessas linhas, estendidas das bases em direção ao \gls{homeplate}.

 TABELA DE DISTÂNCIAS

\begin{center}
	\begin{tabular}{*{6}{|c|c}|}\hline
		\multicolumn{2}{|c|}{\multirow{2}{*}{CATEGORIA}} &
		\multicolumn{2}{c|}{"H. PLATE"-"P. PLATE"}&
		\multicolumn{2}{c|}{\parbox{40mm}{CERCAS DO CAMPO EXTERNO (mínimas)}}\\\cline{2-6}
		\multicolumn{2}{|c|}{}&m&pés&m&pés\\\hline
		Junior Fem. &( 16u) &12,19 & 40 & 67,06 & 220 \\\hline
		Junior Fem. &( 19u) &13,11 & 43 &67,06 &220 \\\hline
		Mulheres Adulto& &13,11 & 43 & 67,06  &220 \\\hline
		Junior Masc. &(16u)& 14,02 &46 & 76,20 &250  \\\hline
		Junior Masc. &(19u)& 14,02 &46 & 76,20  &250  \\\hline
		Homens Adulto& &14,02 &46 & 76,20  &250  \\\hline
	\end{tabular}
\end{center}


\item [\textit{HOME PLATE}] Deve ter cinco lados.

	A borda voltada para o arremessador deve ter 43,20cm (17polegadas) de largura.

	Os lados devem ser paralelos às linhas internas do \gls{battersbox} e devem ter 21,60cm (8 \textonehalf  polegadas) de comprimento.

	Os lados da ponta voltada ao receptor devem ter 30,50cm (12 polegadas) de comprimento.

\item [CAMPO INTERNO] É aquela parte do campo, sem grama, que forma um arco a 18,29m (60 pés) do centro da borda dianteira do \gls{pitcher's plate}.

\item [LINHAS] Devem ter 50mm a 100mm (2 a 4 polegadas) de largura.

\item [CÍRCULO DO BATEDOR PREVENIDO] É um círculo com 1,52m (5 pés), 0,76m (2 pés e 6 polegadas) de raio, localizado próximo ao fim da área do \gls{bench} ou \gls{dugout} dos jogadores mais perto do \gls{homeplate}.

\item [LINHA DE UM METRO] Linha traçada paralelamente à linha de base, e a um metro (3 pés) dessa linha, partindo de um ponto onde inicia a segunda metade da distância entre o \gls{homeplate} e a primeira base.

\item [CÍRCULO DO ARREMESSADOR] É um círculo de 4,88m (16 pés), com raio de 2,44m (8 pés), traçado do centro da borda dianteira do \gls{pitcher's plate}. As linhas são consideradas dentro do círculo.

\item [\textit{PITCHER'S PLATE}]
 É feito de borracha e tem 61cm (24 polegadas) de comprimento e 15,2cm (6 polegadas) de largura. A parte superior da placa tem de estar no mesmo nível do solo.

\item [ZONA DE ADVERTÊNCIA]
 Deve estar marcada a 3,66m (12 pés), no mínimo, e 4,57m (15 pés), no máximo, da cerca do campo externo e/ou das cercas laterais.

 A marcação deve ser feita com material (terra, cascalho) equivalente (mas diferente) ao da superfície do campo.

 O material tem que ser distinguível do material da superfície do campo externo, e deve chamar a atenção dos jogadores quando eles estão se aproximando da cerca.
\end{description}

\section{TRAÇANDO UM \textit{DIAMOND}}

Esta seção serve como um exemplo para traçar um campo (\gls{diamond}) com distância de 18,29m (60 pés) entre as bases e 14,02m (46 pés) entre o \gls{homeplate} e o \gls{pitcher's plate}.

\begin{enumerate}[label = (\arabic*)]


\item   Para determinar a posição do \gls{homeplate},
\begin{enumerate}
	\item trace uma linha na direção em que deseja situar o campo.
	\item fixe uma estaca no canto do \gls{homeplate} mais perto do receptor.
	\item amarre um cordão nessa estaca e dê nós ou marque a corda de outra forma %após medir
	os seguintes comprimentos:

\begin{center}
	\begin{tabular}{c|l}\hline
		m & pés polegadas\\\hline
		14,02 &46 '\\\hline
		18,29 &60 '\\\hline
		25,86 &84 ' e 10 \textonequarter{} " \\\hline
		36,58 &120 '\\\hline
	\end{tabular}
\end{center}


\end{enumerate}

\item   Coloque o cordão (sem esticar) ao longo da linha diretora e coloque uma estaca onde marca 14,02m (46 pés). Esta será a linha de frente no meio do \gls{pitcher's plate}. Ao longo da mesma linha, fixe uma estaca onde marca 25,86m (84 pés e 10 \textonequarter{} polegadas). Este será o centro da segunda base.

\item  Coloque o ponto onde marca 36,58m (120 pés) no local determinado para o centro da segunda base e, pegando o cordão no ponto onde marca 18,29m (60 pés), ande para a direita da linha diretora até esticá-lo e pregue uma estaca no ponto onde marca 18,29m (60 pés) -- este será o canto externo da primeira base, e o cordão, agora, formará a linha entre a primeira e a segunda bases.

\item  Segurando, outra vez, o cordão no ponto onde marca 18,29m (60 pés), atravesse o campo e, da mesma maneira, marque o canto externo da terceira base. O \gls{homeplate}, a primeira base e a terceira base estão inteiramente na parte interna do campo (\gls{diamond}).
\item  Para conferir as medidas do campo (\gls{diamond}), coloque a ponta da corda que marca o \gls{homeplate} na estaca da primeira base, e o ponto onde marca 36,58m (120 pés), na terceira base. O ponto onde marca 18,29m (60 pés) deve, agora, coincidir com os locais marcados para o \gls{homeplate} e a segunda base.
\item  Confira todas as distâncias com uma fita métrica metálica sempre que possível.
\end{enumerate}


\chapter{ESPECIFICAÇÕES DO \textit{BAT}}
\minitoc% Creating an actual minitoc

\section{\textit{BAT} OFICIAL}

\begin{enumerate}[label=(\arabic*)]
	\item  O \gls{bat} tem de ser feito com uma peça só, com várias peças juntadas definitivamente, ou com duas peças trocáveis.
	\item  Quando o \gls{bat} é projetado para ser feito com componentes trocáveis, tem de levar em conta o seguinte critério:

		\begin{enumerate}[label=\roman*.]
			\item os componentes acoplados devem ter um dispositivo de segurança especial para evitar que equipamento com combinações não aprovadas seja usado no campo; e
			\item os \glspl{bat} confeccionados com combinações de componentes têm de seguir os padrões estabelecidos como se fossem um \gls{bat} feito com uma peça só. Os  componentes têm de seguir os padrões estabelecidos como se fossem partes de um \gls{bat} feito com uma peça só.
		\end{enumerate}

	\item  Um \gls{bat} pode ser feito com um pedaço de madeira de lei (madeira dura), ou com um bloco de madeira composto de dois ou mais pedaços de madeira colados entre si com um adesivo, de tal forma que a direção das fibras de todas as peças seja paralela ao comprimento do \gls{bat}.
	\item  Um \gls{bat} pode ser de metal, bambu, plástico, grafite, carbono, magnésio, fibra de vidro, cerâmica, ou qualquer outro material composto aprovado pela WBSC-SD ou ISF \gls{ESC}.
	\item  Um \gls{bat} pode ser laminado, mas deve conter somente madeira ou adesivo, e ter um acabamento perfeito (quando pronto).
	\item  A parte mais grossa do \gls{bat} (do início da parte cônica até a ponta do \gls{bat}) deve ser redonda e lisa.
	\item  Não deve ter mais de 86,40cm (34 polegadas) de comprimento, nem pesar mais de 1077,00g (38 onças).
	\item  Não deve ter mais de 5,70cm (2 \textonequarter{} polegadas) de diâmetro em sua parte mais grossa. É permitida uma tolerância de 0,80mm (1/32 polegada) devido à dilatação que pode haver no material.
	\item  Um taco que tenha quaisquer rebites expostos, pinos, bordas ásperas ou afiadas ou qualquer prendedor externo que seja ou apresente um risco é um taco ilegal. Um taco que não seja de madeira deve estar livre de rebarbas e rachaduras.
	\item  Um taco ou bastão que não seja de madeira, não deve ter cabo de madeira.

	\item  Um \gls{bat} tem de ter uma empunhadura de segurança de cortiça, fita (fita plástica não lisa) ou material composto.

	A empunhadura de segurança não deve ter menos de 25,40cm (10 polegadas) de comprimento e não deve se estender mais de 38,10cm (15 polegadas) da extremidade do cabo.

	É permitido aplicar	resina, alcatrão de pinho ou substâncias em spray somente na empunhadura de segurança, para aumentar a sua eficiência.

	A fita aplicada a qualquer \gls{bat} tem	de ser em espiral contínua. Não precisa ser uma camada sólida de fita.

	Não deve exceder duas camadas.

	\item  Um taco que não seja de madeira e não seja feito de uma peça com a extremidade do cano fechada, deve ter uma borracha ou plástico de vinil ou outro material que seja aprovado pela Comissão de Padrões de Equipamentos WBSC, que esteja firmemente preso na extremidade maior do mesmo."

	\begin{enumerate}[label=\roman*.]
		\item A tampa colocada na extremidade aberta da parte grossa do \gls{bat} tem de estar firme e permanentemente lacrada, para que ela não possa ser removida por qualquer pessoa, exceto o fabricante, sem danificá-la ou  destruí-la.
		\item O \gls{bat} não deve causar ruídos. Um \gls{bat} que causa ruídos será considerado um \gls{bat} ilegal. \gls{bat} não deve ter sinais de adulteração. Um \gls{bat} que mostra sinais de adulteração será considerado um \gls{bat} Adulterado.
	\end{enumerate}

	\item  Um \gls{bat} tem que ter um dispositivo de segurança (saliência arredondada na extremidade do cabo) de, no mínimo, 0,60cm (\textonequarter{} de polegada) ressaltando, a um ângulo de 90 graus, do cabo, e não deve ter bordas afiadas.

	O dispositivo de segurança pode ser moldado, torneado, soldado e permanentemente fixo; pode ser coberto com fita.

	\item  Um \gls{bat} que tenha a informação \gls{bat} Oficial Aprovado" ilegível, devido ao desgaste pelo uso, pode ainda ser utilizado se todos os outros aspectos estiverem de acordo com as regras, e desde que isso possa ser constatado com razoável segurança.

	\item  O peso, a distribuição do peso, ou o comprimento do \gls{bat} têm de ser estabelecidos permanentemente por ocasião da fabricação, e não podem ser modificados de maneira alguma depois disso, excetuando-se algo diferente que esteja especificamente previsto nesta Regra, ou haja uma especificação aprovada pela WBSC-SD ou ISF \gls{ESC}.
\end{enumerate}

\section{\textit{BAT} PARA FAZER AQUECIMENTO}

É um \gls{bat} -- exceto um \gls{bat} oficial -- que tem de ser feito com uma peça só, e deve sujeitar-se aos requisitos exigidos aos dispositivos de segurança (empunhadura de segurança e saliência arredondada na extremidade do cabo) do \gls{bat} oficial.

Tem de estar marcado \textit{warm-up}, com letras de 3,20cm (1 \textonequarter{}  polegada), na extremidade do cilindro.

A extremidade do cilindro tem que ter mais de 5,70cm (2 \textonequarter{}  polegadas).

\chapter{PADRÕES DE BOLA}
\minitoc% Creating an actual minitoc

\section{BOLA OFICIAL}

\begin{enumerate}[label=(\arabic*)]
	\item   Tem que ser uma bola com formato regular, emendas lisas, pontos de costura não salientes ou com superfície plana.
	\item  Tem que ter um núcleo central feito tanto de fibra longa de paina de primeira qualidade, de uma mistura de cortiça e borracha, de uma mistura de poliuretano, como de outros materiais aprovados pela WBSC-SD \gls{ESC}.
	\item  Pode ser enrolada (manualmente ou a máquina) com fio trançado de boa qualidade e coberta com cola de látex ou borracha.
	\item  Tem que ter:
	\begin{itemize}
		\item 	uma cobertura costurada com fio encerado de algodão ou linho, colada à bola mediante aplicação de substância aderente na face inferior (da cobertura), ou
		\item uma cobertura moldada colada ao núcleo, ou
		\item uma cobertura integralmente moldada com o núcleo.
	\end{itemize}


	As peças moldadas devem ter uma reprodução autêntica da costura aprovada pela WBSC-SD \gls{ESC}.
	\item  Tem que ter uma cobertura da melhor qualidade, feita de couro de cavalo ou vaca curtido em cromo No 1, ou de material sintético ou outros materiais aprovados pela WBSC-SD \gls{ESC}.
\end{enumerate}

\section{DIMENSÕES E ESPECIFICAÇÕES}

\begin{enumerate}[label=(\arabic*)]
	\item  A bola de 30,50cm (12 polegadas), pronta, deve

	\begin{itemize}
		\item ter entre 30,20cm (11 7/8 polegadas) e 30,80cm (12 1/8 polegadas) de circunferência, e
		\item pesar entre 178,00g (6 \textonequarter{} onças) e 198,40g (7 onças).
	\end{itemize}


	O tipo "costura plana" deve ter, no mínimo, 88 pontos em cada cobertura, costurados pelo método de duas agulhas.
	\item  A bola pronta deve ter um coeficiente de restituição e um padrão de compressão, que serão determinados e instituídos pela WBSC-SD \gls{ESC}.
	\item  COR significa Coeficiente de Restituição de uma bola quando medido pelo método de teste para medir o Coeficiente de Restituição de bolas da ASTM (American Society for Testing and Materials).
	\item  A bola de 30,50cm (12 polegadas), com costura branca ou vermelha ou cobertura amarela com um COR de .47 ou menos, deve ser usada em jogos de Campeonato da WBSC-SD, nas seguintes categorias: Adultos (Masculino e Feminino), Júnior (Masculino e Feminino). As bolas devem ter a logomarca da WBSC-SD.
	\item  Em bolas usadas em Jogos de Campeonato da WBSC-SD, a força de carga exigida para comprimir a bola 0,64cm (0,25 polegadas) não precisa exceder 170,10kg (375 libras) quando tais bolas são testadas de acordo com o método de provas para medir compressão-deslocamento de bolas de softbol da ASTM, método esse aprovado pela WBSC-SD \gls{ESC}.
\end{enumerate}

 Abaixo, estão relacionados os padrões estabelecidos para cada bola:

 \begin{center}
 \begin{tabular}{ll}\hline
	Bola&  30,5 cm Diametro ( 12')\\\hline
	Cor da Bola&  Branca ou Amarela WBSC SD\\\hline
	Cor da Costura &  Branca ou Vermelho prespontado\\\hline
	Tamanho Minimo & 30,2 cm ( 11-7/8'')\\\hline
	Tamanho Maximo& 30,8 cm ( 12-1/8'')\\\hline
	Peso Minimo & 178.0 g. (6 \textonequarter{}  oz.)\\\hline
	Peso Maximo & 198.4 g. (7 oz.)\\\hline
\end{tabular}
 \end{center}



\chapter{ESPECIFICAÇÕES DA LUVA}
\minitoc% Creating an actual minitoc

ESPECIFICAÇÕES DAS DIMENSÕES:


	\begin{tabular}{llll}\hline
	\multirow{2}{*}{Posição} &
	\multirow{2}{*}{Descrição}&
	\multicolumn{2}{c}{medidas}\\
	&&cm & pol.\\\hline
	A& Largura da palma (parte superior)&20,30&8\\\hline
	B& Largura da palma (parte inferior)&21,60&8 \textonehalf \\\hline
	C& Abertura da parte superior do trançado&12,70&5 \\\hline
	D&  Abertura da parte inferior do trançado&11,50&4 \textonehalf \\\hline
	E&  Topo à base do trançado&18,40&7 \textonequarter{}  \\\hline
	F&  Costura da forquilha do primeiro dedo&19,00&7 \textonehalf \\\hline
	G&  Costura da forquilha do polegar&19,00&7 \textonehalf \\\hline
	H&  Costura da forquilha&44,50&17 \textonehalf \\\hline
	I&  Topo do polegar à borda inferior&23,50&9 \textonequarter{} \\\hline
	J&  Topo do primeiro dedo à borda inferior&35,60&14\\\hline
	K&   Topo do segundo dedo à borda inferior&33,70&13 \textonequarter{} \\\hline
	L&  Topo do terceiro dedo à borda inferior&31,10&12 \textonequarter{} \\\hline
	M& Topo do quarto dedo à borda inferior&27,90&11 \\\hline

\end{tabular}















\chapter{ÁRBITROS}
\minitoc% Creating an actual minitoc

\section{INFORMAÇÕES GERAIS PARA ÁRBITROS}

\begin{enumerate}[label=(\alph*)]
	\item  O árbitro oficial não deveria ser um membro de nenhuma das equipes.

 Exemplos: jogador, \gls{coach}, técnico, dirigente, anotador ou patrocinador.

 	\item O árbitro deve estar seguro quanto à data, ao horário e ao local do jogo, e deve chegar ao campo de jogo com antecedência de 20 a 30 minutos, iniciar o
jogo pontualmente e deixar o campo depois de encerrá-lo.

	\item O árbitro (masculino e feminino) tem de usar:

	\begin{enumerate}[label=(\arabic*)]
		\item Uma camisa azul-claro, com mangas longas ou curtas.
		\item  Meias azul-marinho escuro.
		\item  Calças azul-marinho escuro.
		\item  Boné azul-marinho escuro, com a marca WBSC (em letras brancas decoradas com contornos azuis) pregada na frente.
		\item  Bolsa para bolas azul-marinho escuro (somente árbitro de \gls{home}).
		\item  Jaqueta e/ou pulôver azul-marinho escuro.
		\item  Sapatos e cinto pretos.
		\item  Uma camiseta branca, que deve ser usada por baixo da camisa azul-claro.
	\end{enumerate}

\item Árbitros não devem usar joias expostas que possam oferecer risco.

 EXCETO braceletes e/ou colares com fins medicinais.

\item O árbitro de \gls{home}, na modalidade Arremesso Rápido, tem de usar uma máscara para rosto preta, com estofamento preto ou bege, um protetor de garganta preto, um protetor de tórax e caneleiras que protejam também os joelhos. (Pode ser usada uma máscara que já vem dotada de um protetor de garganta na parte inferior da armação.)
\item Os árbitros devem apresentar-se aos capitães, técnicos e anotadores.
\item  Os árbitros devem inspecionar as delimitações do campo de jogo, o equipamento etc. e esclarecer todas as regras de campo para ambas as equipes e seus \glspl{coach}.
\item  Cada árbitro tem o poder de tomar decisões sobre as infrações cometidas a qualquer momento durante o desenrolar da partida, ou enquanto ela está paralisada, até o jogo ser encerrado.

\item   Nenhum árbitro tem autoridade para desprezar ou questionar as decisões tomadas por outro árbitro dentro dos limites de seus respectivos deveres, conforme está especificado nestas regras.

\item  Um árbitro pode consultar seu companheiro a qualquer momento. Contudo, a decisão final deve ser do árbitro que, mesmo tendo autoridade exclusiva para decidir, recorreu à opinião de outro árbitro.
\item  Para definir suas respectivas obrigações, o árbitro que julga as bolas arremessadas (\gls{ball} ou \gls{strike}) será designado como o "Árbitro de Home", e o árbitro que julga as decisões nas bases, como o "Árbitro de Base".

\item  O árbitro de \gls{home} e o árbitro de base devem ter a mesma autoridade para:

	\begin{enumerate}[label=(\arabic*)]
		\item Declarar declarado \gls{out} um corredor que sai da base antecipadamente.
		\item Declarar \gls{time} para paralisar o jogo.
		\item  Remover, ou expulsar do jogo, um jogador, \gls{coach} ou técnico, por violação de regras.
		\item  Declarar todos os Arremessos Ilegais.
		\item  Determinar e declarar um \gls{infieldfly}. Quando parecer evidente que uma bola batida será um \gls{infieldfly}, o árbitro deverá declarar, imediatamente, \gls{infieldfly} SE FOR \gls{fair}, O BATEDOR É \gls{out}, para beneficiar os corredores.
	\end{enumerate}

\item  O árbitro deve declarar \gls{out} o batedor, batedor-corredor ou corredor, sem esperar por uma apelação para tal decisão, em todos os casos em que esse jogador é declarado \gls{out} de acordo com estas regras.
\item   A menos que haja uma apelação, o árbitro não deve

 declarar \gls{out} um jogador, ou
  penalizá-lo, por
  \begin{itemize}
  	\item não ter tocado uma base;
  	\item ter deixado uma base antecipadamente numa bola \gls{fly} pega no ar;
  	\item ter batido fora de ordem;
	\item ter entrado no jogo como substituto sem ser anunciado ao árbitro;
	\item ter reingressado ilegalmente;
	\item ter entrado no jogo como Jogador de Emergência, ou
	\item ter retornado ao jogo após ter sido removido de acordo com a Regra de Jogador de Emergência, sem comunicação ao árbitro;
	\item ter mudado de posições nas bases com outro corredor; ou
	\item ter tentado ir à segunda base depois de chegar à primeira base, conforme está estabelecido nestas regras.
  \end{itemize}

\item   Os árbitros não devem penalizar uma equipe por infração de uma regra quando a imposição da penalidade pode resultar em vantagem à equipe infratora.
\item  A inobservância, pelos árbitros, das instruções contidas no Anexo 5 não é motivo para protesto. Estas instruções são normas de procedimento para árbitros.
\end{enumerate}

\section{SINAIS}

\begin{enumerate}[label=(\alph*)]
	\item  Para indicar que a partida deve começar, ou ser retomada, o árbitro deve declarar \gls{play ball} e, ao mesmo tempo, sinalizar para o arremessador efetuar arremesso.
	\item Para indicar um \gls{strike}, o árbitro deve levantar a mão direita acima do ombro (dobrar o cotovelo, de modo que forme um ângulo de 90 graus) e, ao mesmo tempo, declarar \gls{strike} com voz clara e firme.
	\item Para indicar um \gls{ball}, não deve ser usado sinal algum com o braço.
	\item Para indicar a CONTAGEM de \glspl{ball} e \glspl{strike}, o árbitro deve declarar a quantidade de \glspl{ball} primeiro.
	\item Para indicar um \gls{foul}, o árbitro deve declarar \gls{foulball} e estender ambos os braços, verticalmente, acima da cabeça.
	\item Para indicar uma bola \gls{fair}, o árbitro deve estender um braço na direção do centro do campo, agitando-o para a frente e para trás, imitando um movimento de bombeamento.
	\item  Para indicar que um batedor ou corredor é \gls{out}, o árbitro deve levantar a mão direita, com o punho cerrado, acima do ombro direito.
	\item  Para indicar que um jogador é \gls{safe}, o árbitro deve estender ambos os braços, horizontalmente, para os lados do corpo, com as palmas das mãos viradas para o solo.

	\item Para indicar a paralisação da partida, o árbitro deve declarar \gls{time} e, ao mesmo tempo, estender ambos os braços acima da cabeça. Os outros árbitros devem confirmar a paralisação da partida, imediatamente, fazendo o mesmo gesto.

	\item  Para indicar uma BOLA MORTA DEMORADA (\gls{delayeddeadball}), o árbitro deve estender o braço esquerdo, horizontalmente, mantendo o punho cerrado.
	\item  Para indicar um \gls{trappedball}(bola bloqueada), o árbitro deve estender ambos os braços, horizontalmente, para os lados do corpo, com as palmas das mãos viradas para o solo.

	\item Para indicar que o batedor-corredor (ou corredor) tem direito a duas bases (\gls{ground rule double}), o árbitro deve estender a mão direita
	 acima da cabeça e, ao mesmo tempo, indicar com dois dedos o número de bases concedidas.
	\item Para indicar um \gls{homerun}, o árbitro deve estender a mão direita com o punho cerrado, acima da cabeça, e fazer um movimento circular em sentido horário.
	\item Para indicar um \gls{infieldfly}, o árbitro deve declarar \gls{infieldfly}. se for \gls{fair}, o batedor é \gls{out}. O árbitro deve estender um braço acima da cabeça.
	\item Para indicar que o arremessador não deve efetuar o arremesso (\gls{nottopitch}), o árbitro deve levantar uma mão com a palma da mão virada para o arremessador. Deve ser declarado "NO PITCH" (arremesso nulo) se o arremessador efetua o arremesso enquanto o árbitro está com sua mão na posição mencionada.
\end{enumerate}

\chapter{ANOTAÇÃO}
\minitoc% Creating an actual minitoc

\section{REGISTRO DE DADOS (\textit{BOX SCORE})}

		O nome de cada jogador e a posição, ou posições, a ser(em) ocupada(s) devem ser relacionadas na ordem em que ele bateu, ou teria batido, a menos que o jogador seja substituído legalmente, expulso ou removido do jogo, ou o jogo termine antes de sua vez de bater.

		Quaisquer dados estatísticos acumulados pelo Jogador de Emergência enquanto estava no jogo são creditados a esse jogador, mesmo que ele seja um substituto constante da lista que, eventualmente, não entre no jogo para substituir outro jogador.

		Quaisquer dados estatísticos acumulados por um Corredor Temporário devem ser creditados ao jogador por quem ele está correndo.

		\begin{enumerate}[label=\arabic*)]
			\item O Jogador Designado (JD) é opcional, mas, se vai utilizar um, seu nome tem de ser anunciado antes do início do jogo e estar relacionado no Formulário de Anotações, na ordem correta em que vai atuar como batedor. Serão relacionados dez nomes, e o décimo nome será o do "JOGADOR FLEX"{} por quem o JD está batendo.

			%% (a) Os dados referentes à atuação de cada jogador, na ofensiva e na defensiva, têm de estar tabulados.

			\item A primeira coluna deve indicar o número de vezes que cada jogador atua como batedor (\textit{at bat}), mas não deve ser imputado um \textit{at bat} contra o jogador quando esse batedor

				\begin{enumerate}[label=(\alph*)]
					\item Bate um \gls{fly} de sacrifício que ocasiona a marcação de um ponto.
					\item É autorizado a "andar" (\gls{walk}, \gls{baseonballs}).
					\item É autorizado a ir à primeira base por ter sofrido Obstrução.
					\item Executa um \gls{bunt} de sacrifício.
					\item É atingido por uma bola arremessada (\gls{hitbypitch}).
			\end{enumerate}

			\item A segunda coluna deve indicar a quantidade de pontos anotados por cada jogador.

			\item A terceira coluna deve indicar a quantidade de \glspl{basehit} (batidas indefensáveis) batidos por cada jogador. Uma batida indefensável é aquela que permite que o batedor chegue a salvo (\gls{safe}) a uma base.

				\begin{enumerate}[label=(\alph*)]
				\item Quando um batedor-corredor chega a salvo (\gls{safe}) à primeira base, ou a qualquer base subsequente, numa bola \gls{fair} que fica no campo, transpõe a cerca, ou atinge a cerca antes de ser tocada por um defensor.
				\item Quando um batedor-corredor chega a salvo (\gls{safe}) à primeira base numa bola \gls{fair} -- muito forte, ou muito lenta, ou que dá um salto incomum -- impossível de ser defendida por um defensor, com um esforço normal, a tempo de eliminar esse batedor-corredor.
				\item Quando uma bola \gls{fair} que não tenha tido contato com um defensor se torna "morta" por ter tocado o corpo ou a roupa de um corredor ou árbitro.
				\item Quando o defensor tenta, sem sucesso, eliminar um corredor precedente e, na opinião do anotador, o batedor-corredor não teria sido declarado \gls{out} na primeira base com uma defesa perfeita.
				\item Quando o batedor termina o jogo com uma batida indefensável que empurra os pontos necessários para colocar a sua equipe em vantagem no placar, a ele deve ser creditado somente um \gls{basehit} de tantas bases quantas tiver avançado o corredor que anotou o ponto da vitória, com a condição de que ele (o batedor) avance o mesmo número de bases.
			\end{enumerate}

			EXCEÇÃO: Quando o batedor termina o jogo com um \gls{homerun} batido para fora do campo, ele deve ser creditado com um \gls{homerun}, e todos os corredores, incluindo ele, devem ser autorizados a anotar ponto.

			\item A quarta coluna deve indicar a quantidade de adversários declarado \gls{out}s por cada jogador (\gls{putout}).
				\begin{enumerate}[label=(\alph*)]
					\item Um \gls{putout} é creditado a um defensor cada vez que ele:

					\begin{enumerate}[label=\arabic*)]
						\item Pega uma bola batida para o ar (\gls{fly}) ou uma batida em linha reta (\gls{line drive}).
						\item Pega uma bola lançada que elimina um batedor ou corredor.
						\item Toca um corredor com a bola quando esse corredor está fora da base na qual tem o direito de permanecer.
						\item Está mais perto da bola quando um corredor é declarado \gls{out} por ter sido atingido por uma bola \gls{fair} ou por ter estorvado um defensor.
						\item Está mais perto do jogador substituto não anunciado, que é declarado \gls{out} de acordo com a Regra 3.2.8.
						\item Está mais perto de um corredor que é declarado \gls{out} por correr fora do caminho da base.
					\end{enumerate}
					\item Um \gls{putout} é creditado ao receptor:

					\begin{enumerate}[label=\arabic*)]
						\item Quando é declarado um terceiro \gls{strike}.
						\item Quando o batedor não bate na ordem correta.
						\item Quando o batedor interfere na ação do receptor.
						\item Quando o batedor é declarado \gls{out} por ter batido ilegalmente.
						\item Quando o batedor é declarado \gls{out} em razão de uma tentativa de \gls{bunt} depois de dois \glspl{strike} ter resultado em \gls{foulball}.
						\item Quando o batedor é declarado \gls{out} por usar um \gls{bat} ilegal ou Adulterado.
						\item Quando o batedor é declarado \gls{out} por ter mudado de um \gls{battersbox} a outro.
					\end{enumerate}
				\end{enumerate}

			\item A quinta coluna deve indicar a quantidade de assistências ("assists") dadas nas eliminações por cada defensor. Deve ser creditada uma assistência

				\begin{enumerate}[label=(\alph*)]
					\item A cada jogador que maneja a bola em qualquer série de jogadas que resulte na eliminação do corredor. Deve ser atribuída somente uma assistência -- não mais -- a um jogador que maneja a bola em qualquer eliminação. Um jogador que tenha auxiliado numa Jogada de Perseguição (\gls{run-down play}) ou outra jogada do gênero pode ser creditado com um "assist" e um \gls{putout}.
					\item A cada jogador que maneja, ou lança, a bola de tal maneira que poderia ter contribuído na eliminação de um corredor se não ocorresse um erro subsequente de um companheiro de equipe.
					\item A cada jogador que, desviando uma bola batida, ajuda na eliminação de um corredor.
					\item A cada jogador que maneja a bola numa jogada que resulte na eliminação de um corredor, por Interferência, ou por correr fora da linha de base.
				\end{enumerate}
			\item A sexta coluna deve indicar a quantidade de erros cometidos por cada jogador.

			Erros são registrados nas seguintes situações:

				\begin{enumerate}[label=(\alph*)]
				\item A cada jogador que executa uma má jogada (\gls{misplay}) que prolongue o turno do batedor, ou a vida de um corredor que está ocupando alguma base.
				\item Ao defensor que deixa de tocar a base após receber a bola para eliminar o corredor, numa \gls{jogadaforcada} ou quando esse corredor é obrigado a retornar à base.
				\item Ao receptor, se um batedor é autorizado a ir à primeira base por Obstrução.
				\item Ao defensor que deixa de completar uma Jogada Dupla ("Double Play") por ter derrubado a bola.
				\item Ao defensor, se um corredor avança uma base em razão da falha desse defensor em parar ou tentar parar uma bola lançada corretamente a uma base, contanto que tenha havido motivo para o lançamento. Quando mais de um jogador poderia ter recebido o lançamento, o anotador tem de determinar a quem atribuir o erro.
			\end{enumerate}

		\end{enumerate}


\section{NÃO DEVEM SER REGISTRADOS \textit{basehitS} (BATIDAS INDEFENSÁVEIS)}

		 Não deve ser anotado um \gls{basehit} nos seguintes casos:

		\begin{enumerate}[label=(\alph*)]
			\item  Quando um corredor é declarado \gls{out} em \gls{jogadaforcada}, ou teria sido declarado \gls{out} em \gls{jogadaforcada} se um defensor não tivesse cometido erro.
			\item Quando um jogador que pega uma bola batida elimina um corredor precedente com um esforço normal.
			\item Quando um defensor tenta, mas não consegue eliminar um corredor precedente e, na opinião do anotador, o batedor-corredor poderia ter sido declarado \gls{out} na primeira base.
			\item Quando um batedor-corredor chega a salvo (\gls{safe}) à primeira base porque um corredor precedente é declarado \gls{out} por ter interferido numa bola batida ou na ação de um jogador da defensiva.

		 EXCEÇÃO: Se, na opinião do anotador, o batedor teria chegado a salvo (\gls{safe}) à primeira base se não tivesse ocorrido a Interferência, a ele deve ser creditado um \gls{safehit} (batida por meio da qual o batedor-corredor chega a salvo à primeira base).
		\end{enumerate}

	\section{\textit{FLY} DE SACRIFÍCIO}

		É anotado um \gls{fly} de sacrifício quando, com menos de dois \glspl{out},

			\begin{enumerate}[label=(\alph*)]
			\item  O batedor empurra um ponto com um \gls{fly} que é pego no ar; ou
			\item Um corredor anota ponto depois que um defensor do campo externo (ou um defensor do campo interno que tenha corrido para o campo externo) derruba um \gls{fly} ou \gls{line drive} após ter tocado a bola, e, na opinião do anotador, esse corredor poderia ter pisado o \gls{homeplate}, mesmo que o defensor a tivesse pegado no ar.
		\end{enumerate}

	\section{PONTOS EMPURRADOS (\textit{RUN BATTED IN})}

		São os pontos anotados por meio de:

			\begin{enumerate}[label=(\alph*)]
				\item  Um \gls{safehit}.
				\item Um \gls{bunt} de sacrifício ou um "slap hit" (SOMENTE AR), ou um \gls{fly} de sacrifício (AR e AL).
				\item Um \gls{foul fly} pego no ar.
				\item Um \gls{infieldputout} (eliminação feita no campo interno) ou \gls{fielder'schoice} (opção feita por um defensor na hora de executar uma jogada).
				\item Ocorrências que forçam um corredor a avançar para \gls{home}: por causa de Obstrução, porque o batedor é atingido por uma bola arremessada (\gls{hitbypitch}), ou em razão da concessão de uma base por \glspl{ball}.
				\item Um \gls{homerun} e todos os pontos anotados como decorrência desse \gls{homerun}.
			\end{enumerate}

	\section{ARREMESSADOR CREDITADO COM UMA VITÓRIA}

		 Um arremessador será creditado com uma vitória, nas seguintes situações:
			\begin{enumerate}[label=(\alph*)]
				\item  Quando, na condição de arremessador abridor, tiver arremessado pelo menos quatro \glspl{inning}, e sua equipe, que estava ganhando no momento em que foi substituído, permanecer liderando o placar pelo resto do jogo.
				\item Quando, numa partida encerrada depois de jogados cinco \glspl{inning}, o arremessador abridor tiver arremessado pelo menos três \glspl{inning}, e sua equipe tiver anotado mais pontos do que a outra no momento em que a partida é dada por terminada.
			\end{enumerate}

	\section{ARREMESSADOR DEBITADO COM UMA DERROTA}

		 Um arremessador será debitado com uma derrota, independentemente do número de \glspl{inning} que tenha arremessado, se for substituído quando sua equipe está perdendo e, depois disso, ela não consegue empatar ou tomar a dianteira no placar.

	\section{RESUMO DO JOGO}

		O resumo deve relacionar os seguintes itens, nesta ordem:
		\begin{enumerate}[label=(\alph*)]
			\item A quantidade de pontos por \gls{inning} e a contagem final.
			\item Os pontos empurrados (\gls{runbattedin}) e quem os empurrou.
			\item \glspl{hit} de duas bases (\glspl{two-base hit}) e quem os bateu.
			\item \glspl{hit} de três bases (\glspl{three-base hit}) e quem os bateu.
			\item \glspl{homerun} e quem os bateu.
			\item \glspl{fly} de sacrifício (\glspl{sacrificefly}) e quem os bateu.
			\item Jogadas Duplas (\glspl{doubleplay}) e os jogadores que delas participaram.
			\item Jogadas Triplas (\glspl{tripleplay}) e os jogadores que delas participaram.
			\item Quantidade de \glspl{walk} (\gls{baseonballs}) concedidos por cada arremessador.
			\item Quantidade de batedores declarado \gls{out}s por \gls{strike} (\glspl{strikeout}) por cada arremessador.
			\item Quantidade de \glspl{hit} e pontos permitidos por cada arremessador.
			\item O nome do arremessador vencedor.
			\item O nome do arremessador perdedor.
			\item O tempo de duração do jogo.
			\item Os nomes dos árbitros e anotadores.
			\item Bases roubadas (\glspl{stolenbase}) e por quem.
			\item \glspl{bunt} de sacrifício.
			\item Os nomes dos batedores atingidos por uma bola arremessada e dos arremessadores que os atingiram.
			\item A quantidade de \glspl{wild pitch} feitos por cada arremessador.
			\item A quantidade de \glspl{passedball} feitos por cada receptor.
		 \end{enumerate}

	\section{BASES ROUBADAS (SOMENTE AR)}

			Deve-se creditar uma base roubada (\gls{stolenbase}) a um corredor, sempre que ele avança uma base sem a ajuda de um \gls{hit}, um \gls{putout}, um erro, uma eliminação forçada, um \gls{fielder'schoice}, um \gls{passedball}, um \gls{wild pitch} ou um arremesso ilegal. Isso inclui um batedor-corredor que avança à segunda base numa base por \glspl{ball} concedida.

	\section{DADOS DE JOGOS CONFISCADOS (\textit{FORFEITED GAMES})}

 	Todos os dados de um jogo confiscado devem ser incluídos nas anotações oficiais, exceto aquele registro de arremessador ganhador/perdedor.

\chapter*{GLOSSÁRIO}

No Softball existem muitos termos que são utilizados pelos seus praticantes, árbitros e até anotadores que não são de conhecimento geral e podem causar um certo desconforto para os árbitros e espectadores que não estão familiarizados com eles. Com o intuito de esclarecer alguns destes termos, que podem ser até termos japoneses devido a influência nipônica e sua importância na prática deste esporte no Brasil, foi criado este glossário básico.

 2B -- Double: um batedor ganha um 2B quando a batida que ele deu proporcionou avanço de duas bases.

 3B -- Triple: um batedor ganha um 3B quando a batida que ele deu proporcionou avanço de três bases.

 Assists: termo utilizado para designar uma assistência de outro defensor para que o assistido consiga eliminar fisicamente um adversário no jogo.

 APP -- Appearance: é creditado uma appearance todas vezes que o pitcher entra em um determinado jopgo

Appeal Plays (Jogadas de Apelação): O time na defensiva tem direito a apelar de algumas jogadas que não tiveram a regra correta aplicada por uma árbitro no entendimento do técnico. Estas apelações servem para alertar o árbitro de infrações que poderiam ser permitidas sem serem apeladas. Nota. Não existem apelações para jogadas de decisão (Ball, strike, safe, out, foul, fair)

 AVG -- Batting Average: é um índice conseguido dividindo-se a quantidade de batidas conseguidas por um batedor pela quantidade de vezes que esteve batendo. É uma das mais comentadas estatísticas neste jogo. Expressa entre zero ( .000 ) e um ( 1.000).

 Ball: é a bola "ruim", que é arremessada fora da "zona de strike". A cada quatro bolas ruins lançadas pela arremessadora, o time adversário ganha o direito de avançar uma base.

 Boro: é o termo japonês para "Bola" ou \gls{ball}.

 Bor Baco: é o termo japonês para \gls{ballback} ou a informação dada pelo árbitro para recolherem as bolas, pois o jogo irá iniciar.

 Bat: taco, bastão.

 Bata: é o termo japonês para "Bastão" ou \gls{bat}.

 Batter's Box (Área do Batedor): um campo regular tem duas áreas de batedores. À esquerda e à direita da placa de home, de onde ele deve se posicionar para bater a bola.

 BB -- Base on Balls: é uma concessão de bases. Ao batedor é permitido avançar à primeira base mediante 4 arremessos inválidos cedidos pelo técnico do time na defensiva. Geralmente ocorre quando o batedor é um bom batedor e time está receoso de tomar vários pontos.

 Blocked Ball: bola desviada por algum material estranho espalhado no campo, bola rebatida, lançada ou arremessada que fica alojada na cerca.

 Bola Morta: diz-se bola morta quando por algum motivo ou necessidade o árbitro paralisa o jogo, as jogadas não podem ocorrer, e os jogadores de ataque devem permanecer nas suas bases. Uma bola rebatida fora do campo é bola morta; uma interferência, é bola morta.

 Bola Viva: quando a bola não está invalidada. As jogadas e eliminações podem ocorrer e os atacantes podem, por sua conta e risco, tentar roubar bases.

 C -- Chances: representa o número de oportunidades que o jogador teve com o intuito de eliminar um adversário. É utilizado em estatísticas de jogo.
 Catcher: é o receptor, jogador do time da defesa que se posiciona atrás do rebatedor e apanha as bolas arremessadas e não são rebatidas.

 Catcher's Interference (Interferência do Catcher): se o catcher ou qualquer outro defensor interferir com o batedor durante um arremesso, é concedida a primeira base ao batedor. A interferência pode ser, por exemplo, encostar a luva no bastão do batedor na hora da batida.

 CH -- Changeup: é um tipo de arremesso mais lento do softball e faz com que a bola caia bem perto do \gls{homeplate}. É um arremesso de efeito.

 CI -- Catcher's Interference: quando um catcher (ou outro defensor) interfere com o batedor em qualquer momento em que ele está tentando realizar uma
 rebatida. Neste caso, o batedor ganha a primeira base. (de novo?)

 Collision at Home Plate (Colisão no Home): o corredor não deve de maneira nenhuma desviar do seu caminho direto para o Homebase. Mas o Catcher deve,
 se não tiver de posse da bola, abrir espaço para a passagem do corredor. Caso haja um contato físico entre os dois, caracteriza esta colisão.

 CS -- Caught Stealing: número de vezes que um jogador é eliminado (por toque) ao tentar roubar uma base.

 CSB -- Caught Stealing (Pitcher/Catcher): número de vezes que um determinado jogador foi eliminado por tentar roubar bases.

 CU -- Curveball: bola curva. É um arremesso de efeito.

 Diamond: diamante ou campo de jogo. Campo interno.

 Deto Boro: é o termo japonês para "bola morta".

 Double Play: é o ato de fazer dois eliminados durante a mesma jogada contínua.

 As jogadas duplas são relativamente comuns, pois podem ocorrer sempre que houver pelo menos um corredor em base e menos de dois outs.


 E -- Errors: um defensor é creditado um erro quando, no julgamento do anotador oficial, ele falhou ao tentar converter uma eliminação que um defensor normal conseguiria.

 ERA -- Earned Run Average: é o numero de pontos concedidos pelo arremessador dividido pelo número de innings jogados (o que depende da categoria). Importante que os pontos considerados são os pontos que foram concedidos sem ajuda de algum erro ou bolas passadas (não agarradas pelo catcher).

 Fair Ball -- Bola Válida: é uma bola rebatida que autoriza o batedor a tentar alcançar a primeira base.

 FA -- Fastball (Bola rápida): é um arremesso direto e rápido.

 First-base Coach: é o técnico ou jogador que fica na área de técnico ao lado da primeira base, e geralmente sinaliza jogadas aos corredores e batedores para orientá-los.

 FLD\% -- Fielding Percentage: indica quão frequentemente um defensor ou até um time consegue realizar jogadas corretas ao receber bolas arremessadas para realizar eliminações. Geralmente tem fórmula que a descreve: número total de putouts e assistências feitas por um defensor dividido pelo numero total de chances (putouts assists e erros).




 Foul Ball (Bola Inválida): é uma bola rebatida que não autoriza o batedor a tentar alcançar a primeira base.

 \gls{foultip}: uma bola batida que vai direto do bastão para as mãos do catcher e é legalmente capturada.

 Furay : é o termo japonês para \gls{fly} ou "bola aérea".

 GDP -- Ground into Double Play: ocorre quando um defensor agarra uma bola rasteira e consegue efetuar jogadas eliminando dois ou mais jogadores nas bases.

 Globo: é o termo japonês para \gls{glove} (luva).

 glove: Luva (Globo)

 Gorô: é o termo japonês para \gls{ground} ball, ou bola rasteira, no chão.

 Ground Ball (Gorô): uma bola rebatida ao chão que vai rolando aos defensores.



 Hit (Batida): acontece quando o batedor consegue acertar a bola e esta cai em território válido, dentro do campo.

 Hito: é o termo japonês para \gls{hit}.

 Home Plate: formalmente designada nas regras como \gls{home}, é a base final que um jogador deve tocar para marcar ponto.

 Home run: jogada em que o rebatedor lança a bola para fora do limite do campo, acima da cerca de proteção, sem que esta toque no chão. Desta forma, o jogador é capaz de percorrer as quatro bases numa mesma corrida, marcando um ponto para a sua equipe

 HR -- Home Run- ocorre quando um batedor consegue acertar uma bola para fora de campo, ou correr as quatro bases sem ser eliminado.

 IP -- Illegal Pitch: ocorre quando o arremessador efetua alguma manobra proibida antes, durante ou após o arremesso. Existem várias possibilidades para um arremesso ilegal e o árbitro deve estar atento a regra.

% \gls{jogadaforcada}: é uma situação em que um corredor de base é compelido (ou forçado, obrigado) a desocupar sua base por conta de outro corredor que está chegando, e assim tentar avançar a salvo para a próxima base.

 Kuniguê: é aquela jogada que no terceiro strike (termo japonês isturaiku) o catcher deixa a bola escapar e o corredor corre para a primeira base. Nota: kuiniguê é comer e fugir (sem pagar a conta).

% Lana: é o \gls{runner}, é o corredor.

 Line Ball: uma bola rebatida em linha reta para dentro do terreno de jogo.

 Obstruction (Obstrução): é considerado o ato de um defensor que não está de posse da bola ou no processo de pegá-la impede o avanço de qualquer corredor.

 Passed Ball: bola defensável que passa para trás do receptor.

 PB -- Passed Ball (Catcher): é um termo estatístico que determina a quantidade de bolas lançadas ao catcher que ele não consegue segurar e como resultado deste erro um jogador consegue avançar uma base.

 Pickle: sanduíche -- é quando um corredor fica correndo entre bases para evitar ser tocado por um defensor que está de posse da bola | a rundown.

 Pickoff Play: jogada em que o arremessador tenta segurar o corredor na base, ou eliminar o corredor que está fora da base.

 Pitcher: arremessador, jogador da defesa que faz os lançamentos ao batedor.

 PO -- Put Outs: um defensor é creditado com um \gls{putout} quando ele, fisicamente, consegue eliminar um jogador do time adversário (quer seja por
 tocá-lo com a bola ou pisando uma base com a posse da bola em jogadas forçadas, ou até mesmo catando um terceiro strike). Também usado em
 estatísticas de jogo.

 R -- Run (Corridas ou Pontos): ocorre quando o corredor consegue cruzar a base principal (home) e marcar um ponto.

 RBI -- Run Batted In: é creditado estatísticamente a um batedor na maioria dos casos nos quais ele vai aparecer no bater box pra bater e pelo menos um ponto é conseguido. Existem algumas exceções: por exemplo, ele não ganha um RBI quando o ponto for resultado de um erro de um defensor.

 Rise: é uma bola de efeito na qual o arremessador da um efeito de giro na bola fazendo com que a mesma "suba" na hora que o batedor iria efetuar a batida, provocando o erro dele.

% Runner (Corredor/Lana): nomenclatura usada para denominar o rebatedor que, depois de bater na bola, chega salvo a uma das bases (seja ela a primeira, a segunda ou a terceira). Sua nova função consiste em correr para alcançar o maior número possível de bases enquanto o seu time estiver no ataque (rebatendo a bola).


 Sado: é o termo japonês para \gls{third}. Geralmente designa o defensor que joga na posição F5 ou na terceira base.

 Sanchin: é o termo japonês para \gls{strikeout}, ou o ato de um batedor ser eliminado pelo terceiro strike virado no qual não acerta a bola.

 SB -- Stolen Base: número de vezes que um jogador consegue roubar uma base.

 SBA -- Stolen Bases Allowed (Pitcher and Catcher): número de vezes que um jogador conseguiu "avançar bases" sem ser eliminado e sem que houvesse uma
 jogada ocorrendo (bases roubadas).

 Secano: é o termo japonês para \gls{second}. Geralmente designa o defensor que joga na posição F4 ou na segunda base.

 SF -- Sacrifice Fly: ocorre quando um batedor acerta uma batida com bola aérea para fora do campo interno, que pode ser facilmente pega por um defensor, mas que permite um corredor marcar ponto (após retocar a base depois da catada).

 SH -- \gls{foultip}: ocorre quando um jogador consegue marcar pontos por uma batida no campo interno ("Bunt") feita por um batedor com este intuito.

 SHO -- \gls{shutout}: um arremessador é premiado com um \textit{Shutout} quando ele entra para arremessar e arremessa o jogo inteiro pelo seu time sem permitir que
 adversário consiga marcar nenhum ponto.

 Shotto: é o termo japonês para \gls{shortstop}. Geralmente designa o defensor que joga na posição F6 ou na interbases (entre a segunda e terceiras bases).

 SO -- Strike Out: ocorre quando o arremessador consegue eliminar o batedor por uma combinação de 3 viradas de bastão ou 3 strikes determinados pelo árbitro de \gls{home}.

 SO -- Strike Out: representa a quantidade de vezes que o batedor foi eliminado pelo 3 strike (quer seja apenas olhando ou virando o bastão).

 Southpaw: é um arremessador canhoto.

 Squeeze Play: é uma manobra que consiste em um sacrifício com um corredor na terceira base. O batedor bate na bola, esperando ser eliminado na primeira base, mas proporcionando ao corredor na terceira base uma oportunidade para marcar um ponto.

 Strike: o mesmo que "bola boa". É o arremesso válido feito pela defesa, que não é rebatido pelo ataque.

 Suberi: é o termo japonês para \gls{slide}, é quando o corredor tenta entrar na base escorregando com a perna esticada e tentando se esquivar de um possível toque do defensor da base.

 TB -- \glspl{totalbase}: quantidade de bases conquistadas por um batedor através de suas batidas.

 Terreno \gls{fair}: é definido como a área do campo de jogo entre as duas linhas laterais que definem o campo de jogo, e inclui as próprias linhas e os postes delimitadores.

 Terreno \gls{foul}: é definido como qualquer área fora do campo de jogo.

 Triple Play: é o ato raro de fazer três eliminações durante durante a mesma jogada contínua. Um \gls{foultip} agarrado pelo catcher é considerado um 3o strike, portanto conta como\gls{strikeout}.



 Zona de Strike: para que um arremesso seja considerado válido, a bola precisa se manter na chamada zona de strike, um retângulo imaginário de mais ou menos 35 centímetros de largura e cuja altura se mede do joelho até axilas do rebatedor. A bola arremessada fora desta área é considerada "ruim". A análise dos arremessos é feita por um juiz que fica posicionado atrás da receptora do time que está defendendo (e do rebatedor).
